\documentclass[../appunti.tex]{subfiles}

\begin{document}

\section{Successioni}
\subsection{Successioni in $\R$}
Sia $X \neq \varnothing$, una qualsiasi funzione $f: \N \to X$ si dice: 
\bt{successione in $X$}. \newline
Una notazione si indica $ \{ f_n \}_{n \in \N} $ o $f_1, f_2, \dots , f_n $ \newline
$ f_n $ si chiama termine n-esimo. \newline
$ k_1, k_2, \dots , k_n $ è una successione di numeri naturali:

\begin{equation}
	k_1 < k_2 < \dots < k_n < k_{n+1} < \dots \quad \forall n \in \N
\end{equation}
La successione $ \{ f_{k_n} \} $ è una \tit{sottosuccessione} di $ \{ f_n \} $.

\begin{defn}
Se $ a_n$ tende a $ l \in \R$ per $n \to \infty$, si dice che 
$ \lim \limits_{n \to \infty} a_n = l $

\begin{equation}
	\forall \epsilon > 0 \exists \ol{n}: \forall n \in \N
	( n > \ol{n}\ \Rightarrow\ | a_n - l | < \epsilon )
\end{equation}
$ \{ a_n \} $ converge ad $l$ ed esso è il limite di $ \{ a_n \} $ 
\end{defn}

\begin{esem}
\begin{equation}
	\lim \limits_{ n \to \infty} \frac{1}{n} = 0
\end{equation}
Ovvero

\begin{equation}
	\forall \epsilon > 0 \exists \ol{n}: \forall n \in \N
	\left( n > \ol{n}\ \Rightarrow\ | \frac{1}{n} - 0 | < \epsilon \right)
\end{equation}
\end{esem}

\begin{dimo}[Il limite se esiste è unico]
\begin{equation}
	\lim \limits_{x \to \infty} a_n = l \quad \land \quad 
	\lim \limits_{x \to \infty} a_n = m \quad
	\iff \quad l\ =\ m
\end{equation}
\end{dimo}

\begin{esem}
Poniamo per assurdo che $l \neq m$
Fissiamo $\epsilon > 0$
\begin{equation}
	\underbrace{
	\underbrace{| a_n - l | < \frac{\epsilon}{2} }_{ n>\overline{n_1} }
	\quad \& \quad 
	\underbrace{| a_n - m | < \frac{\epsilon}{2} }_{ n>\overline{n_2}} }_{
	n\ >\ \max \{ \overline{n_1}, \overline{n_2} \} }
\end{equation}

\begin{equation*}
	\big\Downarrow
\end{equation*}
Ricordiamo che $ | a_n - m | = | m - a_n |$

\begin{equation}
	| \cancel{-a_n} - l - \cancel{-a_n} + m |
	| a_n - l |\ +\ |m - a_n | < 
	\frac{\epsilon}{2} + \frac{\epsilon}{2} = \epsilon
\end{equation}


\begin{equation*}
	\big\Downarrow
\end{equation*}

\begin{equation}
	| m - l | < \epsilon \quad \Longrightarrow \quad | m - l | = 0
\end{equation}
Ma questo è assurdo perchè: $\epsilon > 0, \forall \epsilon \in \R$

\begin{equation}
	m = l 
\end{equation}
\end{esem}






\end{document}
