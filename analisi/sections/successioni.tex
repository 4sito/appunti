\documentclass[../appunti.tex]{subfiles}

\begin{document}

\section{Successioni}
\subsection{Successioni in $\R$}
Sia $X \neq \varnothing$, una qualsiasi funzione $f: \N \to X$ si dice: 
\bt{successione in $X$}. \newline
In notazione si indica $ \{ f_n \}_{n \in \N} $ o $f_1, f_2, \dots , f_n $ \newline
$ f_n $ si chiama termine n-esimo. \newline
$ k_1, k_2, \dots , k_n $ è una successione di numeri naturali:

\begin{equation}
	k_1 < k_2 < \dots < k_n < k_{n+1} < \dots \quad \forall n \in \N
\end{equation}
La successione $ \{ f_{k_n} \} $ è una \tit{sottosuccessione} di $ \{ f_n \} $.\\
Il limite di una successione $ \{ a_n \} = l $. Vale a dire che $ l \in \R $
è un numero vicino ai termini della successione. Esso è più precisamente un 
\bt{numero reale} tale che \tit{comunque si scelga} un intervallo di numeri intorno
ad $a$.
\bda
\begin{equation}
	\underbrace{( a - \epsilon, a + \epsilon)}_{\mbox{un intervallo attorno a 
	$l$}} , \epsilon > 0\ |\ \exists 
	\underbrace{\ol{n}}_{\mbox{un indice $n$ t.c.}} n > \ol{n}.
\end{equation}
$a_n$ si trova in questo \tit{intorno}

\begin{defn}[Successione]
Una \tit{successione} è una legge che ad ogni numero \bt{naturale} $n$ 
fa corrispondere\bt{uno ed uno solo} numero \tit{reale} $a_n$.

\begin{equation}
	\{ a_n \}_{n \in \N} = a_1, a_2, a_3, \dots , a_n
\end{equation}
Una successione è una funzione che collega degli \tit{indici $n$} a dei numeri 
\tit{ reali $ a \in \R $ }
\end{defn}


\begin{defn}
Se $ a_n$ tende a $ l \in \R$ per $n \to \infty$, si dice che 


\begin{equation}
	\begin{aligned}
 	\lim \limits_{n \to \infty} a_n &= l\\
	\big\Downarrow\\
	\forall \epsilon > 0,\ \exists \ol{n}:\
	&( n > \ol{n}\ \Rightarrow\ | a_n - l | < \epsilon )\\
	\big\Downarrow\\
	| a_n - l | &< \epsilon
	\end{aligned}
\end{equation}

$ \{ a_n \} $ converge ad $l$ ed esso è il \bt{limite} di tale \bt{tale successione}
\end{defn}

\begin{esem}
\begin{equation}
	\lim \limits_{ n \to \infty} \frac{1}{n} = 0
\end{equation}
Ovvero

\begin{equation}
	\forall \epsilon > 0 \exists \ol{n}: \forall n \in \N
	\left( n > \ol{n}\ \Rightarrow\ | \frac{1}{n} - 0 | < \epsilon \right)
\end{equation}
\end{esem}

\begin{dimo}[Il limite se esiste è unico]
\begin{equation}
	\lim \limits_{x \to \infty} a_n = l \quad \land \quad 
	\lim \limits_{x \to \infty} a_n = m \quad
	\iff \quad l\ =\ m
\end{equation}
\end{dimo}

\begin{esem}
Poniamo per assurdo che $l \neq m$
Fissiamo $\epsilon > 0$
\begin{equation}
	\underbrace{
	\underbrace{| a_n - l | < \frac{\epsilon}{2} }_{ n>\overline{n_1} }
	\quad \& \quad 
	\underbrace{| a_n - m | < \frac{\epsilon}{2} }_{ n>\overline{n_2}} }_{
	n\ >\ \max \{ \overline{n_1}, \overline{n_2} \} }
\end{equation}

\begin{equation*}
	\big\Downarrow
\end{equation*}
Ricordiamo che $ | a_n - m | = | m - a_n |$

\begin{equation}
	| \cancel{-a_n} - l - \cancel{-a_n} + m |
	| a_n - l |\ +\ |m - a_n | < 
	\frac{\epsilon}{2} + \frac{\epsilon}{2} = \epsilon
\end{equation}


\begin{equation*}
	\big\Downarrow
\end{equation*}

\begin{equation}
	| m - l | < \epsilon \quad \Longrightarrow \quad | m - l | = 0
\end{equation}
Ma questo è assurdo perchè: $\epsilon > 0, \forall \epsilon \in \R$

\begin{equation}
	m = l 
\end{equation}
\end{esem}





\begin{defn} 
Se $\{ a_n \}_{n \in \N} \subseteq \R$ converge ad $l \in \R$ \bt{ogni}
sua sottosuccessione $ \{ a_{k_n} \}_{n \in \N}$ converge ad $ l $
\end{defn}

\input{preamble}

\begin{document}
\maketitle

\begin{dimo}[Limiti]

Se $ \{ a_n \}_{n \in \N}$ converge $ l \in \R $
$\quad \Longrightarrow \quad \{ a_{k_{n}} \}_{k_{n \in \N}}$ 
converge $ l \in \R $

\begin{equation*}
	\big\Downarrow
\end{equation*}
Si ha che:

\begin{equation}
	\forall \epsilon > 0\ \exists \overline{n} \in \N\ :\ 
	n > \overline{n}\ \Longrightarrow\ | a_n - l | <  \epsilon 
\end{equation}


\begin{equation}
	\forall \epsilon > 0\ \exists \overline{n} \in \N\ :\ 
	n > \overline{n}\ \Longrightarrow\ | a_{k_n} - l | <  \epsilon 
\end{equation}

\begin{equation}
	\lim \limits_{n \to \infty}a_{k_n}\ =\ l
\end{equation}
\end{dimo}

\begin{esem}
\begin{equation}
	\lim \limits_{n \to +\infty} \frac{1}{n} = 0 \qquad \& \qquad 
	k=2, \lim \limits_{k_n \to +\infty} \frac{1}{k_n}= 0 
\end{equation}
\end{esem}

\end{document}



\begin{eser}

\documentclass[../rip.tex]{subfiles}

\begin{document}

\begin{dimo}

\begin{equation}
	\lim \limits_{n \to +\infty}\underset{(a_n + b_n)} = l + m
\end{equation}

\begin{equation}
	\lim \limits_{n \to +\infty} a_n = l \quad \& \quad
	\lim \limits_{n \to +\infty} b_n = m
\end{equation}

\begin{equation*}
	\big\Downarrow
\end{equation*}

\begin{equation}
	| a_n - l | < \frac{\epsilon}{2} \quad \mbox{se}\quad  n > \overline{n_1}
\end{equation}
\begin{equation}
	| b_n - m | < \frac{\epsilon}{2} \quad \mbox{se}\quad  n > \overline{n_2}
\end{equation}
$n> \max\{ \overline{n_1}, \overline{n_2} \}$

\begin{equation}
	| a_n + b_n - l - m | \leq | a_n - l | + | b_n - m | < 
	\frac{\epsilon}{2} + \frac{\epsilon}{2} = \epsilon
\end{equation}

\begin{equation*}
	\big\Downarrow
\end{equation*}

\begin{equation}
	\forall \epsilon > 0, \exists \overline{n} \equiv \max \{
	\overline{n_1}, \overline{n_2} \} \ :\ n > \overline{n}\ 
	\Rightarrow\ \underbrace{| (a_n + b_n) - ( l + m ) |}_{0} < \epsilon
\end{equation}

\begin{equation}
	( a_n + b_n ) - ( l + m ) = 0
\end{equation}
\begin{equation}
	a_n + b_n = l + m
\end{equation}

\end{dimo}

\end{document}


\end{eser}

\documentclass[../rip.tex]{subfiles}

\begin{document}

\begin{dimo}

\begin{equation}
	\forall \epsilon > 0, \exists \overline{n} \in \N\ :\ n > \overline{n}\ 
	\Rightarrow\ \underbrace{| a_n - l < \epsilon |}_{
	l - \epsilon < a_n < l + \epsilon \quad \forall n > \overline{n}}
\end{equation}

\begin{equation*}
	\epsilon = | l |
\end{equation*}
Da questo otteniamo che

\begin{equation}
	\underbrace{ l - | l | }_{0} < a_n < 
	\underbrace{ l + | l | }_{2l}
\end{equation}
In conclusione avremo che: \newline
se $ l > 0 \Rightarrow a_n > 0 $ \newline
se $ l < 0 \Rightarrow a_n < 0 $

\end{dimo}

\end{document}


\input{preamble}

\begin{document}
\maketitle

\begin{defn}[Teorema dei 2 carabinieri]

Se $ \underbrace{ \{ a_n \}\ , \{ b_n \} }_{\mbox{convergono a $\ l$}} , \{ c_n \}$

\begin{equation}
	\mbox{è ovvio che:}\ a_n \leq c_n \leq b_n \ \Longrightarrow\ 
	c_n \mbox{converge a}\ l
\end{equation}
\end{defn}

\begin{dimo}
\begin{equation}
	\forall \epsilon > 0, \exists \overline{n_1}, \overline{n_2} \in \N :
\end{equation}

\begin{equation*}
	\big\Downarrow
\end{equation*}

\begin{equation}
	l - \epsilon < a_n < l + \epsilon \qquad \& \qquad
	l - \epsilon < b_n < l + \epsilon
\end{equation}
se $ n > \max \{ \overline{n_1}, \overline{n_2} \} $

\begin{equation*}
	\big\Downarrow
\end{equation*}

\begin{equation}
	l - \epsilon < a_n \leq c_n \leq b_n  < l + \epsilon \qquad
	\forall n > \overline{n}
\end{equation}

\begin{equation}
	\underbrace{ l - \epsilon < c_n < l + \epsilon }_{
	| c_n - l | < \epsilon } \Longrightarrow 
	\lim \limits_{n \to +\infty} c_n = l
\end{equation}

\end{dimo}
 
\end{document}


\begin{defn}
Sia una successione $ \{ a_n \}_n \subseteq \R $ è detta:

\begin{itemize}
	\item \tit{superiormente limitata}, se 
		$ \exists M \in \R\ :\ a_n \leq M\ \forall n \in \N $
	\item \tit{inferiormente limitata}, se
		$ \exists M \in \R\ :\ a_n \geq M\ \forall n \in \N $
	\item \tit{limitata}, se 
		$ \exists M \in \R\ :\ | a_n | \leq M\ \forall n \in \N $ 
\end{itemize}
\end{defn}

\begin{defn}[Ogni successione convergente è limitata]
Sia $\{ a_n \}_{n \in \N} \subseteq \R,\ a_n \underset{n \to \infty}{\to} l$ \\
Allora (con $\epsilon = 1$) 

\begin{equation}
	\exists \ol{n}: \forall n \in \N
        ( n > \ol{n}\ \Rightarrow\ | a_n - l | < 1 )
\end{equation}
Segue quindi che $ | a_n | \leq | a_n - l | + | l | < 1 + | l |, \ n > \ol{n} $ 

\begin{equation}
	| a_n | \leq 1 + | l |
\end{equation}

\begin{defn}[Retta reala ampliata] 
	\begin{equation}
		\ol{\R} := \R \cup \{ + \infty \} \cup \{ - \infty \}
	\end{equation}
\end{defn}

\begin{defn}
Sia $ \{ a_n \}_{n \in \N} \subseteq \R $

\begin{equation}
	\begin{aligned}
	\lim \limits_{n \to + \infty} & a_n = + \infty \\	
	&\big\Downarrow \\
	\forall k \in \R \exists \ol{n} \in N\ &:\
	\forall n \in \N ( n > \ol{n} \implies a_n > k)
	\end{aligned}
\end{equation}
La scrittura è analoga per $ - \infty $ invertendo il segno: $ ( a_n < k) $ \\
Potremo dire che $ a_n $ \tit{diverge positivamente o negativamente}
\end{defn}

\subsection{Forme indeterminate}
Se $ \{ a_n \}, \{ b_n \} \subseteq \R$ e 
$ \{ a_n \} \to + \infty, \{ b_n \} \to - \infty \}$ allora:

\begin{equation} 
	a_n + b_n \to + \infty - \infty =\ ?
\end{equation}
$ + \infty$ e $ - \infty$ non sono veri e propri numeri, piuttosto sono dei 
\bt{simboli}, quindi il risultato sarà detto:\textsc{forma indeterminata 
$ +\infty - \infty$}

\end{defn}
Altri tipi di forme indeterminate sono:
	 
\begin{equation} 
	\frac{\infty}{\infty},\  \frac{0}{0},\ 
	0 \cdot \infty,\ 1^{\infty},\
	0^0,\ \infty^0
\end{equation}

\subsection{Teoremi generali di esistenza}
Una successione $\{ a_n \}_{n \in \N} \subseteq \R$ è detta monotona crescente se 


\begin{equation}
	a_n \leq a_{ n + 1 },\  \forall n \in \N 
\end{equation}
Si dice invece monotona decrescente se


\begin{equation}
	a_n \geq a_{ n + 1}, \forall n \in \N
\end{equation}
Sono rispettivamente \bt{strettamente} monotone crescenti o decrescenti 
se le disuguaglianze sono \bt{strette} $ ( <, >) $\\
Le scritture $ a_n \nearrow$ e $ a_n \searrow $ indicano monotonia crescente e decrescente  

\begin{defn}[Successioni costanti]
Se $ a_n = a \ \forall n \in \N$, con $a$ numero reale fissato si dice che 


\begin{equation}
	\{ a_n \}_{n \in \N} = l,\ l \in \R \quad 
	\{ a_n \} \nearrow \searrow = \mbox{costante}
\end{equation}

\end{defn}

\begin{defn}
Ogni successione monotona ammette limite: \\
Se $ \{ a_n \}_{n \in \N} \subseteq \R $:

\begin{enumerate}
	\item $ a_n \nearrow \implies \lim \limits_{n \to + \infty} a_n = 
	\sup_{n \in \N} a_n$
	\item $ a_n \searrow \implies \lim \limits_{n \to + \infty} a_n =
	\inf_{n \in \N} a_n $
\end{enumerate}
\end{defn}

\begin{dimo}
Se $ \{ a_n \}$ è superiormente limitata per l'assioma di completezza:

\begin{equation}
	\exists \underset{n \in \N}{\sup a_n} = \lambda
\end{equation}
Per la proprietà del $ \sup$ si ha che $ a_n \leq \lambda, \forall n \in \N$ dunque:

\begin{equation}
	a_n < \lambda + \epsilon \ \forall n \in \N, \ \forall \epsilon > 0
\end{equation}

\begin{equation}
	\forall \epsilon > 0, \exists \ol{n} \in \N\ :\
	\lambda < a_{\ol{n}} + \epsilon
\end{equation}
La definizione di limite è:

\begin{equation}
	\lim \limits_{n \to + \infty} a_n = \lambda
\end{equation}
\end{dimo}

\begin{eser}[Il numero di nepero $ e $]

\begin{equation}
	e \equiv \lim \limits_{n \to + \infty} \left( 1 + \frac{1}{n} \right)^n
\end{equation}
Si nota che $a_n = \left( 1 + \frac{1}{n} \right)^n$ e $ b_n =
\left( 1 + \frac{1}{n} \right)^{n+1}$ sono successioni \bt{convergenti} che hanno lo stesso limite $ e $, inoltre sono \bt{strettamente monotone} 

\begin{equation}
	a_n < a_{n + 1} \quad \mbox{e} \quad b_n > b_{n+1} \ \forall n \in \N
\end{equation}
Inoltre 

\begin{equation}
	a_n < b_n\ \forall n \in \N 
\end{equation}
allora: 

\begin{equation}
	a_n < a_p < b_p < b_m \quad \forall n,m,p; p = \max \{n, m\}
\end{equation}
Entrambe le successioni convergono: $a_n$ è monotona crescente e superiormente limitata e $b_n$ è monotona decrescente e inferiormente limitata.

\begin{equation}
	\lim \limits_{n \to + \infty} \frac{b_n}{a_n} = 
	\lim \limits_{n \to + \infty} \left( 1 + \frac{1}{n} \right) = 1
\end{equation}
Questo implica che:

\begin{equation} 
	\lim \limits_{n \to + \infty} b_n = 
	\lim \limits_{n \to + \infty} a_n = e
\end{equation}

\end{eser}



\begin{dimo}

\begin{equation}
	b_n = \bn \quad \& \quad b_{n+1} = \bnn
\end{equation}

\begin{equation}
	\begin{aligned}
	& \frac{b_n}{b_{n+1}} > 1 \implies
	\frac{\bn}{\bnn} > 1 \quad \forall n \in \N \\
	&=\left( 1 + \frac{1}{n} \right)^{n+1} \left( 1 + \frac{1}{n} \right) =
	\left( \frac{n+1}{n}^n \right) \left( \frac{n+1}{n} \right) >1 \\
	&=\left( 1 + \frac{1}{n+1} \right)^{n+2} \left( 1 + \frac{1}{n+1}\right)^2 =
	\left( \frac{n+2}{n+1}^n \right) \left( \frac{n+2}{n+1} \right)^2 >1 \\
	&=\left( \frac{ \cancel{(n+1)} (n+2)}{n \cancel{(n+1)}} \right)^n 
	\cdot \left( \frac{\cancel{n+1}}{n} \right) \cdot 
	\left( \frac{n+2}{\cancel{n+1}} \right)^2 >1\\
	&=\left( \frac{n+2}{n} \right)^n \cdot \left( \frac{n+2}{n} \right)
	\cdot \left( \frac{n+2}{n+1} \right) >1 \\
	&=\underbrace{\left( \frac{n+2}{n} \right)^{n+1}}_{>1} > 
	\underbrace{\left( \frac{n+2}{n+1} \right)}_{<1}\\
	\end{aligned}
\end{equation}	



\end{dimo}



\begin{defn}[Bolzano - Weierstrass]
Ogni successione reale limitata ammette una sottosuccessione convergente.
\end{defn}
\begin{dimo}
	Per ogni $ \{ a_n \}_{n \in \N} \subseteq \R $ esiste 
	$M > 0\ :\ | a_n | \leq M,\ \forall n \in \N\ 
	\exists k_n \nearrow\ :\ a_{k_n} \underset{n \to +\infty}{\to} l \in \R $  
	
	\begin{equation}
	-M \leq a_n \leq M\ \forall n \in \N
	\end{equation}

	\begin{equation}
	\alpha_n = \sup{a_k : k \geq n},\ n \in \N \quad \implies \quad
	-M \leq \alpha_n \leq M\ \forall n \in \N
	\end{equation}
	Quindi dalla definizione ne segue che:

	\begin{equation}
		\begin{aligned}
		\alpha_{n+1} \leq \alpha_n\ \forall n \in \N & 
		\Rightarrow \alpha_n \searrow\\	
		&\big\Downarrow
		\end{aligned}
	\end{equation}

	\begin{equation}
		\exists \lim \limits_{n \to + \infty} \alpha_n \equiv l
		\quad \implies l \equiv \inf_{n \in \N} \alpha_n
	\end{equation}

	\begin{equation}
		\begin{aligned}
		\forall \epsilon > 0,\ \forall p \in \N\ 
		&\exists n \geq p: l - \epsilon \leq a_n\\
		\alpha_p\searrow \Rightarrow l \leq \alpha_p \Rightarrow
		& l - \epsilon < \alpha_p\ \forall \epsilon > 0\ \forall p
		\end{aligned}
	\end{equation}
Dato che $\alpha_p = \sup\{ a_n : n \geq p\}$, deve esistere 
$n \geq p : a_n > l - \epsilon$ \\
Sia $ k_n : \N \to \N$ definita per ricorrenza:

\begin{equation}
	\left\{ 
	\begin{aligned}
	& k_1 = \min\{ k \in \N: l -1 < a_k\} \\
	& k_{n+1} = \min\{ k \in \N: k > k_n \land l - \frac{1}{n+1} < a_k \}
\end{aligned} \right.
\end{equation}
	
\begin{equation*}
	\big\Downarrow
\end{equation*}

\begin{equation}
	k_{n+1} > k_n, \ \forall n \quad \land \quad 
	l - \frac{1}{n} < a_{k_n}\ \forall n
\end{equation}
Questo implica che $ \{ a_{k_n} \}_{n \in \N}$ verifica le disuguaglianze

\begin{equation}
	l - \frac{1}{n} < a_{k_n} \leq \alpha_{k_n}\ \implies\ 
	\alpha_{k_n} \underset{n \to + \infty}{\to} l \implies\
	a_{k_n} \to l
\end{equation}

\end{dimo}

\begin{defn}[Successioni di Cauchy]
	Una successione $ \{ a_n \}_{n \in \N} \subseteq \R $ si chiama successione di Cauchy se:

\begin{equation}
	\forall \epsilon > 0,\ \exists \ol{n} \in \N:\ \forall n,m \in \N 
	(n,m > \ol{n} \Rightarrow | a_n - a_m | < \epsilon )
\end{equation}
Una successione si dice di Cauchy se i suoi termini sono "arbitrariamente" vicini tra loro.
\end{defn}

\begin{defn}[Ogni successione convergente è di Cauchy]

\begin{equation}
	\{ a_n \}_{n \in \N} \subseteq \R, a_n \to l \in \R = \mbox{di Cauchy}
\end{equation}
\end{defn}

\begin{dimo} 
Se $ \lim \limits_{n \to + \infty} a_n = l \in \R $ implica che:

\begin{equation}
	\forall \epsilon, \epsilon > 0 \exists \ol{n} \forall n 
	\left( n > \ol{n} \Rightarrow | a_n - l | < \frac{\epsilon}{2} \right)
\end{equation}
La scrittura $ \exists \ol{n} $ significa che esiste un indice dopo il quale 
ogni indice successivo sarà maggiore di quello.\\
Di conseguenza:

\begin{equation}
	| a_n - a_m | = | (a_n - l) + (l - a_m) \leq 
	| a_n - l | + | a_m - l | \leq \cancel{2} \cdot \frac{\epsilon}{\cancel{2}}
	\quad n,m > \ol{n}
\end{equation}
$ \{ a_n \} $ è di Cauchy 
\end{dimo}

\begin{equation}
	\{ a_n \} \mbox{ di Cauchy} \Rightarrow \{ a_n \} \nearrow \quad \iff
 	\quad \{ a_n \}  \nearrow \Rightarrow  \{a_n \}  \mbox{ di Cauchy}
\end{equation}
\subsection{Rappresentazione decimale di numeri reali}
Se $ x \in \R $ è:

\begin{equation}
	[ x ] = \mbox{parte intera} = max\{ p \in \Z: p < x \} 
\end{equation}
\bda

\begin{equation}
	[ x ] \leq x < [ x ] + 1\ \forall x \in \R 
\end{equation}
\bda 

\begin{equation}
	x_n = \frac{[b^nx]}{b^n}
\end{equation}
Le seguenti affermazioni sono vere:

\begin{enumerate}
	\item $\{ x_n \} \nearrow $ 
	\item $x_n \leq x < x_n + \frac{1}{b_n}\ \forall n \in \N $
	\item $\lim \limits_{n \to + \infty} x_n = x $
	\item $\exists \alpha_0 \in \Z, \exists \{ \alpha_n \}_{n \in \N} \subseteq \Z $
\end{enumerate}


\begin{defn}[Decimali]
I numeri decimali sono i numeri razionali:

\begin{equation}
	\frac{m}{10^n} ( m \in \Z, n \in \N )
\end{equation}
Ogni numero decimale si può scrivere come 

\begin{equation}
	x = \alpha_0 + \frac{\alpha_1}{10} + \frac{\alpha_2}{10^2} + \dots + \frac{\alpha_n}{10^n}
\end{equation}
con $ \alpha_0 \in \Z,\ \alpha_1 \alpha_2, \dots, \alpha_n \in \{ 0,1,2,\dots,9 \}$

\end{defn}

\begin{defn}[Decimali propri]
Sia $ x \in \R,\ x = \alpha_0, \alpha_1, \dots, \alpha_n, \dots$.
La rappresentazione decimale di $x$ si dice \bt{propria} se:

\begin{equation}
	\nexists p \in \N : \alpha_n = 9\ \forall n \geq p
\end{equation}
\end{defn}
Ogni numero reale ammette un'\bt{unica} rappresentazione decimale propria.\\
Se $ x \in \R\ \Rightarrow\ x = \alpha_0, \alpha_1, \dots, \alpha_n, \dots$
è la rappresentazione decimale propria di x \bt{se e solo se}

\begin{equation}
	\alpha_0, \alpha_1, \dots, \alpha_n \leq x < 
	\alpha_0, \alpha_1, \dots, \alpha_n + \frac{1}{10^n}\ \forall n \in \N
\end{equation}

\subsection{Cardinalità di insiemi}
Due insiemi $ A, B \neq \varnothing $ si dicono \bt{equipotenti} se 

\begin{equation}
	\exists f: A \underset{1 - 1}{\overset{su}{\to}} B \quad
	\implies \quad A \cong B
\end{equation}
Vale a dire che esiste una \bt{funzione biunivoca} fra i due insiemi ed essi hanno stessa
\bt{cardinalità}

\begin{equation}
	card(A)=card(B)
\end{equation}
\bda

\begin{equation}
	I_n = \{ k \in \N : 1 \leq k \leq n \}, \quad card(A) \cong card(I_n)
	\implies card(A) = n
\end{equation}
$A$ è un insieme finito. Un iniseme è infinito se non è finito. 

\begin{itemize}
	\item $A$ è finito $ B \subseteq A, B \neq \varnothing \implies B$ è finito
	\item $A$ è finito e $B$ è sottoinsieme proprio di $A \implies A \cong B$
	\item $A$ è finito, allora il numero dei suoi elementi è unico
	\item $B$ è infinito e $ B \subseteq A \implies A$ è infinito
\end{itemize}
Altre proposizione che ne conseguono sono:

\begin{itemize} 
	\item $A \neq \varnothing \implies A \cong A$
	\item $A \cong B \iff B \cong A$
	\item $ A \cong B, B \cong C \implies A \cong C$
\end{itemize}
L'equipotenza è una relazione di \bt{equivalenza}

\begin{defn}[ $ \N $ è infinito]
Dimostriamo che $ \N $ è equivalente ad un suo sottoinsieme proprio: 

\begin{equation} 
	P = \{ n \in \N : = 2m, n \in \N \}, \
	f: P \to \N, f(n) = \frac{n}{2}
\end{equation}
$f$ è biunivoca quindi $ P \subset \N \implies P \cong \N $ 
\end{defn}

\begin{defn}[ $ \Z, \Q, \R$ sono infiniti]
Tutti questi insiemi contengono $ \N $ 
\end{defn}

\begin{defn}[Insiemi numerabili]
	Un insieme si dice \bt{numerabile} se è \bt{equipotente} ad $\N$
\end{defn}
Un insieme $A$ è numerabile se si possono \tit{elencare i suoi elementi}:\\
ovvero se esiste una successione biiettiva $\{ a_n \}_{n \in \N}$ che ha come
immagine $A$, il nome di tale successione è \bt{numerazione}

\begin{defn}
Sia $A$ un insieme numerbaile se $ M \subseteq A$, $M$ è infinito\\
$\implies M \cong \N$
\end{defn}
Ogni sottoinsieme infinito di un insieme numerabile \bt{è numerabile}.\\
\bt{Ogni} sottoinsieme di $\N$ è un insieme \bt{infinito o numerabile}.
\begin{defn}[Assioma della scelta]
Sia $\mathcal{B}$ una famiglia $ \neq \varnothing$ di insiemi. Sia $A$ un insieme t.c.

\begin{equation}
	B \subseteq A \ \forall B \in \mathcal{B}
\end{equation}
\bda

\begin{equation}
	\exists \varphi: \mathcal{B} \to A\ :\ \varphi(B) \in B \ 
	\forall B \in \mathcal{B} \Rightarrow \bt{AC}
\end{equation}
Ovvero in parole: data una \bt{famiglia} di insiemi $\mathcal{B}$ non vuoti, esiste
una funzione che ad ogni insieme della famiglia fa corrispondere un suo elemento.
\end{defn}
L'assioma assicura che, quando viene data una collezione di insiemi non vuoti si 
può sempre costruire un nuovo insieme \tit{"scegliendo"} un \bt{singolo elemento}
da ciascuno di quelli di partenza.

\begin{defn}
Dati due insiemi $A,B$:
\begin{equation}
	card(A) \leq card(B)
\end{equation}
Se esiste $B_0 \subseteq B$, t.c. $card(A) = card(B_0)$. Se $card(A) \leq card(B)$
e $card(A) \neq card(B) \implies card(A) < card(B)$
\end{defn}
Un insieme si dice finito \bt{se e solo se}: $ card(A) < card(\N)$


\begin{defn}[Numeri algebrici]
Un numero reale si dice \bt{algebrico} se risolve un'equazione 

\begin{equation}
	p(x) = 0, \qquad p \in \Z 
\end{equation}
con $p$ un polinomio con coefficienti in $\Z$.\\
I nuemri reali \bt{non} algebrici si dicono \bt{trascendenti}
\end{defn}

\begin{dimo}
I numeri algebrici sono i razionali , infatti essi sono:


\begin{equation}
	x = \frac{m}{n}\ \implies 
	nx - m = 0
\end{equation}


\begin{equation}
	p(x) = 3x^7 - 5x^2 + 3 \to h = 7 + 3 + | - 5 | + 2 + 3 = 20
\end{equation}

\end{dimo}


\begin{defn}[Gerarchia di infiniti]
Esiste una gerarchi di \bt{infiniti}, ovvero "certi infiniti valgono di più di altri
infiniti" 


\begin{equation}
	\mathcal{N}_0 = card(\N), \mathcal{N}_1 = card(\R)\ \implies\
	\mathcal{N}_0 < \mathcal{N}_1
\end{equation}
\end{defn}

\subsection{O grande, o piccolo, $\sim$ equivalente}

\begin{defn}[o piccolo]
Siano $ \{ a_n \}_{n \in \N},\ \{ b_n \}_{n \in \N} \subseteq \R $ due successioni
reali, si dice che $a_n$ è un \tit{"o piccolo"} di $b_n$ per $n \to + \infty$

\begin{equation}
	a_n = o(b_n) \ ( n \to + \infty) \quad \iff \quad
	\lim \limits_{n \to + \infty} \frac{a_n}{b_n} = 0
\end{equation}
\end{defn}

\begin{defn}[O grande]
Siano $ \{ a_n \}_{n \in \N},\ \{ b_n \}_{n \in \N} \subseteq \R $ due successioni
reali, si dice che $a_n$ è un \tit{"O grande"} di $b_n$ per $n \to + \infty$

\begin{equation}
	a_n = O(b_n)\ (n \to + \infty) \quad \iff \quad
	\exists \ol{n} \in \N, \exists M \in \R\ :\ \left(
	\left|\frac{a_n}{b_n}\right| \leq M\ \forall n > \ol{n} \right)
\end{equation}
Occore notare che se esiste il limite di $ \frac{a_n}{b_n} $


\begin{equation}
	\limm \frac{a_n}{b_n} = l \in \R
\end{equation}
Si può sempre scrivere $ a_n = O(b_n)$
\end{defn}

\begin{defn}[ $ \sim $ equvialente]
Siano $ \{ a_n \}_{n \in \N},\ \{ b_n \}_{n \in \N} \subseteq \R $ due successioni
reali, si dice che $a_n$ è un \tit{"equivalente"} di $b_n$ per $n \to + \infty$


\begin{equation}
	a_n \sim b_n \ (n \to + \infty) \quad \iff \quad
	\limm \frac{a_n}{b_n} = 1
\end{equation}
Questo vale a dire che le due successioni hanno lo stesso limite.\\ 
\end{defn}
Consideriamo anche che, se due successioni sono equivalenti $ a_n \sim b_n $ implica
che :


\begin{equation} 
	a_n \sim b_n\ \implies \ a_n = O(b_n)\ \iff\ b_n=  O(a_n) 
\end{equation}


\end{document}
