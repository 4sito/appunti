\documentclass[../analisi.tex]{subfiles}

\begin{document}

\section{Limiti per funzioni}
\subsection{Definizioni}

\begin{defn}[Intorno di $\pm \infty$] 
\bt{L'intorno} di $ + \infty$ è l'insieme:


\begin{equation}
	]k, + \infty [,\ k \in \R
\end{equation}
\bt{L'intorno} di $ - \infty$ è l'insieme:


\begin{equation}
	] - \infty, k [, \ k \in \R
\end{equation}
\end{defn}
Intendiamo che se $ \acr $ come i seguenti


\begin{equation}
	\begin{aligned}
	sup A &= + \infty\ \iff\ + \infty \in D(A)\\
	inf A &= - \infty\ \iff\ - \infty \in D(A)
	\end{aligned}	
\end{equation}
Un limite $\lambda$ può essere:


\begin{equation}
	\lambda \in \ol{R}\ \iff\ 
	\begin{cases}
		\lambda \in R\\
		\lambda = +\infty\\
		\lambda = -\infty
	\end{cases}
\end{equation}

\begin{defn}
Siano $\acr,\ x_0 \in D(A),\ \lambda \in \ol{R}.\ f: A \to \R$.\\
$A$ un insieme contenuto fra i numeri reali, 
$x_0$ si trovi in un punto di accumulazione di $A$, 
$\lambda$ contenuto nella retta reale \tit{estesa} e $f$ una funzione che ad ogni 
elemento di $A$ corrisponde un elemento di $R$.\\
Diremo che $f(x)$ tende a $\lambda$ per $x$ che tende a $x_0$.


\begin{equation}
	\forall V \in \mathcal{U}_\lambda\ \exists W \in \mathcal{U}_{x_0}\ :\
	f(x) \in V\ \forall x \in A \setminus \{ x_0 \}\cap W
\end{equation}
Qualunque intorno $V$ all'interno della famiglia degl intorni $\mathcal{U}_\lambda$
di $\lambda$ (ricordando che $\lambda \in \ol{R}$ quindi può assumere o
un valore reale o è uguale a  $ \pm \infty$), esiste un intorno $W$ all'interno
della famiglia degli insiemi di $\mathcal{U}_{x_0}$ di $x_0$ (un punto di 
accumulazione) 

\begin{equation}
	x_0 \in \R \qquad ] x_0 - \delta, x_0 + \delta [
\end{equation}
 tale che l'immagine $x$  sia in $V$ tutte le volte che $x$ in $A \setminus
 {x_0}$ intersecato con l'intorno $W$, $W_{x_0}$ è per definzione un punto
 di accumulazione quindi l'interesezione \bt{è non vuota}\\
La scrittura semplificata:


\begin{equation}
	\limxo f(x) = \lambda
\end{equation}
se \bt{per ogni} $ \epsilon > 0 $ esiste un numero $\delta > 0$ tale che per ogni
$ x \in A, \ 0 < | x - x_0 | < \delta$.

\begin{equation}
	| f(x) - \lambda | < \epsilon 
\end{equation}
\end{defn}

\begin{esem}
Abbiamo $x_0 \in \R$ e $ \lambda = - \infty$ avremo che:

\begin{equation}
	\begin{aligned}
	\mathcal{U}_{x_0} &= \{ ] x_0 - \delta, x_0 + \delta[ : \delta > 0\}\\
	\mathcal{U}_{\lambda} &= \mathcal{U}_{- \infty} = \{
	] -\infty, k [: k \in \R \}	
	\end{aligned}
\end{equation}
Quindi avremo che:

\begin{equation}
	\limxo f(x) = - \infty
\end{equation}
se e solo se:

\begin{equation}
	\forall k \in \R\ \exists \delta > 0 :\ \forall x \in A \setminus \{
	x_0 \} \ \{ | x - x_0 | < \delta\ \implies\ f(x) < k\}
\end{equation}
Vale a dire che: qualunque valore noi diamo a $k$ che è un numero appartenente 
ai numeri reali (quindi ha un valore finito), esiste un  numero $\delta$ maggiore di
0 tale che, quale che sia $x$ contenuta in $A$ meno $x_0$, il \bt{modulo} della
differenza di $x$ e $x_0$ ($| x - x_0 |$) è \bt{minore} di $\delta$, questo vuol
dire che l'immagine di $f(x)$ è sempre \bt{strettamente minore} di k.\\
Ovvero $f(x)$ avrà sempre un valore piccolissimo inferiore a qualsiasi numero reale

\end{esem}

Anche se $f(x)$ è definita nel punto $x_0$ non è necessario
che  soddisfare $\limxo f(x) = \lambda$, quindi nel punto in cui 
$x = x_0$. Affermiamo che il valore del limite $\lambda$ è indipendente 
dal valore della funzione nel punto $x_0$.
%TODO stuff i didnt understand at page 86/87


\subsection{Teoremi Fondamentali}%
\label{sub:teoremi_fondamentali}


\begin{defn}[Unicità del limite]
Sia $\acr$ e siano $x_0 \in D(A),\ x_0 \in \ol{R},\ f: A \to \R$. \\
Se esistono $ \lambda, \mu \in \ol{\R}$ t.c.:


\begin{equation}
	\limxo f(x) = \lambda \quad \land \quad  \limxo f(x) = \mu \quad 
	\iff \quad \lambda = \mu
\end{equation}
\end{defn}
Prendiamo un sottoinsieme $A$ di $\R$, che è il dominio di una funzione $f(x): A 
\to \R$, e prendiamo $x_0$ un punto di accumulazione dell'insieme $A$, contenuto 
in $\R$. Supponiamo che ci siano \bt{due} limiti  (per assurdo), ma allora questi 
due limiti \bt{SONO UGUALI}.

\begin{dimo}
Partiamo dalla nozione che: \bt{punti distinti} ammettono \bt{intorni disgiunti}:


\begin{equation}
	\lambda, \mu \in \ol{\R} \qquad \implies \qquad
	\exists V \in \mcl{U}_{\lambda}, W \in \mcl{U}_{\mu}:\ V \cap W \neq
	\varnothing
\end{equation}
Supponendo per assurdo che i due numeri siano diversi fra loro, allora 
ci basterà prendere un numero $\epsilon > 0$ minore della \bt{metà} della distanza
fra questi due numeri e noteremo che i due numeri hanno intorni distinti.\\
Questo è assurdo in quanto avevamo affermato che l'interesezione dei loro 
intorni era \bt{non vuota}. Questo quando i due numeri $\lambda,\ \mu$ sono 
numeri reali, quando uno di questi due invece é $\pm \infty$ allora per
l'assioma di completezza c'è sempre un numero reale che si trova fra questi due per
seprararli.\\
\linebreak
Per assurdo \Lightning, poniamo $\lambda \neq \mu$, per quanto abbaimo dimostrato
prima abbiamo che \bt{punti disgiunti} hanno \bt{intorni disgiunti}. Ovvero:

\begin{equation}
	\exists V \in \mcl{U}_{\lambda}, \exists W \in \mcl{U}_{\mu}\ :\
	V \cap W = \varnothing.
\end{equation}
Stiamo affermando che l'interesezione dei due intorni è un'insieme vuoto.\\
Notiamo anche che:

\begin{equation}
	\limxo f(x) = \lambda \ \iff\ \exists U_1 \in \mcl{U}_{x_0} :
	f(x) \in V\ \quad \forall x \in (A \setminus \{x_0\} ) \cap U_1
\end{equation}
e che

\begin{equation}
	\limxo f(x) = \mu \ \iff\ \exists U_2 \in \mcl{U}_{x_0} :
	f(x) \in W\ \quad \forall x \in (A \setminus \{x_0\} ) \cap U_2
\end{equation}
Quindi avremo che:

\begin{equation}
	f(x) \in V \cap W\ \forall x \in ( A \setminus \{ x_0 \}) \cap (U_1 \cap
	U_2)
\end{equation}
Questo é assurdo perchè $U_1 \cap U_2 \in \mcl{U}_{x_0}$ e 

\begin{equation}
	x_0 \in D(A) \iff (A \setminus \{ x_0 \} ) \cap ( U_1 \cap U_2 ) \neq
	\varnothing
\end{equation}
e questo dimostra che $V \cap W \neq \varnothing$ che contraddice il fatto che
$V \cap W = \varnothing$

\end{dimo}


\begin{defn}[Località del limite]
Siano $ \acr, x_0 \in D(A)$ intendendo che $x_0 \in \ol{\R}$ e $f,g: A \to \R$
e siano $f, g$ due funzioni con dominio l'insieme $A$.\\
Se esiste $\tilde{W} \in \mcl{U}_{x_0}$:

\begin{equation}
	f(x) = g(x)\ \forall x \in \tilde{W} \cap (A \setminus \{ x_0 \})
\end{equation}
se esiste il $\limxo f(x)$, quindi esiste $\limxo g(x)$ e i due limiti coincidiono.
\end{defn}


\begin{esem}
Prendiamo due funzioni $f, g: \R \to \R$ con:


\begin{equation}
	g(x) =
	\begin{cases}
		1 &\mbox{se} x \neq 0\\
		0 &\mbox{se} x = 0
	\end{cases}
\end{equation}
e 

\begin{equation}
	f(x) = 1
\end{equation}
Possiamo subito osservare che $f(x) \neq g(x)$ se e solo se $x = 0$ \\
Inoltre il limite di:

\begin{equation}
	\lim \limits_{x \to 0} g(x) = 1 \qquad g(x) = 1 \ \forall x \in \R
	\setminus \{0\}
\end{equation}
Infatti non è importante il valore della funzione in quello specifico punto ma
il valore che assume \bt{nell'intorno} di quel punto.\\
Abbiamo dimostrato quindi che $\limxo f(x) = \limxo g(x)$.
\end{esem}
\bt{il limite è unico e dipende solo da un intorno del punto in cui lo calcoliamo
e non dal valore assoluto della funzione nel punto}.

Il vaolre assoluto avrà un valore importante nella nozione di continuità.


\begin{defn}[Restrizione di limiti]
Sia $f: A \to \R$ e $B \subseteq A$ diremo che la sua restrizione per $B$ è:


\begin{equation}
	f_{|_B}: B \to \R \qquad f_{|_B}(x) = f(x)^{\forall x \in B}
\end{equation}
I punti della funzione ristretta in $B$ sono tutti i punti della funzione di
di partenza restrigendo il Dominio da $A$ a $B$.\\
Se esiste il limite di $\limxo f(x)$ allora esite anche il limite
di $\limxo f_{|_B} (x)$ e questi due limiti coincidono:

	\begin{equation}
		\limxo f(x) = \limxo f_{|_B} (x)
	\end{equation}
\end{defn}


\begin{defn}
Se $A = \R \setminus \{ x_0 \}, \ B = ] x_0, +\infty [ $ oppure $B = ] -\infty
, x_0[$ \bt{limite destro} e \bt{limite sinistro} di $f$ in $x_0$ si definiscono:


\begin{equation}
	\lim \limits_{x \to x_0^+} f(x) = \limxo f_{|_{]x_0, +\infty[}}(x) \qquad
	\land \qquad
	\lim \limits_{x \to x_0^-} f(x) = \limxo f_{|_{]-\infty, x_0[}}(x)
\end{equation}
\end{defn}
Con $x_0^+$ intendiamo un numero poco più grande di $x_0$


\begin{defn}
La definizione di limite destro è:

\begin{equation}
	\forall \epsilon > 0\ \exists \delta > 0: ( x_0 < x < x_0 + \delta\
	\Longrightarrow\ |f(x) - \lambda| < \epsilon)
\end{equation}
per il limite sinistro serve una sostituzione ovvero:

\begin{equation}
	x_0 - \delta < x < x_0
\end{equation}
\end{defn}
Da quest'ultima definizione otteniamo che se $f(x)$ ha limite per $ \xtx$ allora 
esiste il limite per $\xtx^+$ e per $\xtx^-$\\
Serve a precisare anche che se i limiti destro e sinistro di una 
funzione sono diversi fra loro in uno specifico punto $x_0$ allora
il limite di $f(x)$ in $x_0$ non esiste


\begin{defn}[Teorema del collegamento]
Siano $\acr, x_0 \in D(A), \lambda, x_0 \in \ol{\R}, f: A \to \R$
Abbiamo che:

\begin{equation}
	\limxo f(x) = \lambda \ \iff \lim \limits_{\nti} f(x_n) = \lambda\
	\forall \{ x_n \}_{n \in \N} \subseteq A \setminus \{ x_0 \},\
	x_n \underset{\nti}{\to} x_0
\end{equation}
Vuol dire che se abbiamo una funzione $f(x)$ con \bt{limite}, possiamo 
"collegare" \bt{ogni} successione contenuta nel dominio e se usiamo 
come immagine della funzione un qualsiasi valore della successione, avremo come
limite di questa nuova funzione il limite iniziale.\\
Questo teorema è fondamentale in quanto riconduce il limite di una funzione a 
quello di una successione.
\end{defn}

\begin{Large}
	\bt{IMPORTANTE:} il simbolo di $ \infty $ non preceduto da nessun segno non
	ha nessun valore e non dobbiamo utilizzarlo nella risoluzione degli 
	esercizi in quanto è un simbolo troppo \bt{impreciso e fuorviante}.
\end{Large}


\begin{defn}
Il limite della somma di due limiti è la somma dei due limiti.\\
Il limite del prodotto di due funzioni è il prodotto dei due limiti.\\
Abbiamo vari casi a seconda del valore dei limiti:\\
Poniamo $ \acr, f,g: A \to \R, x_0 \in D(A) \cap \ol{\R} $, allora:

\begin{enumerate}[label*=\arabic*.]

	\item	 
	\begin{enumerate}[label=\Alph*.]
	\item i limiti sono dei numeri reali:\\
		
		\begin{equation}
		\begin{aligned}
		f(x) \txtx &\lambda \in \R\\
		g(x) \txtx &\mu \in \R\\
			   &\Downarrow\\
		\mbox{SOMMA:}\ f(x)+g(x) \txtx& \lambda+\mu\\
		\mbox{PRODOTTO:}\ f(x)\cdot g(x) \txtx& \lambda \cdot \mu\\
		\mbox{QUOZIENTE:}\ \frac{f(x)}{g(x)} \txtx& 
		\frac{\lambda}{\mu} \qquad (\mu \neq 0; g(x) \neq 0)
		\end{aligned}
		\end{equation}
	\item i limiti sono $ \pm \infty $\\
		Se le due funzioni $f,g$ hanno segni concordi per $\xtx$, allora:\\ 
		il limite della somma è $ \pm \infty $\\
		il limite del prodotto è il prodotto dei limiti e il segno è 
		definito dal prodotto dei segni.
	\end{enumerate}

\item Se $f(x) \txtx \pm \infty, f(x) \neq 0 $ in $A\setminus\{x_0\}$:
		
	\begin{equation}
		\limxo \frac{1}{f(x)} = 0
	\end{equation}
	e viceversa:\\
	Se $f(x) \txtx 0, f(x) > 0 $ in $A\setminus\{x_0\}$:

	\begin{equation}
		\limxo \frac{1}{f(x)} = \pm\infty
	\end{equation}

\item 
	\begin{enumerate}[label=\Alph*.]
	\item
	Se $f(x) \leq g(x),\ \forall x \in A \setminus \{x_0\},\ 
	\limxo f(x) = +\infty
	$ allora questo implica che $\limxo g(x) = +\infty$
	
	\item
	Se $f(x) \leq g(x),\ \forall x \in A \setminus \{x_0\},\ 
	\limxo f(x) = -\infty
	$ allora questo implica che $\limxo g(x) = -\infty$
	\end{enumerate}

\item $f(x) \txtx \lambda\ \implies |f(x)| \txtx |\lambda|$

	\end{enumerate}


\end{defn}


\begin{defn}[2 carabinieri]
	Poniamo $\acr, x_0 \in D(A)\cap\ol{\R};\ f,g,h: A \to \R.$
	A è un sottoinsieme di $\R$, $x_0$ è un punto di accumulazione di A, e
	$f,g,h$ sono delle funzioni con dominio A.\\
	Se esiste un intorno $W \in \mathcal{U}_{x_0} $ tale che:

	\begin{equation}
		f(x) \leq h(x) \leq g(x)\ \forall x \in (A \setminus \{ x_0 \}\cap W
	\end{equation}
	allora è vero che:
	\begin{equation}
		\limxo f(x) = \limxo g(x) = \lambda \in \ol{\R} \ \implies
		\ \limxo h(x) = \lambda
	\end{equation}
\end{defn}

\begin{dimo}
Per dimostrare questo teorema ci sono due casi differenti:

\begin{enumerate}
	\item $\lambda = \pm\infty$, la dimostrazione è banale in quanto basta 
		rivedere i lemmi derivati dai prodotti e le somme dei limiti.
	\item $\lambda \in \R$.\\
		Definiamo una successione $\{x_n\}_{n \in \N} \subseteq A
		\setminus \{x_0\},\ x_n \underset{\nti}{\to} x_0$ 
		convergente in $x_0$.\\
		Dal teorema del "collegamento" sappiamo che 

		\begin{equation}
			f(x_n) \underset{\nti}{\to} \lambda\ \&\
			g(x_n) \underset{\nti}{\to} \lambda
		\end{equation}
		Sappiamo inoltre che esiste un indice $\ol{n}$ per il quale
		$x_n \in (A \setminus \{ x_0 \}) \cap W;\ \forall n > \ol{n}$\\
		Quindi è vero che:

		\begin{equation}
			f(x_n) \leq h(x_n) \leq g(x_n)\ \forall n > \ol{n}
		\end{equation}
		Dal Teorema dei 2 Carabinieri (per successioni) otteniamo che
		$\lim \limits_{\nti} h(x_n) = \lambda$
\end{enumerate}
	
\end{dimo}
\begin{defn}[Cauchy]
Poniamo $\acr, x_0 \in D(A) \cap \ol{\R}, f: A \to \R.$\\
Allora sono equivalenti le seguenti:

\begin{enumerate}
	\item 
	\begin{equation}
		\exists \lambda \in \R: \limxo f(x) = \lambda
	\end{equation}
	\item 
	\begin{equation}
		\forall \epsilon > 0 \exists W \in \mathcal{U}_{x_0}:\
		\forall x,y \in A \setminus \{ x_0 \}\
		(x,y \in W \implies | f(x) - f(y)| < \epsilon
	\end{equation}
\end{enumerate}
\end{defn}
Questo teorema afferma che quando una funzione converge al proprio limite, esiste
un numero positivo ($\epsilon$) tale per cui esiste un intorno ($W$) per cui 
la differenza del modulo di due termini arbitrari ($x,y$) sono minori
di qualsiasi numero positivo ($\epsilon$) noi scegliamo.
\begin{dimo}

\begin{enumerate}
	\item $1 \implies 2$\\
	\begin{equation}
		\forall \epsilon > 0 \exists W \in \mathcal{U}_{x_0}:
		| f(x) - \lambda | < \frac{\epsilon}{2}\
		\forall x \in W \cap (A \setminus \{ x_0 \}
	\end{equation}
	Allora utilizzando un'applicazione della disuguaglianza triangolare:

	\begin{equation}
		| f(x) - f(y) | \leq | f(x) - \lambda + \lambda | \leq
		| f(x) - \lambda | + | \lambda - f(y) < \cancel{2} 
		\frac{\epsilon}{\cancel{2}} = \epsilon 
	\end{equation}
	
	\item $ 2 \implies 1 $\\
	Sia $ \{ x_n \} $ una successione convergente (e quindi di Cauchy)
	$ x_n \underset{\nti}{\to}
	x_0$, dalle nostre ipotesi sappiamo che:
	\begin{equation}
		\forall \epsilon > 0 \exists W \in \mathcal{U}_{x_0}:\
		\forall x,y \in A \setminus \{ x_0 \}\
		(x,y \in W \implies | f(x) - f(y)| < \epsilon
	\end{equation}
	Dato che la successione è convergente dai teoremi sulle successioni
	possiamo dedurre che esiste un indice $\ol{n}$ per il quale 
	$n,m > \ol{n}$ il che implica che:

	\begin{equation}
		| f(x_n) - f(x_m) | < \epsilon\ \forall n,m > \ol{n}
	\end{equation}
	Quindi la successione $ \{ f(x_n) \}_{n \in \N} $ è di Cauchy e 
	dall'assioma di completezza sappiamo anche che:

	\begin{equation}
		\exists \lambda \in \R :
		\lim \limits_{\nti} f(x_n) = \lambda
	\end{equation}
	Ovvero che esiste un numero reale che è il limite di questa successione.\\
	Per dimostrare che questa condizione è valida \bt{per ogni} successione
	convergente a $x_0$.\\
	Consideriamo un'altra successione $ \{ y_n \} \subseteq A \setminus 
	\{ x_0 \},\ y_n \underset{\nti}{\to} x_0 $. Allora 
	deve esistere un indice $\ol{\ol{n}} \in \N$ per cui ogni indice 
	maggiore di esso "cade" nell'intervallo $W$, ovvero:

	\begin{equation}
		n > \ol{\ol{n}}\ \implies y_n \in W \cap (A \setminus \{x_0\})
	\end{equation}
	Se allora consideriamo un indice $n$ tale che esso sia maggiore del
	massimo dei due altri indici implica che le due successioni si trovano
	all'interno dell'intervallo $W$:

	\begin{equation}
		n > max\{ \ol{n}, \ol{\ol{n}} \} \implies
		x_n, y_n \in (A \setminus \{ x_0 \} ) \cap W
	\end{equation}
	Dalla prima dimostrazione sappiamo che la differenza del modulo delle 
	due funzioni è più piccola di un numero arbitrariamente piccolo ($\epsilon$)

	Perciò:

	\begin{equation}
		\lim \limits_{\nti} ( f (x_n) - f(y_n) ) = 0
	\end{equation}
	Ma dato che il limite di $f(x_n)$ è $\lambda$ allora esso sarà
	il limite anche della funzione dell'altra successione:

	\begin{equation}
		\lim \limits_{\nti} f(y_n) = \lambda
	\end{equation}
	Questo è valido per qualsiasi arbitraria funzione $\{y_n\}$
\end{enumerate}
\end{dimo}


\subsection{Limiti di funzioni monotone}%
\label{sub:limiti_di_funzioni_monotone}



\begin{defn}
Poniamo $A \neq \varnothing, f: A \to \R$\\.
Si dice una funzione \bt{superiormente limitata} se è superiormente
limitato l'insieme $f(A)$.\\
Si dice una funzione \bt{inferiormente limitata} se è inferiormente
limitato l'insieme $f(A)$.\\
Più semplicemente diremo che esiste una costante ($M$) sempre maggiore (o minore)
dei valori della funzione nel suo dominio:

\begin{equation}
	\exists M \in \R: f(x) \geq M,\ \forall x \in A
\end{equation}

\begin{equation}
	\exists M \in \R: f(x) \leq M,\ \forall x \in A
\end{equation}
Si dice che una funzione è limitata se essa è minore del modulo di una costante
(M):

\begin{equation}
	\exists M \in \R: f(x) \leq |M|,\ \forall x \in A
\end{equation}


\end{defn}






\end{document}
