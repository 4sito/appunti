\documentclass[../analisi.tex]{subfiles}

\begin{document}

\section{Limiti per funzioni}
\subsection{Definizioni}

\begin{defn}[Intorno di $\pm \infty$] 
\bt{L'intorno} di $ + \infty$ è l'insieme:


\begin{equation}
	]k, + \infty [,\ k \in \R
\end{equation}
\bt{L'intorno} di $ - \infty$ è l'insieme:


\begin{equation}
	] - \infty, k [, \ k \in \R
\end{equation}
\end{defn}
Intendiamo che se $ \acr $ come i seguenti


\begin{equation}
	\begin{aligned}
	sup A &= + \infty\ \iff\ + \infty \in D(A)\\
	inf A &= - \infty\ \iff\ - \infty \in D(A)
	\end{aligned}	
\end{equation}
Un limite $\lambda$ può essere:


\begin{equation}
	\lambda \in \ol{R}\ \iff\ 
	\begin{cases}
		\lambda \in R\\
		\lambda = +\infty\\
		\lambda = -\infty
	\end{cases}
\end{equation}

\begin{defn}
Siano $\acr,\ x_0 \in D(A),\ \lambda \in \ol{R}.\ f: A \to \R$.\\
$A$ un insieme contenuto fra i numeri reali, 
$x_0$ si trovi in un punto di accumulazione di $A$, 
$\lambda$ contenuto nella retta reale \tit{estesa} e $f$ una funzione che ad ogni 
elemento di $A$ corrisponde un elemento di $R$.\\
Diremo che $f(x)$ tende a $\lambda$ per $x$ che tende a $x_0$.


\begin{equation}
	\forall V \in \mathcal{U}_\lambda\ \exists W \in \mathcal{U}_{x_0}\ :\
	f(x) \in V\ \forall x \in A \setminus \{ x_0 \}\cap W
\end{equation}
Qualunque intorno $V$ all'interno della famiglia degl intorni $\mathcal{U}_\lambda$
di $\lambda$ (ricordando che $\lambda \in \ol{R}$ quindi può assumere o
un valore reale o è uguale a  $ \pm \infty$), esiste un intorno $W$ all'interno
della famiglia degli insiemi di $\mathcal{U}_{x_0}$ di $x_0$ (un punto di 
accumulazione) 

\begin{equation}
	x_0 \in \R \qquad ] x_0 - \delta, x_0 + \delta [
\end{equation}
 tale che l'immagine $x$  sia in $V$ tutte le volte che $x$ in $A \setminus
 {x_0}$ intersecato con l'intorno $W$, $W_{x_0}$ è per definzione un punto
 di accumulazione quindi l'interesezione \bt{è non vuota}\\
La scrittura semplificata:


\begin{equation}
	\limxo f(x) = \lambda
\end{equation}
se \bt{per ogni} $ \epsilon > 0 $ esiste un numero $\delta > 0$ tale che per ogni
$ x \in A, \ 0 < | x - x_0 | < \delta$.

\begin{equation}
	| f(x) - \lambda | < \epsilon 
\end{equation}
\end{defn}

\begin{esem}
Abbiamo $x_0 \in \R$ e $ \lambda = - \infty$ avremo che:

\begin{equation}
	\begin{aligned}
	\mathcal{U}_{x_0} &= \{ ] x_0 - \delta, x_0 + \delta[ : \delta > 0\}\\
	\mathcal{U}_{\lambda} &= \mathcal{U}_{- \infty} = \{
	] -\infty, k [: k \in \R \}	
	\end{aligned}
\end{equation}
Quindi avremo che:

\begin{equation}
	\limxo f(x) = - \infty
\end{equation}
se e solo se:

\begin{equation}
	\forall k \in \R\ \exists \delta > 0 :\ \forall x \in A \setminus \{
	x_0 \} \ \{ | x - x_0 | < \delta\ \implies\ f(x) < k\}
\end{equation}
Vale a dire che: qualunque valore noi diamo a $k$ che è un numero appartenente 
ai numeri reali (quindi ha un valore finito), esiste un  numero $\delta$ maggiore di
0 tale che, quale che sia $x$ contenuta in $A$ meno $x_0$, il \bt{modulo} della
differenza di $x$ e $x_0$ ($| x - x_0 |$) è \bt{minore} di $\delta$, questo vuol
dire che l'immagine di $f(x)$ è sempre \bt{strettamente minore} di k.\\
Ovvero $f(x)$ avrà sempre un valore piccolissimo inferiore a qualsiasi numero reale

\end{esem}

Anche se $f(x)$ è definita nel punto $x_0$ non è necessario
che  soddisfare $\limxo f(x) = \lambda$, quindi nel punto in cui 
$x = x_0$. Affermiamo che il valore del limite $\lambda$ è indipendente 
dal valore della funzione nel punto $x_0$.
%TODO stuff i didnt understand at page 86/87


\subsection{Teoremi Fondamentali}%
\label{sub:teoremi_fondamentali}


\begin{defn}[Unicità del limite]
Sia $\acr$ e siano $x_0 \in D(A),\ x_0 \in \ol{R},\ f: A \to \R$. \\
Se esistono $ \lambda, \mu \in \ol{\R}$ t.c.:


\begin{equation}
	\limxo f(x) = \lambda \quad \land \quad  \limxo f(x) = \mu \quad 
	\iff \quad \lambda = \mu
\end{equation}
\end{defn}
Prendiamo un sottoinsieme $A$ di $\R$, che è il dominio di una funzione $f(x): A 
\to \R$, e prendiamo $x_0$ un punto di accumulazione dell'insieme $A$, contenuto 
in $\R$. Supponiamo che ci siano \bt{due} limiti  (per assurdo), ma allora questi 
due limiti \bt{SONO UGUALI}.

\begin{dimo}
Partiamo dalla nozione che: \bt{punti distinti} ammettono \bt{intorni disgiunti}:


\begin{equation}
	\lambda, \mu \in \ol{\R} \qquad \implies \qquad
	\exists V \in \mcl{U}_{\lambda}, W \in \mcl{U}_{\mu}:\ V \cap W \neq
	\varnothing
\end{equation}
Supponendo per assurdo che i due numeri siano diversi fra loro, allora 
ci basterà prendere un numero $\epsilon > 0$ minore della \bt{metà} della distanza
fra questi due numeri e noteremo che i due numeri hanno intorni distinti.\\
Questo è assurdo in quanto avevamo affermato che l'interesezione dei loro 
intorni era \bt{non vuota}. Questo quando i due numeri $\lambda,\ \mu$ sono 
numeri reali, quando uno di questi due invece é $\pm \infty$ allora per
l'assioma di completezza c'è sempre un numero reale che si trova fra questi due per
seprararli.\\
\linebreak
Per assurdo \Lightning, poniamo $\lambda \neq \mu$, per quanto abbaimo dimostrato
prima abbiamo che \bt{punti disgiunti} hanno \bt{intorni disgiunti}. Ovvero:

\begin{equation}
	\exists V \in \mcl{U}_{\lambda}, \exists W \in \mcl{U}_{\mu}\ :\
	V \cap W = \varnothing.
\end{equation}
Stiamo affermando che l'interesezione dei due intorni è un'insieme vuoto.\\
Notiamo anche che:

\begin{equation}
	\limxo f(x) = \lambda \ \iff\ \exists U_1 \in \mcl{U}_{x_0} :
	f(x) \in V\ \quad \forall x \in (A \setminus \{x_0\} ) \cap U_1
\end{equation}
e che

\begin{equation}
	\limxo f(x) = \mu \ \iff\ \exists U_2 \in \mcl{U}_{x_0} :
	f(x) \in W\ \quad \forall x \in (A \setminus \{x_0\} ) \cap U_2
\end{equation}
Quindi avremo che:

\begin{equation}
	f(x) \in V \cap W\ \forall x \in ( A \setminus \{ x_0 \}) \cap (U_1 \cap
	U_2)
\end{equation}
Questo é assurdo perchè $U_1 \cap U_2 \in \mcl{U}_{x_0}$ e 

\begin{equation}
	x_0 \in D(A) \iff (A \setminus \{ x_0 \} ) \cap ( U_1 \cap U_2 ) \neq
	\varnothing
\end{equation}
e questo dimostra che $V \cap W \neq \varnothing$ che contraddice il fatto che
$V \cap W = \varnothing$

\end{dimo}


\begin{defn}[Località del limite]
Siano $ \acr, x_0 \in D(A)$ intendendo che $x_0 \in \ol{\R}$ e $f,g: A \to \R$
e siano $f, g$ due funzioni con dominio l'insieme $A$.\\
Se esiste $\tilde{W} \in \mcl{U}_{x_0}$:

\begin{equation}
	f(x) = g(x)\ \forall x \in \tilde{W} \cap (A \setminus \{ x_0 \})
\end{equation}
se esiste il $\limxo f(x)$, quindi esiste $\limxo g(x)$ e i due limiti coincidiono.
\end{defn}


\begin{esem}
Prendiamo due funzioni $f, g: \R \to \R$ con:


\begin{equation}
	g(x) =
	\begin{cases}
		1 &\mbox{se} x \neq 0\\
		0 &\mbox{se} x = 0
	\end{cases}
\end{equation}
e 

\begin{equation}
	f(x) = 1
\end{equation}
Possiamo subito osservare che $f(x) \neq g(x)$ se e solo se $x = 0$ \\
Inoltre il limite di:

\begin{equation}
	\lim \limits_{x \to 0} g(x) = 1 \qquad g(x) = 1 \ \forall x \in \R
	\setminus \{0\}
\end{equation}
Infatti non è importante il valore della funzione in quello specifico punto ma
il valore che assume \bt{nell'intorno} di quel punto.\\
Abbiamo dimostrato quindi che $\limxo f(x) = \limxo g(x)$.
\end{esem}
\bt{il limite è unico e dipende solo da un intorno del punto in cui lo calcoliamo
e non dal valore assoluto della funzione nel punto}.

Il vaolre assoluto avrà un valore importante nella nozione di continuità.


\begin{defn}[Restrizione di limiti]
Sia $f: A \to \R$ e $B \subseteq A$ diremo che la sua restrizione per $B$ è:


\begin{equation}
	f_{|_B}: B \to \R \qquad f_{|_B}(x) = f(x)^{\forall x \in B}
\end{equation}
I punti della funzione ristretta in $B$ sono tutti i punti della funzione di
di partenza restrigendo il Dominio da $A$ a $B$.\\
Se esiste il limite di $\limxo f(x)$ allora esite anche il limite
di $\limxo f_{|_B} (x)$ e questi due limiti coincidono:

	\begin{equation}
		\limxo f(x) = \limxo f_{|_B} (x)
	\end{equation}
\end{defn}


\begin{defn}
Se $A = \R \setminus \{ x_0 \}, \ B = ] x_0, +\infty [ $ oppure $B = ] -\infty
, x_0[$ \bt{limite destro} e \bt{limite sinistro} di $f$ in $x_0$ si definiscono:


\begin{equation}
	\lim \limits_{x \to x_0^+} f(x) = \limxo f_{|_{]x_0, +\infty[}}(x) \qquad
	\land \qquad
	\lim \limits_{x \to x_0^-} f(x) = \limxo f_{|_{]-\infty, x_0[}}(x)
\end{equation}
\end{defn}
Con $x_0^+$ intendiamo un numero poco più grande di $x_0$


\begin{defn}
La definizione di limite destro è:

\begin{equation}
	\forall \epsilon > 0\ \exists \delta > 0: ( x_0 < x < x_0 + \delta\
	\Longrightarrow\ |f(x) - \lambda| < \epsilon)
\end{equation}
per il limite sinistro serve una sostituzione ovvero:

\begin{equation}
	x_0 - \delta < x < x_0
\end{equation}
\end{defn}
Da quest'ultima definizione otteniamo che se $f(x)$ ha limite per $ \xtx$ allora 
esiste il limite per $\xtx^+$ e per $\xtx^-$\\
Serve a precisare anche che se i limiti destro e sinistro di una 
funzione sono diversi fra loro in uno specifico punto $x_0$ allora
il limite di $f(x)$ in $x_0$ non esiste


\begin{defn}[Teorema del collegamento]
Siano $\acr, x_0 \in D(A), \lambda, x_0 \in \ol{\R}, f: A \to \R$
Abbiamo che:

\begin{equation}
	\limxo f(x) = \lambda \ \iff \lim \limits_{\nti} f(x_n) = \lambda\
	\forall \{ x_n \}_{n \in \N} \subseteq A \setminus \{ x_0 \},\
	x_n \underset{\nti}{\to} x_0
\end{equation}
Vuol dire che se abbiamo una funzione $f(x)$ con \bt{limite}, possiamo 
"collegare" \bt{ogni} successione contenuta nel dominio e se usiamo 
come immagine della funzione un qualsiasi valore della successione, avremo come
limite di questa nuova funzione il limite iniziale.\\
Questo teorema è fondamentale in quanto riconduce il limite di una funzione a 
quello di una successione.
\end{defn}

\begin{Large}
	\bt{IMPORTANTE:} il simbolo di $ \infty $ non preceduto da nessun segno non
	ha nessun valore e non dobbiamo utilizzarlo nella risoluzione degli 
	esercizi in quanto è un simbolo troppo \bt{impreciso e fuorviante}.
\end{Large}


\begin{defn}
Il limite della somma di due limiti è la somma dei due limiti.\\
Il limite del prodotto di due funzioni è il prodotto dei due limiti.\\
Abbiamo vari casi a seconda del valore dei limiti:\\
Poniamo $ \acr, f,g: A \to \R, x_0 \in D(A) \cap \ol{\R} $, allora:

\begin{enumerate}[label*=\arabic*.]

	\item	 
	\begin{enumerate}[label=\Alph*.]
	\item i limiti sono dei numeri reali:\\
		
		\begin{equation}
		\begin{aligned}
		f(x) \txtx &\lambda \in \R\\
		g(x) \txtx &\mu \in \R\\
			   &\Downarrow\\
		\mbox{SOMMA:}\ f(x)+g(x) \txtx& \lambda+\mu\\
		\mbox{PRODOTTO:}\ f(x)\cdot g(x) \txtx& \lambda \cdot \mu\\
		\mbox{QUOZIENTE:}\ \frac{f(x)}{g(x)} \txtx& 
		\frac{\lambda}{\mu} \qquad (\mu \neq 0; g(x) \neq 0)
		\end{aligned}
		\end{equation}
	\item i limiti sono $ \pm \infty $\\
		Se le due funzioni $f,g$ hanno segni concordi per $\xtx$, allora:\\ 
		il limite della somma è $ \pm \infty $\\
		il limite del prodotto è il prodotto dei limiti e il segno è 
		definito dal prodotto dei segni.
	\end{enumerate}

\item Se $f(x) \txtx \pm \infty, f(x) \neq 0 $ in $A\setminus\{x_0\}$:
		
	\begin{equation}
		\limxo \frac{1}{f(x)} = 0
	\end{equation}
	e viceversa:\\
	Se $f(x) \txtx 0, f(x) > 0 $ in $A\setminus\{x_0\}$:

	\begin{equation}
		\limxo \frac{1}{f(x)} = \pm\infty
	\end{equation}

\item 
	\begin{enumerate}[label=\Alph*.]
	\item
	Se $f(x) \leq g(x),\ \forall x \in A \setminus \{x_0\},\ 
	\limxo f(x) = +\infty
	$ allora questo implica che $\limxo g(x) = +\infty$
	
	\item
	Se $f(x) \leq g(x),\ \forall x \in A \setminus \{x_0\},\ 
	\limxo f(x) = -\infty
	$ allora questo implica che $\limxo g(x) = -\infty$
	\end{enumerate}

\item $f(x) \txtx \lambda\ \implies |f(x)| \txtx |\lambda|$

	\end{enumerate}


\end{defn}


\begin{defn}[2 carabinieri]
	Poniamo $\acr, x_0 \in D(A)\cap\ol{\R};\ f,g,h: A \to \R.$
	A è un sottoinsieme di $\R$, $x_0$ è un punto di accumulazione di A, e
	$f,g,h$ sono delle funzioni con dominio A.\\
	Se esiste un intorno $W \in \mathcal{U}_{x_0} $ tale che:

	\begin{equation}
		f(x) \leq h(x) \leq g(x)\ \forall x \in (A \setminus \{ x_0 \}\cap W
	\end{equation}
	allora è vero che:
	\begin{equation}
		\limxo f(x) = \limxo g(x) = \lambda \in \ol{\R} \ \implies
		\ \limxo h(x) = \lambda
	\end{equation}
\end{defn}

\begin{dimo}
Per dimostrare questo teorema ci sono due casi differenti:

\begin{enumerate}
	\item $\lambda = \pm\infty$, la dimostrazione è banale in quanto basta 
		rivedere i lemmi derivati dai prodotti e le somme dei limiti.
	\item $\lambda \in \R$.\\
		Definiamo una successione $\{x_n\}_{n \in \N} \subseteq A
		\setminus \{x_0\},\ x_n \underset{\nti}{\to} x_0$ 
		convergente in $x_0$.\\
		Dal teorema del "collegamento" sappiamo che 

		\begin{equation}
			f(x_n) \underset{\nti}{\to} \lambda\ \&\
			g(x_n) \underset{\nti}{\to} \lambda
		\end{equation}
		Sappiamo inoltre che esiste un indice $\ol{n}$ per il quale
		$x_n \in (A \setminus \{ x_0 \}) \cap W;\ \forall n > \ol{n}$\\
		Quindi è vero che:

		\begin{equation}
			f(x_n) \leq h(x_n) \leq g(x_n)\ \forall n > \ol{n}
		\end{equation}
		Dal Teorema dei 2 Carabinieri (per successioni) otteniamo che
		$\lim \limits_{\nti} h(x_n) = \lambda$
\end{enumerate}
	
\end{dimo}
\begin{defn}[Cauchy]
Poniamo $\acr, x_0 \in D(A) \cap \ol{\R}, f: A \to \R.$\\
Allora sono equivalenti le seguenti:

\begin{enumerate}
	\item 
	\begin{equation}
		\exists \lambda \in \R: \limxo f(x) = \lambda
	\end{equation}
	\item 
	\begin{equation}
		\forall \epsilon > 0 \exists W \in \mathcal{U}_{x_0}:\
		\forall x,y \in A \setminus \{ x_0 \}\
		(x,y \in W \implies | f(x) - f(y)| < \epsilon
	\end{equation}
\end{enumerate}
\end{defn}
Questo teorema afferma che quando una funzione converge al proprio limite, esiste
un numero positivo ($\epsilon$) tale per cui esiste un intorno ($W$) per cui 
la differenza del modulo di due termini arbitrari ($x,y$) sono minori
di qualsiasi numero positivo ($\epsilon$) noi scegliamo.
\begin{dimo}

\begin{enumerate}
	\item $1 \implies 2$\\
	\begin{equation}
		\forall \epsilon > 0 \exists W \in \mathcal{U}_{x_0}:
		| f(x) - \lambda | < \frac{\epsilon}{2}\
		\forall x \in W \cap (A \setminus \{ x_0 \}
	\end{equation}
	Allora utilizzando un'applicazione della disuguaglianza triangolare:

	\begin{equation}
		| f(x) - f(y) | \leq | f(x) - \lambda + \lambda | \leq
		| f(x) - \lambda | + | \lambda - f(y) < \cancel{2} 
		\frac{\epsilon}{\cancel{2}} = \epsilon 
	\end{equation}
	
	\item $ 2 \implies 1 $\\
	Sia $ \{ x_n \} $ una successione convergente (e quindi di Cauchy)
	$ x_n \underset{\nti}{\to}
	x_0$, dalle nostre ipotesi sappiamo che:
	\begin{equation}
		\forall \epsilon > 0 \exists W \in \mathcal{U}_{x_0}:\
		\forall x,y \in A \setminus \{ x_0 \}\
		(x,y \in W \implies | f(x) - f(y)| < \epsilon
	\end{equation}
	Dato che la successione è convergente dai teoremi sulle successioni
	possiamo dedurre che esiste un indice $\ol{n}$ per il quale 
	$n,m > \ol{n}$ il che implica che:

	\begin{equation}
		| f(x_n) - f(x_m) | < \epsilon\ \forall n,m > \ol{n}
	\end{equation}
	Quindi la successione $ \{ f(x_n) \}_{n \in \N} $ è di Cauchy e 
	dall'assioma di completezza sappiamo anche che:

	\begin{equation}
		\exists \lambda \in \R :
		\lim \limits_{\nti} f(x_n) = \lambda
	\end{equation}
	Ovvero che esiste un numero reale che è il limite di questa successione.\\
	Per dimostrare che questa condizione è valida \bt{per ogni} successione
	convergente a $x_0$.\\
	Consideriamo un'altra successione $ \{ y_n \} \subseteq A \setminus 
	\{ x_0 \},\ y_n \underset{\nti}{\to} x_0 $. Allora 
	deve esistere un indice $\ol{\ol{n}} \in \N$ per cui ogni indice 
	maggiore di esso "cade" nell'intervallo $W$, ovvero:

	\begin{equation}
		n > \ol{\ol{n}}\ \implies y_n \in W \cap (A \setminus \{x_0\})
	\end{equation}
	Se allora consideriamo un indice $n$ tale che esso sia maggiore del
	massimo dei due altri indici implica che le due successioni si trovano
	all'interno dell'intervallo $W$:

	\begin{equation}
		n > max\{ \ol{n}, \ol{\ol{n}} \} \implies
		x_n, y_n \in (A \setminus \{ x_0 \} ) \cap W
	\end{equation}
	Dalla prima dimostrazione sappiamo che la differenza del modulo delle 
	due funzioni è più piccola di un numero arbitrariamente piccolo ($\epsilon$)

	Perciò:

	\begin{equation}
		\lim \limits_{\nti} ( f (x_n) - f(y_n) ) = 0
	\end{equation}
	Ma dato che il limite di $f(x_n)$ è $\lambda$ allora esso sarà
	il limite anche della funzione dell'altra successione:

	\begin{equation}
		\lim \limits_{\nti} f(y_n) = \lambda
	\end{equation}
	Questo è valido per qualsiasi arbitraria funzione $\{y_n\}$
\end{enumerate}
\end{dimo}


\subsection{Limiti di funzioni monotone}%
\label{sub:limiti_di_funzioni_monotone}



\begin{defn}
Poniamo $A \neq \varnothing, f: A \to \R$\\.
Si dice una funzione \bt{superiormente limitata} se è superiormente
limitato l'insieme $f(A)$.\\
Si dice una funzione \bt{inferiormente limitata} se è inferiormente
limitato l'insieme $f(A)$.\\
Più semplicemente diremo che esiste una costante ($M$) sempre maggiore (o minore)
dei valori della funzione nel suo dominio:

\begin{equation}
	\exists M \in \R: f(x) \geq M,\ \forall x \in A
\end{equation}

\begin{equation}
	\exists M \in \R: f(x) \leq M,\ \forall x \in A
\end{equation}
Si dice che una funzione è limitata se essa è minore del modulo di una costante
(M):

\begin{equation}
	\exists M \in \R: f(x) \leq |M|,\ \forall x \in A
\end{equation}


\end{defn}


\subsection{Notazione}%
\label{sub:notazione}
Poniamo:

\begin{equation}
	\underset{A}{sup} f = sup (f(A)), \quad \& \quad
	\underset{A}{inf} f = inf (f(A))
\end{equation}
Allora sono vere le seguenti affermazioni:


\begin{equation}
	\underset{A}{sup} f = \lambda \iff\
	\begin{cases}
		f(x) \leq \lambda, & \forall x \in A,\\
		\forall \epsilon > 0,\ \exists \ol{x} \in A: &
		 \lambda - \epsilon
		< f(\ol{x})
	\end{cases}
\end{equation}
La spiegazione segue che:\\
(1): se $\lambda$ è il superiore della funzione $f(x), A \to \R$, allora \bt{
ogni} immagine di essa sarà inferiore o uguale a $\lambda$ e questo 
e' valido per ogni elemento del dominio.\\
(2): Vale a dire che il sup é il minimo maggiorante, prendendo un qualsiasi 
	numero positivo ($\epsilon$) se sottraiamo
	questo numero dal $sup$ questa differenza sarà minore di una certa immagine
	della funzione, se fosse vero il contrario \Lightning, il $sup$ non sarebbe
	il minimo maggiorante.\\
Delle derivazioni di queste affermazioni sono che:

\begin{itemize}
	\item $\underset{A}{sup} f = + \infty$,\\
		implica che comunque noi scegliamo una costante $M$, esiste un
		elemento di x tale che la sua immagine sia maggiore di questa 
		costante.

		\begin{equation}
			\forall M \in \R,\ \exists x \in A: f(x) > M
		\end{equation}
		si ha un discorso analogo se l'$inf$ e' $- \infty$
\end{itemize}
Il $sup$ di una funzione puó appartenere alla funzione o no, mentre
il $max$ di una funzione 

\begin{equation}
	\underset{A}{max} f = max (f(A))
\end{equation}
appartiene alla funzione stessa perchè è un'elemento del dominio.

\begin{defn}[Monotonia di una funzione]
Siano $\acr, f: A \to \R$ si dice che questa funzione è monotona crescente se

\begin{equation}
	f(x) \leq f(y)\ \forall x,y \in A,\ x \leq y
\end{equation}
È \bt{strettamente} crescente se le disuguaglianze sono \bt{strette}. ($<$)

\begin{equation}
	f(x) < f(y)\ \forall x,y \in A,\ x < y
\end{equation}
Mentre una funzione è strettamente decrescente se la definzione precedente è valida
con le disuguaglianze invertite.
\end{defn}

\begin{equation}
	f \nearrow \equiv f\text{ crescente}
\end{equation}
Con $x_0$ di solito ci si riferisce al punto di massimo della funzione.

\begin{defn}
Poniamo $\acr, f: A \to \R, f \nearrow$, allora è vero che:

\begin{enumerate}
	\item $x_0 \in D(A \cap ] x_0, +\infty[)\ \implies\
		\lim \limits_{\xtx^+} f(x)$\\
		se $x_0$ è un punto di accumulazione dell'insieme formato dal suo
		dominio intersecato con $]x_0,+\infty[$, esiste
		il limite destro di questo punto.
	\item $x_0 \in D(A \cap ] -\infty, x_0[)\ \implies\
		\lim \limits_{\xtx^-} f(x)$\\
		se $x_0$ è un punto di accumulazione dell'insieme formato dal suo
		dominio intersecato con $]-\infty, x_0[$, esiste
		il limite sinistro di questo punto.

\end{enumerate}
\end{defn}

\subsection{Funzioni Continue}%
\label{sub:funzioni_continue}


\begin{defn}[Continuità]
Un modo pratico per definire la continuità di una funzione è quello di pensare di
dover disegnare il grafico della funzione, se non si è mai staccata la penna 
dal foglio, allora la funzione sarà continua.\\
Definiamo $\acr, f: A \to \R, x_0 \in A$.\\
Una funzione si definisce \bt{continua} in $x_0$ se:

\begin{equation}
	\forall V \in \mathcal{U}_{f(x_0)} \exists W \in \mathcal{U}_{x_0}:\
	\forall x \in A ( x \in W \Longrightarrow f(x) \in V)
\end{equation}
Questa definizione è molto simile alla definzione di limite, in questa circostanza
però il punto $x_0$ è incluso.
\end{defn}


La notazione di continuità:
\begin{equation}
	f \in C(A,\R) = \{ f: A \to \R: f \text{continua} \}
\end{equation}
Inoltre una funzione continua è composta da due punti diversi:

\begin{enumerate}
	\item Se $x_0$ è un punto isolato:\\
		$x_0 \notin D(A)$, implica che la funzione è continua in $x_0$
		quale che sia la funzione.\\

		\begin{equation}
			x_0 \notin D(A) \Longrightarrow \exists W \in
			\mathcal{U}_{x_0} : A \cap W = \{ x_0 \}
		\end{equation}
		Se $x_0$ è un punto isolato (quindi non un punto di accumulazione)
		esiste un intorno nella famiglia degli intorni di $x_0$. Se 
		interesechiamo l'intorno scelto con il dominio, otterremo 
		esattamente $x_0$.\\
		\begin{equation}
			\begin{aligned}
			\forall V \in \mathcal{U}_{f(x_0}) &\Longrightarrow
			f(x) \in V,\ \in A \cap W\\
			x \in A \cap W &\iff x = x_0
			\end{aligned}
		\end{equation}
		Ogni intorno dell'immagine di $x_0$, ogni immagine di $x$ si 
		trova in questo intorno: se $x \in A\cap W$
	\item Se $x_0$ è un punto di accumulazione:\\
		Allora $f$ è continua in $x_0$ se e solo se
		$\limxo f(x) = f(x_0)$
\end{enumerate}
Dei teoremi che derivano da questa definzione:
Se $\acr,\ f,g \in C(A)$:
\begin{itemize}
	\item $f+g \in C(A)$;
	\item $f \cdot g \in C(A)$;
	\item se $g \neq 0$ in $A$, allora $\frac{f}{g} \in C(A)$ 
	\item $ | f | \in C (A)$
\end{itemize}

\begin{defn}[Continuità e successioni]
Siano $\acr, x_0 \in A, f: A \to \R$.

\begin{equation}
	f \in C(A)\ \iff\ \forall \{ x_n \}_{n \in \N} \subseteq
	A, x_n \underset{\nti}{x_0};\ \lim \limits_{\nti} f(x_n) = f(x_0)
\end{equation}
Se abbiamo una funzione continua in un certo dominio, questo è valido se 
e solo se per ogni \bt{successione} contenuta in questo dominio converge a
$x_0$ e il limite dell'immagine di questa successione è l'immagine del punto
di convergenza.
\end{defn}

\begin{dimo}
Per dimostrare partendo dalla definzione di continuità:
\begin{equation}
	\forall V \in \mathcal{U}_{f(x_0)} \exists W \in \mathcal{U}_{x_0}:\
	\forall x \in A ( x \in W \Longrightarrow f(x) \in V)
\end{equation}
Sappiamo anche che per una successione convergente esiste un indice $\ol{n} \in \N$
tale che $x_n \in W, n > \ol{n}\ \Longrightarrow x_n \in A \implies x_n \in W 
\cap A\ (n > \ol{n}$


\begin{equation}
	\forall V \in \mathcal{U}_{f(x_0)} \exists \ol{n} \in \N:\
	\forall n( n > \ol{n} \Longrightarrow f(x) \in V)
\end{equation}
Ma questo vuol dire che il limite dell'immagine della successione è l'immagine del
punto di convergenza.

\begin{equation}
	\lim \limits_{\nti} f(x_n) = f(x_0)
\end{equation}
\end{dimo}


\begin{dimo}
Dimostriamo la continuità partendo dalla nozione per cui ogni successione
convergente in un punto $x_0$ di $A$ il limite dell'immagine della successione
è l'immagine del punto di convergenza:\\
Per assurdo \Lightning pensiamo che la funzione sia \bt{non} continua:\\
(Neghiamo la definizione di continuità)

\begin{equation}
	\exists V \in \mathcal{U}_{fx_0)}: \forall W \in \mathcal{U}_{x_0}
	\ \exists x \in A \cap W\ \&\ f(x) \notin V
\end{equation}
Si può riscrivere come:

\begin{equation}
	\forall n \in \N,\ \exists x_n \in A \cap \left] x_0 - \frac{1}{n},
	x_0 + \frac{1}{n} \right[\ \& f(x_n) \notin V
\end{equation}
La successione allora è contenuta in $A, x_n \to x_0$, dato che per ogni 
$n \in \N$:

\begin{equation}
	x_0 - \frac{1}{n} < x_n < x_0 + \frac{1}{n}  
\end{equation}
Per ipotesi abbiamo posto che l'immagine della successione non si trova nell'intorno
$V$ ma allo stesso tempo è nell'intorno il che è assurdo.

\begin{equation}
	f(x_n) \notin V ; f(x_n) \in V \text{\Lightning}
\end{equation}
\end{dimo}



\begin{defn}[Teorema di Weierstrass]
Sia $\acr$ un insieme compatto, ovvero un insieme chiuso e limitato, se $f \in C(A)$
Allora la funzione ha \bt{massimo} e \bt{minimo}
\end{defn}

\begin{dimo}
Sia $A \neq \varnothing, f: A \to \R$:

\begin{itemize}
	\item $\exists \{a_n\}_{n \in \N} \subseteq A: \lim
		\limits_{\nti} f(a_n) = \underset{A}{sup} f $
	\item $\exists \{b_n\}_{n \in \N} \subseteq A: \lim
		\limits_{\nti} f(b_n) = \underset{A}{inf} f $

\end{itemize}
Dimostriamo l'esistenza del massimo $max_A f$.\\
Sappiamo che esiste una successione che tende al $sup$ della funzione.

\begin{equation}
	\exists \{x_n\}_{n \in \N} \subseteq A\ :\ \lim \limits_{\nti} f(x) =
	sup_A f
\end{equation}
Dato che l'insieme in cui la successione è contenuta possiamo estrarre una 
sottosuccessione $ \{ x_{k_n} \}$ che converge in un certo punto $x_0$.\\
Allo stesso tempo per la continuità della funzione che l'immagine della 
sottosuccessione converge all'immagine del punto in cui converge 

\begin{equation}
	\lim \limits_{\nti} f (x_{k_n}) = f (x_0)
\end{equation}
Dai teoremi sulle successioni sappiamo che se una successione converge
in un certo punto anche \bt{ogni} sua sottosuccessione convergerà a 
quel punto. Quindi:

\begin{equation}
	\lim \limits_{\nti} f(x_{k_n}) = \underset{A}{sup} f
\end{equation}
Per il teorema dell'unicità del limite allora 

\begin{equation}
	f(x_0) = \underset{A}{sup} f\ \implies\ 
	f(x_0) = \underset{A}{max} f
\end{equation}
\end{dimo}

\begin{defn}[Teorema di Bolzano]
Presupponiamo come ipotesi che:\\
$a,b \in \R, a < b$ e $f$ sia continua in questo intervallo, $f \in C([a,b])$, 
tale che $f(a) \cdot f(b) \leq 0$.\\
Ipotizziamo che esista $ x_0 \in [a,b]: f(x_0) = 0$
\end{defn}

\begin{dimo}
La dimostrazione di questo teorema si basa sul dividere la funzione a metà 
(trovandone il punto medio $ \frac{a+b}{2} = c$ ).\\
In questo punto ci sono 3 scenari possibili:

\begin{enumerate}
	\item $c = 0$, il punto medio è = 0, non abbiamo bisogno di procedere
	\item la prima metà è = 0, quindi iteriamo il procedimento con questa
		parte della funzione.
	\item la seconda metà della funzione è = 0, iteriamo il procedimento.
\end{enumerate}
Iterando questo procedimento, per un certo indice $n \in \N$ otterremo che:

\begin{enumerate}
	\item  $f(c_n) = 0$, troviamo un certo punto medio uguale a 0;
	\item otteniamo 2 successioni, $\{ a_n \}, \{ b_n \} \subseteq
		[a,b]$ :

		\begin{equation}
			a_n \leq a_{n+1} \leq b_{n+1} \leq b_n, \quad
			f(a_n) \cdot f(b_n) \leq 0
		\end{equation}

		\begin{equation}
			b_n - a_n = \frac{b - a}{2^n}, \forall n \in \N 
		\end{equation}
		e il limite di queste due successioni è lo stesso

		\begin{equation}
			\lim \limits_{\nti} b_n = 
			\lim \limits_{\nti} a_n = x_0 \in [a,b]
		\end{equation}
		Per la continuità della funzione, presupposta nell'ipotesi,
	
		\begin{equation}
			\begin{aligned}
			\lim \limits_{\nti} f (a_n) \cdot f(b_n) &= (f(x_0)^2)\\
			f^2 (x_0) \leq 0 \Longleftarrow
			&f(a_n) \cdot f(b_n) \leq 0\ \forall n \in \N\\
			f(x_0) &=0
			\end{aligned}
		\end{equation}
\end{enumerate}
\end{dimo}

\begin{defn}[Teorema dei valori intermedi]
Una conseguenza del teorema di Blzano è il teorema dei valori intermedi:\\
Se $f: I \to \R, f \in C(I)$, allora $f(I)$ è un intervallo. Vale a dire che 
se abbiamo una funzione continua in un certo intervallo, l'immagine di questo
intervallo è un intervallo.\\
Sia $ I \subseteq R, f\in C(I), f$ non costante (altrimenti sarebbe banale).

\begin{equation}
	\begin{aligned}
		\forall x_1, x_2 \in I &: f(x_1) < f(x_2)\\
		\forall y \in ]f(x_1), f(x_2)[\ &\exists x \in I : f(x) = y
	\end{aligned}
\end{equation}
\end{defn}

\begin{dimo}
	Sappiamo che $ x_1 \neq x_2 $, allora poniamo $ x_1 < x_2 $, (ciò potrebbe
	essere invertito senza nessun problema).\\
	Definiamo una funzione $g$ che ha come dominio l'intervallo chiuso
	definito dalle due variabili prima definite:

	\begin{equation}
		g: [ x_1, x_2 ] \to \R, g(x) = f(x) - y\ 
		\implies\ g \in C([x_1, x_2])
	\end{equation}
	Definiamo la funzione $g$ nell'intervallo chiuso $ [x_1, x_2] $, questa
	funzione sarà uguale alla funzione di partenza $f$ se gli sottraiamo
	una costante $y$, che è un punto intermedio fra i due punti di partenza.

	\begin{equation}
		f(x_1) < f(x_0) = y < f(x_2)
	\end{equation}
	Come conseguenza del teorema di Bolzano, esiste un punto della funzione 
	$x_0 \in [x_1, x_2]$ tale che la funzione si annulli. $g(x_0) = 0 $ 

\end{dimo}
Se $f: I \to \R, f \in C(I) \implies f(I)$ è un intervallo e: 

\begin{enumerate}
	\item $f(I) = ]inf_I f,\ sup_I f[$ se $f$ non ha nè $max$ nè $min$
	\item $f(I) = ]inf_I f,\ max_I f[$ se $f$ ha $max$ e non ha $min$
	\item $f(I) = ]min_I f,\ sup_I f[$ se $f$ non ha $max$ ma ha $min$
	\item $f(I) = ]min_I f,\ max_I f[$ se $f$ ha $max$ e $min$
\end{enumerate}
Possiamo osservare che se abbiamo un intervallo $I \subseteq \R$:

\begin{itemize}
	\item Se $f \in C(I)$ e iniettiva, allora la funzione è \bt{monotona}
	\item Se $f$ è monotonae $f(I)$ è un intervallo allora la funzione è
		continua $ f \in C(I) $
\end{itemize}
Un teorema che ne deriva è:\\
Se abbiamo una funzione continua in un certo intervallo $f \in C(I), I \subseteq \R$
e la funzione è iniettiva, poniamo $J = f(I)$:

\begin{itemize}
	\item $J$ è un intervallo e $ f: I \to J $ è sia suriettiva che iniettiva
	\item L'inversa della funzione è continua nell'intervallo $J$:\\
		$ f^{-1} \in C(J)$
\end{itemize}


\subsection{Uniforme Continuità}%
\label{sub:uniforme_continuita}
Siano: $\acr, f: A \to \R$, si dice che una funzione è \bt{uniformemente continua}
su $A$, se:

\begin{equation}
	\forall \epsilon > 0,\ \exists \delta > 0: \forall x,y \in A
	(|x - y | ) < \delta \Longrightarrow | f(x) - f(y) | < \epsilon
\end{equation}
Si può leggere come: qualsiasi numero reale positivo noi scegliamo ($\epsilon$), 
per quantopiccolo esso sia, esiste un altro numero reale positivo ($\delta$), tale 
per cui, qualsiasi due elementi di una fuznione ($x,y$), la loro differenza sarà
\bt{sempre} minore di $\delta$, e ciò implica che la differenza delle immagini
di questi di punti sia minore di $\epsilon$, questa è la condizione di \bt{
uniforme continuità}\\
{\Large Notiamo che la continuità non implica l'uniforme continuità, invece
l'uniforme continuità implica la continuità.}


\begin{defn}[Heine-Cantor]
Sia $\acr$ un insieme compatto, poniamo una funzione $f \in C(A)$ continua, allora 
questo implica che è anche uniformemente continua.\\
Più semplicemente, una funzione continua in un insieme compatto è uniformemente 
continua.
\end{defn}

\begin{dimo}[Facoltativa]
Ipotizziamo per assurdo \Lightning che la funzione non sia uniformemente continua:


\begin{equation}
	\exists \epsilon > 0:\ \forall \delta > 0\ \exists x,y \in A:\
	| x - y | < \delta\ \& | f(x) - f(y) | \geq \epsilon
\end{equation}
Traducendo in termini di successioni (in particolare in una successione convergente
per $x_0$) sostituiamo $\delta$ con un certo indice $n$ della successione.

\begin{equation}
	\forall n,\ \exists x_n, y_n \in A: | x_n - y_n | < \frac{1}{n}\ \&\
	| f (x_n) - f(y_n) \geq \epsilon
\end{equation}
Sapendo che il dominio della funzione è un insieme compatto (ovvero chiuso e
limitato), possiamo estrarre una sotto-successione $\{x_{k_n} \}$ che converge
ad un punto $x_0 \in A $

\begin{equation}
	| x_{k_n} - y_{k_n} | < \frac{1}{k_n}\ \forall n \in \N
\end{equation}
Questo implica che anche $y_{k_n}$ converge a $x_0$\\
Inoltre possiamo dedurre dalla nozione di continuità che:

\begin{equation}
	\lim \limits_{\nti} ( f(x_{k_n} ) - f(y_{k_n}) ) = f(x_0) - f(x_0) = 0
\end{equation}
Questo fatto è assurdo poichè per ipotesi avevamo posto che: 

\begin{equation}
	| f(x_{k_n} - f(y_{k_n}) | \geq \epsilon > 0\ \forall n \in \N 
\end{equation}


\end{dimo}

\begin{center}
	\LARGE FINE PARTE DI TEORIA
\end{center}
\end{document}

