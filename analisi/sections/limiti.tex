\documentclass[../analisi.tex]{subfiles}

\begin{document}

\section{Limiti per funzioni}
\subsection{Definizioni}

\begin{defn}[Intorno di $\pm \infty$] 
\bt{L'intorno} di $ + \infty$ è l'insieme:


\begin{equation}
	]k, + \infty [,\ k \in \R
\end{equation}
\bt{L'intorno} di $ - \infty$ è l'insieme:


\begin{equation}
	] - \infty, k [, \ k \in \R
\end{equation}
\end{defn}
Intendiamo che se $ \acr $ come i seguenti


\begin{equation}
	\begin{aligned}
	sup A &= + \infty\ \iff\ + \infty \in D(A)\\
	inf A &= - \infty\ \iff\ - \infty \in D(A)
	\end{aligned}	
\end{equation}
Un limite $\lambda$ può essere:


\begin{equation}
	\lambda \in \ol{R}\ \iff\ 
	\begin{cases}
		\lambda \in R\\
		\lambda = +\infty\\
		\lambda = -\infty
	\end{cases}
\end{equation}

\begin{defn}
Siano $\acr,\ x_0 \in D(A),\ \lambda \in \ol{R}.\ f: A \to \R$.\\
$A$ un insieme contenuto fra i numeri reali, 
$x_0$ si trovi in un punto di accumulazione di $A$, 
$\lambda$ contenuto nella retta reale \tit{estesa} e $f$ una funzione che ad ogni 
elemento di $A$ corrisponde un elemento di $R$.\\
Diremo che $f(x)$ tende a $\lambda$ per $x$ che tende a $x_0$.


\begin{equation}
	\forall V \in \mathcal{U}_\lambda\ \exists W \in \mathcal{U}_{x_0}\ :\
	f(x) \in V\ \forall x \in A \setminus \{ x_0 \}\cap W
\end{equation}
Qualunque intorno $V$ all'interno della famiglia degl intorni $\mathcal{U}_\lambda$
di $\lambda$ (ricordando che $\lambda \in \ol{R}$ quindi può assumere o
un valore reale o è uguale a  $ \pm \infty$), esiste un intorno $W$ all'interno
della famiglia degli insiemi di $\mathcal{U}_{x_0}$ di $x_0$ (un punto di 
accumulazione) 

\begin{equation}
	x_0 \in \R \qquad ] x_0 - \delta, x_0 + \delta [
\end{equation}
 tale che l'immagine $x$  sia in $V$ tutte le volte che $x$ in $A \setminus
 {x_0}$ intersecato con l'intorno $W$, $W_{x_0}$ è per definzione un punto
 di accumulazione quindi l'interesezione \bt{è non vuota}\\
La scrittura semplificata:


\begin{equation}
	\limxo f(x) = \lambda
\end{equation}
se \bt{per ogni} $ \epsilon > 0 $ esiste un numero $\delta > 0$ tale che per ogni
$ x \in A, \ 0 < | x - x_0 | < \delta$.

\begin{equation}
	| f(x) - \lambda | < \epsilon 
\end{equation}
\end{defn}

\begin{esem}
Abbiamo $x_0 \in \R$ e $ \lambda = - \infty$ avremo che:

\begin{equation}
	\begin{aligned}
	\mathcal{U}_{x_0} &= \{ ] x_0 - \delta, x_0 + \delta[ : \delta > 0\}\\
	\mathcal{U}_{\lambda} &= \mathcal{U}_{- \infty} = \{
	] -\infty, k [: k \in \R \}	
	\end{aligned}
\end{equation}
Quindi avremo che:

\begin{equation}
	\limxo f(x) = - \infty
\end{equation}
se e solo se:

\begin{equation}
	\forall k \in \R\ \exists \delta > 0 :\ \forall x \in A \setminus \{
	x_0 \} \ \{ | x - x_0 | < \delta\ \implies\ f(x) < k\}
\end{equation}
Vale a dire che: qualunque valore noi diamo a $k$ che è un numero appartenente 
ai numeri reali (quindi ha un valore finito), esiste un  numero $\delta$ maggiore di
0 tale che, quale che sia $x$ contenuta in $A$ meno $x_0$, il \bt{modulo} della
differenza di $x$ e $x_0$ ($| x - x_0 |$) è \bt{minore} di $\delta$, questo vuol
dire che l'immagine di $f(x)$ è sempre \bt{strettamente minore} di k.\\
Ovvero $f(x)$ avrà sempre un valore piccolissimo inferiore a qualsiasi numero reale

\end{esem}

Anche se $f(x)$ è definita nel punto $x_0$ non è necessario
che  soddisfare $\limxo f(x) = \lambda$, quindi nel punto in cui 
$x = x_0$. Affermiamo che il valore del limite $\lambda$ è indipendente 
dal valore della funzione nel punto $x_0$.
%TODO stuff i didnt understand at page 86/87

\begin{defn}[Unicità del limite]
Sia $\acr$ e siano $x_0 \in D(A),\ x_0 \in \ol{R},\ f: A \to \R$. \\
Se esistono $ \lambda, \mu \in \ol{\R}$ t.c.:


\begin{equation}
	\limxo f(x) = \lambda \quad \land \quad  \limxo f(x) = \mu \quad 
	\iff \quad \lambda = \mu
\end{equation}
\end{defn}
Prendiamo un sottoinsieme $A$ di $\R$, che è il dominio di una funzione $f(x): A 
\to \R$, e prendiamo $x_0$ un punto di accumulazione dell'insieme $A$, contenuto 
in $\R$. Supponiamo che ci siano \bt{due} limiti  (per assurdo), ma allora questi 
due limiti \bt{SONO UGUALI}.

\begin{dimo}
Partiamo dalla nozione che: \bt{punti distinti} ammettono \bt{intorni disgiunti}
\begin{dimo}

\end{dimo}

\end{document}
