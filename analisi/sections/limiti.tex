\documentclass[../analisi.tex]{subfiles}

\begin{document}

\section{Limiti per funzioni}
\subsection{Definizioni}

\begin{defn}[Intorno di $\pm \infty$] 
\bt{L'intorno} di $ + \infty$ è l'insieme:


\begin{equation}
	]k, + \infty [,\ k \in \R
\end{equation}
\bt{L'intorno} di $ - \infty$ è l'insieme:


\begin{equation}
	] - \infty, k [, \ k \in \R
\end{equation}
\end{defn}
Intendiamo che se $ \acr $ come i seguenti


\begin{equation}
	\begin{aligned}
	sup A &= + \infty\ \iff\ + \infty \in D(A)\\
	inf A &= - \infty\ \iff\ - \infty \in D(A)
	\end{aligned}	
\end{equation}
Un limite $\lambda$ può essere:


\begin{equation}
	\lambda \in \ol{R}\ \iff\ 
	\begin{cases}
		\lambda \in R\\
		\lambda = +\infty\\
		\lambda = -\infty
	\end{cases}
\end{equation}

\begin{defn}
Siano $\acr,\ x_0 \in D(A),\ \lambda \in \ol{R}.\ f: A \to \R$.\\
Diremo che $f(x)$ tende a $\lambda$ per $x$ che tende a $x_0$.


\begin{equation}
	\forall V \in \mathcal{U}_\lambda\ \exists W \in \mathcal{U}_{x_0}\ :\
	f(x) \in V\ \forall x \in A \setminus \{ x_0 \}\cap W
\end{equation}
La scrittura semplificata:


\begin{equation}
	\limxo f(x) = \lambda
\end{equation}
\end{defn}

%TODO stuff i didnt understand at page 86/87

\begin{defn}[Unicità del limite]
Sia $\acr$ e siano $x_0 \in D(A),\ x_0 \in \ol{R},\ f: A \to \R$.
Se esistono $ \lambda, \mu \in \ol{\R}$ t.c.:


\begin{equation}
	\limxo f(x) = \lambda\ \&\ \limxo f(x) = \mu \quad 
	\iff \quad \lambda = \mu
\end{equation}
\end{defn}

\begin{dimo}

\end{dimo}

\end{document}
