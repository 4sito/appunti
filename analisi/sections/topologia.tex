\documentclass[../appunti.tex]{subfiles}

\begin{document}

\section{Topologia della retta euclidea}

\subsection{Intervalli}
Siano $a,b \in \R$, si pone per definizione 


\begin{equation}
	\begin{aligned}
	\mbox{aperto}    \ ] a,b [ &= \{ x \in \R : a < x < b \}\\
	\mbox{semiaperto}\ [ a,b [ &= \{ x \in \R : a \leq x < b\}\\
	\mbox{semiaperto}\ ] a,b ] &= \{ x \in \R : a < x \leq b\}\\
	\mbox{chiuso}    \ [ a,b ] &= \{ x \in \R : a \leq x \leq b\}
	\end{aligned} % ]
\end{equation}
Se uno degli estremi è $\pm \infty,\ (a = - \infty, b = + \infty)$


\begin{equation}
	\begin{aligned}
		] a, +\infty [ &= \{ x \in \R: x > a \}\\
		] -\infty, a [ &= \{ x \in \R: x < a \}\\
		[ a, +\infty [ &= \{ x \in \R: x \geq a \}\\
		] -\infty, a ] &= \{ x \in \R: x \leq a \}\\
	\end{aligned}
\end{equation}

\begin{defn}
Se $ x_0 \in \R, \rho > 0$ , si pone 


\begin{equation}
	B( x_0, \rho) = ] x_0 - \rho, x_0 + \rho [
\end{equation}
si chiama \tit{intorno aperto} di $x_0$ di raggio $\rho \in \R_+$.\\
La \tit{famiglia} degli intorni aperti di $x_0$ si denota come


\begin{equation}
	\mathcal{U}_{x_0} = \{ ] x_0 - \rho, x_0 + \rho [ : \rho > 0 \}
\end{equation}
\end{defn}

\subsection{Punti di accumulazione, isolati e aderenti}

\begin{defn}[Punti di accumulazione]
	Siano $ \acr, x_0 \in \R $, si dice che $x_0$ è un 
	\bt{punto di accumulazione} di $A$ se per ogni $W$ (intorno) $\in 
	\mathcal{U}_{x_0}$ (famiglia degli intorni):
	

	\begin{equation}
		A \setminus \{ x_0 \} \cap W = \varnothing
	\end{equation}
	\bda


	\begin{equation}
		(A \setminus \{ x_0 \} ) \cap\ ] x_0 - \rho, x_0 + \rho [
		\neq \varnothing\ \forall \rho > 0
	\end{equation}
In parole, se prendiamo l'insieme $A$ e ad esso sottriamo un qualsiasi punto $x_0$
e ad esso intersechiamo l'intervallo formato di raggio 
\tit{rho} "$\rho$" e con centro $x_0$. 
Questo insieme è un punto di accumulazione se quanto citato prima
\bt{non è} un insieme vuoto, ovvero in esso troviamo almeno \bt{un elemento}. \\
L'\bt{insieme} dei punti di accumulazione si chiama \bt{derivato} di $A\ =\ D(A)$.\\
Per definizione poniamo $D(\varnothing) = \varnothing$
\end{defn}
\begin{defn}[Punto isolato]
	Se $ x \in A $ e $ x \notin  D(A)$ si dice che $x$ è un \bt{punto isolato}
\end{defn}

\begin{dimo}
Se $A \subset R$ è un insieme finito, questo implica che $D(A) = \varnothing$.\\

\begin{enumerate}

\item 
Se $A = \varnothing\ \implies D(\varnothing) = \varnothing$

\item 
Se $A = \{ x_1, x_2, \dots, x_p \}$ Nessuno $ z \in \R$ 
è punto di accumulazione per A:
\begin{equation}
	z \notin D(A),\ \forall z \in \R
\end{equation}	

	\begin{itemize}
	\item 
	
	Supponiamo che $z$ non sia in $\R$: \\
	$ z \in \R \setminus A\ \to\ z \neq x_j,\ \forall j = 1, \dots, p$
	\begin{equation}
	\rho = \{ | z - x_j |:\ j\ =\ 1, \dots, p \}
	\end{equation}
	$ | z - x_j | = 0 \ \iff z = x_j $ ma $z$ è escluso dall'insieme $A$
	\begin{equation} 
		(A \setminus \{ z \}\ \cap\ ] z - \rho, z + \rho [ = \varnothing
	\end{equation}

	\item

	Supponiamo invece che $z$ sia in $A$:\\
	$ z \in A \to z = x_1$
	\begin{equation}
	\rho = \{  | x_1 - x_j | :\ j\ =\ 2, \dots, p \}
	\end{equation}
	$\rho > 0 $ dato che i punti di $x_j \in A$ sono diversi tra loro.\\
	Se ne deduce quindi che l'intorno aperto $B(x_1, \rho)$ di centro
	$x_1$ di raggio $\rho$ esclude qualsiasi altro punto di $A$
	\begin{equation}
	( A \setminus \{ x_1 \} )\ \cap\ ] x_1 - \rho, x_1 + \rho [ = \varnothing
	\end{equation}

	\end{itemize}
\end{enumerate}
\end{dimo}

\begin{esem}[$ \acr $ e $ D(A) \neq \varnothing\ \implies\ A$ è \bt{infinito}]
Ovvero, se $A$ è un insieme contenuto nell'insieme dei numeri reali e 
l'insieme dei suoi punti di accumulazione \bt{non} è vuoto allora $A$ 
è infinito.\\
Ciò non è vero in quanto questa proposizione è solamente una 
\bt{condizione necessaria} ma \bt{non sufficiente}. \\
$\N$ è infinito ma $D(\N) = \varnothing$
\end{esem}

\begin{defn} 
Siano $ \acr,\ x_0 \in \R$, allora:


\begin{equation}
	x_0 \in D(A)\ \iff\ \exists \{ x_n \}_{n \in \N} \subseteq A
	\setminus \{ x_0 \}: x_n \underset{\nti}{\to} x_0
\end{equation}
% TODO dimostrazione tramite assioma scelta%

\end{defn}

\begin{defn}
Sia $ \acr $, $A$ infinito e limitato, allora $D(A) \neq \varnothing$ 
\end{defn}

\begin{dimo}
	$A$ è infinito quindi esiste $\{ x_n \}_{n \in \N} \subseteq A$ t.c. 
	$x_n \neq x_m, \ \forall n,m \in \N, n \neq m$.\\
	$A$ è limitato quindi $ \{ x_n \}_{n \in \N}$\\
	 TEOREMA BOLZANO-WEIERSTRASS : 
	
	
	\begin{equation}
		\exists \{ x_n \}_{n \in \N}\ :\ x_{k_n} 
		\underset{\nti}{\to} x_0 \in \R
	\end{equation}
	Se $ x_{k_n} \neq x_0$ per ogni $ n \in \N$, allora 
	$\{ x_{k_n} \}_{n \in \N} \subseteq A \setminus \{x_0\}$ e converge a $x_0$
	Quindi $ x_0 \in D(A) $\\
	Se $ x_{k_p} = x_0,\ p \in \N$, avremo che 
	$\{ x_{k_{n + p}}  \}_{n \in \N} \subseteq A \setminus \{ x_0 \}$ e 
	questa successione converge ad $x_0$. 
	$x_{k_n} \neq x_{k_p},\ \forall n \neq p$.\\
	In entrambi i casi: $ \exists x_0 \in D(A)$
\end{dimo}


\begin{defn}[Punti aderenti]
Sia $\acr$, sia $x_0 \in \R$, $x_0$ è aderente in $A$ se 


\begin{equation}
	A \cap W \neq \varnothing\ \forall W \in \mathcal{U}_{x_0}
\end{equation}
\bda


\begin{equation}
	A \cap ] x_0 - \rho, x_0 + \rho [ \neq \varnothing\ \forall \rho > 0
\end{equation}
\end{defn}


\begin{defn}[Chiusura]
Si dice \bt{chiusura} di $A$, $\ol{A}$, l'insieme dei punti aderenti ad $A$:


\begin{equation}
	\ol{A} = \{ x \in \R: x \mbox{è aderente ad} A \}
\end{equation}
\end{defn}
La chisura dell'insieme vuoto corrsiponde per convenzione all'insieme vuoto.


\begin{equation} 
	\ol{\varnothing} = \varnothing
\end{equation}
Dalle definizioni precedenti ricaviamo:


\begin{equation}
	D(A) \subseteq \ol{A}\ \&\ A \subseteq \ol{A}
\end{equation}
Il fatto per cui, l'insieme dei punti di accumulazione di $A$ è contenuto
nell'insieme dei punti di aderenza di $A$ è dovuto dal fatto che al primo insieme
"togliamo" $x_0$, ovvero il centro dell'intervallo di riferimento, mentre
fa parte del secondo.


\end{document}
