\documentclass[../appunti.tex]{subfiles}
\graphicspath{{\subfix{../pics/}}}



\begin{document}

\section{Combinatoria e probabilità}
\subsection{Introduzione}

\begin{defn}
  L'\bt{analisi combinatoria} è la branca della matematica applicata per risolvere problemi nel quale è necessario saper "contare" efficacemente esiti e probabilità di determinate situazioni.\\
  Essa è infatti la disciplina che ci permette di \bt{contare senza contare}
\end{defn}

\subsection{Combinatoria}
\begin{defn}[Principio di moltiplicazione]
  Un insieme $X$ soddisfa le ipotesi del principio di moltiplicazione se:

  \begin{itemize}
    \item è possibile ottenere ciascuno dei suoi elementi come risultato di una procedura composta da $n$ fasi successive.
    \item se ad una fase interemedia si sono ottenuti due esisti distinti allora la procedura conduce ad elementi distinti di $X$
  \end{itemize}

Nella prima fase avremo $m_1$ possibili esiti nella seconda fase avremo $m_2$ esiti sino alla n-esima fase avremo $m_n$ esiti

\begin{equation}
  |X|\ = m_1\ \times\ m_2\ \times\ ...\ \times\ m_k
\end{equation}
\end{defn}

\begin{eser}
  Calcoliamo il numero di coppie ordinate $(a,b)$ contenenti un numero primo ed uno non primo compresi tra 1 ed 8
\end{eser}

\begin{dimo}
  I numeri primi tra 1 e 8 sono $\{ 2, 3, 5, 7\}$ mentre i numeri non primi tra 1 e 8 sono $\{ 1, 4, 6, 8\}$

\begin{enumerate}[label=\Roman*.]
  \item Scegliamo un qualsiasi elemento di $I_8$: abbiamo 8 possibilità.
  \item Se il primo elemento era primo il secondo non lo sarà, e viceversa se il numero non era primo. In ogni caso avremo 4 distinte possiblità

\end{enumerate}
Il numero di coppie è: $8\ \times\ 4\ =\ 32$
\end{dimo}

\begin{eser}
  Consideriamo un'estrazione in successione di 3 numeri della tombola \bt{tenendo conto dell'ordine}. Quanti sono i possibili esiti?
\end{eser}

\begin{dimo}
  I numeri della tombola sono 90. Gli scenari possibili sono 2:\\
  Nel primo caso \bt{senza rimpiazzo} se ogni numero può essere scelto una volta sola, mentre sarà \bt{con rimpiazzo} se un numero può essere scelto più di una volta.\\

Nel primo caso $(a_1, a_2, a_3)\ :\rightarrow\ (a_1\ \neq\ a_2\ \neq\ a_3\ )$ :

\begin{enumerate}[label=\Roman* \textsc{fase}:]
    \item  $a_1 = 90$
    \item  $a_2 = 90 - 1 = 89$
    \item  $a_3 = 90 - 2 = 88$
\end{enumerate}

Quindi il numero di possibili esiti è:
\begin{equation}
  90\ \times\ 89\ \times\ 88\ =\ 704880
\end{equation}

Nel secondo caso $(a_1,a_2,a_3) :\rightarrow (a_1\ =\ a_2\ =\ a_3)$:
\begin{enumerate}[label=\Roman* \textsc{fase}:]
    \item  $a_1\ =\ 90$
    \item  $a_2\ =\ 90$
    \item  $a_3\ =\ 90$
\end{enumerate}
Quindi il numero di possibili esiti è:

\begin{equation}
  90\ \times\ 90\ \times\ 90\ =\ 90^3\ =\ 729000
\end{equation}
\end{dimo}

\begin{defn}
  Definiamo una regola general per $k$-sequenze di $I_n$.
  Siano $k,n\ \in\ \N$ definiamo $k$-sequenza di $I_n$ una $k$-upla \bt{ordinata} $(a_1,\ldots ,a_k)$ di elementi \bt{non necessariamente distinti} di $I_n$ Ovvero:
  \begin{equation}
    (a_1,\ldots,a_k)\in\ \underbrace{I_n\times\ \ldots\ \times\ I_n\ }
  \end{equation}
  % with the command |underbrace or upbrace we can have curly braces up or down a section of a given equation
\end{defn}

Nella definzione di sequenze l'ordine degli elementi della $k$-upla è importante: le 3-sequenze (2, 1 ,3 ) e (3, 1, 2) sono diverse anceh se composte dagli stessi numeri. Vengono comunemente dette \bt{disposizioni} di \bt{n} oggetti a $k$ a $k$

\begin{esem}
  Sia $I_4\ =\ {1,2,3,4}$. Allora
  \begin{equation}
    (1,2,3,3,4),\qquad (1,1,1,1,1),\qquad (2,2,1,3,4)
  \end{equation}
  sono 5-sequenze di $I_4$. Invece
  \begin{equation}
    (1,2,3),\qquad (1,1,1),\qquad (2,3,4)
  \end{equation}
  sono 3-sequenze di $I_4$
\end{esem}

\subsection{Fattoriale}

\begin{equation}
  5!=\ 5\ \times\ 4\ \times\ 3\ \times\ 2\ \times\ 1
\end{equation}

\begin{equation}
  n!=
  \begin{cases}
    n\ \times\ (n-1)\ \times\ (n-2)\ \times\ \ldots\ 3\ \times\ 2\
    \times\ 1 &  \mbox{se } n\geq\ 1 \\
    1 &  \mbox{se }n\ =\ 0
  \end{cases}
      % 'cases' is used for system of equations. Write the conditions of the if statements AFTER the & symbol
\end{equation}

\begin{defn}
  Il \bt{fattoriale} di un numero equivale al prodotto di quel numero per tutti i numeri che lo precedono. I valori dei fattoriale crescono esponenzialmente
  \begin{equation}
    0!=\ 1 \qquad 5!=\ 120 \qquad 6!=\ 720 \qquad 7!=\ 5040 \qquad 10!=\ 3628800
  \end{equation}
\end{defn}

\subsection{Numero di Insiemi}
\begin{defn}
	Il \bt{numero di sottoinsiemi} di $k$ elementi di $I_n$ si distinguono esclusivamente dagli elementi di cui fanno parte: \bt{l'ordine non conta}.
\end{defn}

Spesso un sottoinsieme di $k$ elementi di un insieme di $n$ elementi viene chiamato \bt{combinazione} (semplice, senza ripetizioni) di $n$ elementi a $k$ a $k$

\begin{defn}
	Siano $k,n \in \N$ il \bt{binomiale} di $n$ su $k$ è:
	\begin{equation}
		\bfrac{n}{k}\ =
		\begin{cases}
			\frac{n!}{k!(n-k)!}, & \mbox{se} k\ \leq\ n,\\
			0, 		    & \mbox{se} k\ > n.
		\end{cases}
	\end{equation}
	Il numero di sottoinsiemi di $k$ elementi di $I_n$ è 
	\begin{equation}
		\bfrac{n}{k}.
	\end{equation}
\end{defn}


\begin{esem}
	Calcola i sotttoinsiemi con $3$ elementi di $I_6$
\end{esem}

\begin{dimo}
	La soluzione è data da una semplice applicazione della formula prima vista:
	\begin{equation}
		\bfrac{6}{3}\ =\ \frac{6!}{3!3!}\ =\ 20
	\end{equation}
\end{dimo}


\begin{esem}
	Calcola il numero di partite giocate nella fase a gironi dei Mondiali di calcio. Ci sono 3$2$ squadre divise in $8$ gironi da $4$ squadre ed in ogni girone una squadra deve giocare contro le altre una volta sola.
\end{esem}

\begin{dimo}
	Il numero di partite totale è $8$ volte le partite giocate in un singolo girone.
	L'insieme delle 4 squadre in un girone possiamo identificarlo con $I_4$, e una partita tra 2 squadre con un sottoinsieme di $2$ elementi di $I_4$. Il numero di partite giocate in un girone\bt{è il numero di sottoinsiemi} di $2$ elementi di $I_4$ ovvero:
	\begin{equation}
		\bfrac{4}{2}\ =\ \frac{4!}{2!(4-2)! }\ =\ \frac{4\times 3\times 2\times 1}{2\times 2}\ = \frac{24}{4}\ =\ 6
	\end{equation}
	Infine il risultato equivale a: $ 6 \times 8 = 48$
\end{dimo}





\end{document}
