\documentclass[../analisi.tex]{subfiles}

\begin{document}


\section{Derivate}%
\label{sec:derivate}

\subsection{Rapporto incrementale e derivate}%
\label{sub:rapporto_incrementale_e_derivate}
Per poter  comprendere le derivate è essenziale comprendere il concetto di 
\bt{rapporto incrementale}\\

\begin{defn}[Rapporto Incrementale]
Premettiamo: $ \acr,\ f: A \to \R,\ x_0 \in A \cap D(A)$\\
$ x_0$ è un punto di accumulazione, non isolato e $f$ una funzione con dominio
$A$ e codominio $\R$.\\

\begin{equation}
	R_f ( x_0 ): A \setminus \{ x_0 \} \to \R, \quad 
	R_f ( x_0 ) (x) = \frac{f(x) - f(x_0) }{x - x_0} 
\end{equation}
Una funzione si dice \bt{derivabile} in $x_0$ se esiste il limite, ed è finito:

\begin{equation}
	\limxo R_f ( x_0 ): A \setminus \{ x_0 \} \to \R, =
	\limxo R_f ( x_0 ) (x) = \frac{f(x) - f(x_0) }{x - x_0} 
\end{equation}
Il limito che abbiamo appena definito si chiama \bt{derivata} di $f$ in $x_0$
\begin{equation}
	f'( x_0 ) =
	\limxo R_f ( x_0 ) (x) = \frac{f(x) - f(x_0) }{x - x_0},\
	f'(x_0) \in \R
\end{equation}
\end{defn} 
Se il limite, del rapporto incrementale, non appartiene ai numeri reali ed è 
$\pm \infty$, allora la funzione è derivabile in senso esteso.\\
Il limite del rapporto incrementale si può riscrivere come:

\begin{equation}
	\lim_{h \to 0} \frac{f(x_0 + h ) - f(x_0) }{h} \qquad (h = x - x_0)
\end{equation}

\begin{defn}
	Se una funzione $f$ è derivabile in un punmto $x_0$ allora:

	\begin{equation}
		\exists \lambda \in \R, \exists \omega : A \to \R\
		\omega (x) \to \omega (x_0)  = 0 | \xtx
	\end{equation}
	Allora:

	\begin{equation}
		\begin{aligned}
		f(x) =& f(x_0) + \lambda \cdot ( x - x_0 ) + \omega (x) (x-x_0)\ 
		\forall x \in A\\
		\lambda = &\frac{f(x) - f(x_0)}{x - x_0} \qquad w(x) 
		\text{è infinitesima (=0)}\\
		\lambda = & f'( x_0 )
		\end{aligned}
	\end{equation}
\end{defn}


\begin{defn}
Premettiamo: $ \acr,\ f: A \to \R,\ x_0 \in A \cap D(A)$\\
$ x_0$ è un punto di accumulazione, non isolato e $f$ una funzione con dominio
$A$ e codominio $\R$.\\
Se $f$ è derivabile in $x_0$ allora $f$ è continua in $x_0$.\\
Questa nozione è dimostrabile sapendo che: $\limxo f(x) = f(x_0) $, e la tesi 
viene provata dal fatto che $x_0$ è un punto di accumulazione del suo dominio $A$.
\end{defn}

\begin{esem}
Non è vero l'opposto di quanto abbiamo appena affermato: esistono infatti funzioni
continue non derivabili, come per esempio la funzione \bt{modulo}:

\begin{equation}
	f(x) = | x |
\end{equation}
\end{esem}

\begin{dimo}
	Lo si può facilmente dimostrare per il limite destro e sinistro in 0:

	\begin{equation}
		\nexists \limxo | x | :
		\begin{cases}
			\lim_{\xtx^+} |x|, & 1\\
			\lim_{\xtx^-} |x|, & -1
		\end{cases}
	\end{equation}
	Dato che i due limiti non coincidiono il limite, nel punto $x_0$ non esiste.
\end{dimo}

\begin{esem}
	La derivata di $ f'(e^x) = e^x$
\end{esem}
Alcune proprietà delle derivate:

\begin{itemize}
	\item La somma delle derivate è la derivata della somma, ed è derivabile
		in $x_0$:
	
		\begin{equation}
			(f+g)'(x_0) = f'(x_0) + g'(x_0)
		\end{equation}
	\item La regola di Leibniz: ($f\cdot g$ è derivabile in $x_0$)

		\begin{equation}
			(f \cdot g)'(x_0)= f'(x_0)g(x_0) +
					   f(x_0)g'(x_0)
		\end{equation}
	\item La derivata del quoziente (ponendo $g(x_0) \neq 0$),
		$\frac{f}{g}$ è derivabile in $x_0$;

		\begin{equation}
			\left( \frac{f}{g} \right)' (x_0) =
			\frac{f' (x_0) g(x_0) - f(x_0) g' (x_0)}{g^2 (x_0)} 
		\end{equation}
	\item La derivata della composta è:\\
		Prendiamo due funzioni tali che:\\
		$A,B \subseteq \R, f: A \to \R, g: B \to \R, f(A) \subseteq B;\
		x_0 \in A \cap D(A), f$ derivabile in $x_0,\ f(x_0) \in D(B), g$
		derivabile in $f(x_0)$:

		\begin{equation}
			(g \circ f)' (x_0) = g' (f(x_0)) \cdot f' (x_0)
		\end{equation}
	\item L'inversa della derivata è la complementare della derivata:

		\begin{equation}
			\left( f^{-1} \right)' (y_0) = \frac{1}{f'(x_0)} 
		\end{equation}
\end{itemize}

\begin{defn}[Estremanti massimo, minimi e relativi]
	Sia $\acr, f: A \to \R$ derivabile, sia $x_0 \in \dot{A}$ ( che vuol dire 
	che esiste un intorno di $x_0$ tutto dentro a ad $A$).\\
	Se $x_0$ è un punto \bt{estremante relativo} (min o max realtivo) allora 
	$f'(x_0) = 0$
\end{defn}

\begin{dimo}
	Prendiamo come esempio $x_0$ max relativo, allora sarà vero:

	\begin{equation}
		\exists \rho > 0: f(x)- f(x_0) \leq 0\ \forall x \in ] 
			x_0 - \rho, x_0 + \rho [
	\end{equation}
	Sapendo che $x_0$ è tutto interno, supponiamo che sia tutto incluso
	in $A$:

	\begin{equation*}
		\frac{f(x)-f(x_0)}{x - x_0} \leq 0\ \forall x \in 
		] x_0, x_0 + h [
	\end{equation*}

	\begin{equation*}
		\frac{f(x)-f(x_0)}{x - x_0} \geq 0\ \forall x \in 
		] x_0 - h, x_0 [
	\end{equation*}
	Dalla derivabilità della funzione possiamo capire che:

	\begin{equation}
		\begin{cases}
			f'(x_0)= \lim_{x \to x_0^+} \frac{f(x) - f(x_0)}{x - x_0}
				\leq 0\\
			f'(x_0)= \lim_{x \to x_0^-} \frac{f(x) - f(x_0)}{x - x_0}
				\geq 0\\
		\end{cases}
	\end{equation}
	Quindi $f'(x_0)$ deve essere uguale a $0$   
\end{dimo}

\subsection{Teoremi Fondamentali delle derivate}%
\label{sub:teoremi_fondamentali_delle_derivate}

\begin{defn}[Rolle]
Prendiamo due numeri naturali, di cui uno strettamente maggiore dell'altro: 
$ a,b \in \R; a < b $ e una funzione continua nell'intervallo formato da questi 
due numeri: $ f \in C ( [a,b] )  $, e questa funzione è derivabile nell'insieme
formato dai due numeri, estremi esclusi: $ ] a,b [ $ Se l'immagine del primo 
elemento è uguale a quella dell'altro $ f ( a ) = f ( b )  $ allora 
è vero che:

\begin{equation}
	\exists c \in ]a,b[: f'( c ) = 0	
\end{equation}
Esiste un elemento all'interno dell'intervallo formato dai due punti, la cui
derivata è uguale a 0.
\end{defn}

\begin{dimo}
	La dimostrazione di questo teorema deriva dal fatto che:\\
	essendo la funzione limitata, applicando il teorema di Weierstrass
	sappiamo che esistono due punti della funzione $ x_1, x_2 \in [ a,b ] $
	che sono punti di $ min, max $ assoluti nell'intervallo.\\
	Quindi, o le due immagini sono uguali, se e solo se
	sono i due estremi $ x_1, x_2 \in { a,b } $:

	\begin{equation}
		f ( x_1 ) = min f,\ f ( x_2 ) = max f \Longrightarrow
		f ( x_1 ) = f ( x_2 ) 
	\end{equation}
	La funzione è costante e quindi la derivata di un qualsiasi punto:

	\begin{equation}
		f'(c) = 0 \forall c \in ]a,b[
	\end{equation}
	Se invece uno dei punti è all'interno dell'intervallo
	$ x_1 \in ]a,b[ $, abbiamo prima dimostrato che se un elemento 
	ha un intorno tutto all'interno di un intervallo ed esso
	è $ min,max $ relativo la sua derivata è 0

\end{dimo}

\begin{defn}[Valor medio o Lagrange]
Siano $ a,b \in \R, a < b, f: [ a,b ]  \to \R, f \in C ( [ a,b ]  ), f  $ derivabile
all'interno dell'intervallo dei due punti $ ]a,b[ $, allora:

\begin{equation}
	\exists c \in ]a,b[ : \frac{f ( b ) - f ( a ) }{b-a} = f' ( c )  
\end{equation}

\end{defn}

%Lagrange dimostrazione da capire
\begin{dimo}
	Consideriamo una funzione $ g (x): [a,b] \to \R,\ g(x) = f ( x ) -
	\frac{f ( b ) - f ( a ) }{b - a} \cdot ( x - a )  $.\\
	Sostituendo la variabile $ x $ con $ a \lor b $, otteniamo 

	\begin{equation}
		g(a) = f ( a ) = f(a) - \frac{f ( b ) - f ( a ) }{b - a} \cdot 
		( \underset{0}{\cancel{a - a}} ) 
	\end{equation}
	Inoltre:

	\begin{equation}
		g(b) = f(b) - \frac{f ( b ) - f ( a ) }{ \cancel{b - a}} 
		\cdot  ( \cancel{b - a} )
		= \cancel{ f ( b ) - f ( b )} + f ( a ) = f ( a )  
	\end{equation}
	Quindi logicamente

	\begin{equation}
		g (a) = f(a),\ g(b) = f ( a )\quad \implies g(a) = g(b)
	\end{equation}
	Un'altra nozione che abbiamo premesso è che la funzione è continua in 
	$ [ a,b ]  $ e derivabile in $ ]a,b[ $, dalle conclusioni del 
	teorema di Rolle:

	\begin{equation}
		\exists c \in ]a,b[: g'(c) = 0		
	\end{equation}
	Che non vuol dire niente di meno di quanto abbia affermato nella
	nostra tesi:

	\begin{equation}
		f' ( c ) = \frac{f ( b ) - f ( a )}{b-a}
	\end{equation}
\end{dimo}


\begin{defn}[Cauchy]
Siano $ a,b \in \R, a<b,\ f,g \in C ( [ a,b ]  )  $ e derivabili 
in $ ]a,b[,\ g' \neq  0 $, allora:

\begin{equation}
	\exists c \in ]a,b[ : \frac{f ( b ) - f ( a ) }{ b - a } = 
	\frac{f' ( c ) }{g' (c) } 
\end{equation}
Questo si può facilemente dimostrare applicando il teorema di 
Rolle
\end{defn}
Delle brevi osservazioni:\\
Se abbiamo un intervallo non vuoto $ I \subseteq \R $ e una funzione $ f: I \to \R $
derivabile su questo intervallo, se: 

\begin{equation}
	f' ( x ) = 0, \ \forall x \in I 
\end{equation}
Allora la funzione è costante in tutto l'intervallo.\\
Se invece:

\begin{equation}
	f' ( x ) \geq 0 \ \forall x \in I	
\end{equation}
La funzione è monotona crescente nell'intervallo. Viceversa se è monotona
crescente allora sarà sempre minore uguale a 0 la derivata della funzione.


\begin{defn}[Darboux]
Sia $ I \subseteq \R  \neq \varnothing $ un intervallo non vuoto e sia 
$ f: I \to \R $ una funzione derivabile nell'intervallo, alllora:

\begin{equation}
	f' ( I ) = { f' ( x ) : x\in I } 
\end{equation}
È un intervallo di $ \R $.\\
% dimostrazione facoltativa
\end{defn}

\begin{defn}[Primitive]
Sia $ I \subseteq \R $ un intervallo e sia una funzione $ f: I \to \R $,
diremo che la primitva di questa funzione è una qualsiasi funzione derivabile
$ \phi : I \to \R $ tale che:

\begin{equation}
	\phi' (x) = f ( x )\ \forall x \in I
\end{equation}
In parole povere la primitiva è come se fosse "l'operazione inversa"
alla derivata, ovvero la primitiva di una funzione è l'immagine di partenza
$f ( x ) $.\\
Il teorema di Darboux dice inoltre che se esiste una primitiva di una funzione 
$ f $ su $ I $ allora $ f ( I ) = \phi ' (I) $ è un intervallo, questa 
è una condizione necessaria affinchè la funzione abbia una primitiva.
\end{defn}

\begin{defn}[]
\end{defn}

























\subsection{Tablle Utili}%
\label{sub:tablle_utili}



\begin{center}
\begin{TAB}(r,0.5cm,1cm)[5pt]{|c|c|}{|c|c|c|c|c|c|c|c|c|c|c|c|c|c|c|c|c|c|c|c|c|}
	Funzione $f(x)$ & Derivata della funzione $f'(x_0)$\\
	$ k $ & 0\\
	$ |x| $ &  $sgn x$\\
	$ x^\alpha $ & $\alpha x^{\alpha -1}$\\
	$ x $ & $1$\\
	$ \log_a x $ & $\frac{1}{x}\log_a e$\\
	$ \log x $ & $\frac{1}{x}$\\
	$ \log_a |x| $ & $ \frac{1}{|x|} $\\
	$ \log |x| $ & $ \frac{1}{x}  $\\
	$ \log | f(x) | $ & $ \frac{f'(x_0)}{f(x)}  $\\
	$ a^x  $ & $ a^x \log a$\\
	$ e^x  $ & $ e^x $ \\
	$ \sinh x $ & $ \cosh x $\\
	$ \cosh	x $ & $ \sinh x $\\
	$ \sin x $ & $ \cos x $\\
	$ \cos x $ & $ - \sin x $\\
	$ \arcsin x $ & $ \frac{1}{\sqrt[]{1-x^2}}$\\
	$ \arccos x $ & $ - \frac{1}{\sqrt[]{1-x^2}}  $\\
	$ \tan x $ & $ 1 + \tan^2 x $\\
	$ \arctan x  $ & $ \frac{1}{1+x^2}  $\\
	$ \sqrt[]{x} $ & $ \frac{1}{2 \cdot \sqrt[]{x}}  $\\
\end{TAB}
	
\end{center}

\clearpage



%appunti lezione
Integrali per parti
\end{document}

