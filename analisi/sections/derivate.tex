\documentclass[../analisi.tex]{subfiles}

\begin{document}


\section{Derivate}%
\label{sec:derivate}

\subsection{Rapporto incrementale e derivate}%
\label{sub:rapporto_incrementale_e_derivate}
Per poter  comprendere le derivate è essenziale comprendere il concetto di 
\bt{rapporto incrementale}\\

\begin{defn}[Rapporto Incrementale]
Premettiamo: $ \acr,\ f: A \to \R,\ x_0 \in A \cap D(A)$\\
$ x_0$ è un punto di accumulazione, non isolato e $f$ una funzione con dominio
$A$ e codominio $\R$.\\

\begin{equation}
	R_f ( x_0 ): A \setminus \{ x_0 \} \to \R, \quad 
	R_f ( x_0 ) (x) = \frac{f(x) - f(x_0) }{x - x_0} 
\end{equation}
Una funzione si dice \bt{derivabile} in $x_0$ se esiste il limite, ed è finito:

\begin{equation}
	\limxo R_f ( x_0 ): A \setminus \{ x_0 \} \to \R, =
	\limxo R_f ( x_0 ) (x) = \frac{f(x) - f(x_0) }{x - x_0} 
\end{equation}
Il limito che abbiamo appena definito si chiama \bt{derivata} di $f$ in $x_0$
\begin{equation}
	f'( x_0 ) =
	\limxo R_f ( x_0 ) (x) = \frac{f(x) - f(x_0) }{x - x_0},\
	f'(x_0) \in \R
\end{equation}
\end{defn} 
Se il limite, del rapporto incrementale, non appartiene ai numeri reali ed è 
$\pm \infty$, allora la funzione è derivabile in senso esteso.\\
Il limite del rapporto incrementale si può riscrivere come:

\begin{equation}
	\lim_{h \to 0} \frac{f(x_0 + h ) - f(x_0) }{h} \qquad (h = x - x_0)
\end{equation}

\begin{defn}
	Se una funzione $f$ è derivabile in un punmto $x_0$ allora:

	\begin{equation}
		\exists \lambda \in \R, \exists \omega : A \to \R\
		\omega (x) \to \omega (x_0)  = 0 | \xtx
	\end{equation}
	Allora:

	\begin{equation}
		\begin{aligned}
		f(x) =& f(x_0) + \lambda \cdot ( x - x_0 ) + \omega (x) (x-x_0)\ 
		\forall x \in A\\
		\lambda = &\frac{f(x) - f(x_0)}{x - x_0} \qquad w(x) 
		\text{è infinitesima (=0)}\\
		\lambda = & f'( x_0 )
		\end{aligned}
	\end{equation}
\end{defn}


\begin{defn}
Premettiamo: $ \acr,\ f: A \to \R,\ x_0 \in A \cap D(A)$\\
$ x_0$ è un punto di accumulazione, non isolato e $f$ una funzione con dominio
$A$ e codominio $\R$.\\
Se $f$ è derivabile in $x_0$ allora $f$ è continua in $x_0$.\\
Questa nozione è dimostrabile sapendo che: $\limxo f(x) = f(x_0) $, e la tesi 
viene provata dal fatto che $x_0$ è un punto di accumulazione del suo dominio $A$.
\end{defn}

\begin{esem}
Non è vero l'opposto di quanto abbiamo appena affermato: esistono infatti funzioni
continue non derivabili, come per esempio la funzione \bt{modulo}:

\begin{equation}
	f(x) = | x |
\end{equation}
\end{esem}

\begin{dimo}
	Lo si può facilmente dimostrare per il limite destro e sinistro in 0:

	\begin{equation}
		\nexists \limxo | x | :
		\begin{cases}
			\lim_{\xtx^+} |x|, & 1\\
			\lim_{\xtx^-} |x|, & -1
		\end{cases}
	\end{equation}
	Dato che i due limiti non coincidiono il limite, nel punto $x_0$ non esiste.
\end{dimo}

\begin{esem}
	La derivata di $ f'(e^x) = e^x$
\end{esem}
Alcune proprietà delle derivate:

\begin{itemize}
	\item La somma delle derivate è la derivata della somma, ed è derivabile
		in $x_0$:
	
		\begin{equation}
			(f+g)'(x_0) = f'(x_0) + g'(x_0)
		\end{equation}
	\item La regola di Leibniz: ($f\cdot g$ è derivabile in $x_0$)

		\begin{equation}
			(f \cdot g)'(x_0)= f'(x_0)g(x_0) +
					   f(x_0)g'(x_0)
		\end{equation}
	\item La derivata del quoziente (ponendo $g(x_0) \neq 0$),
		$\frac{f}{g}$ è derivabile in $x_0$;

		\begin{equation}
			\left( \frac{f}{g} \right)' (x_0) =
			\frac{f' (x_0) g(x_0) - f(x_0) g' (x_0)}{g^2 (x_0)} 
		\end{equation}
	\item La derivata della composta è:\\
		Prendiamo due funzioni tali che:\\
		$A,B \subseteq \R, f: A \to \R, g: B \to \R, f(A) \subseteq B;\
		x_0 \in A \cap D(A), f$ derivabile in $x_0,\ f(x_0) \in D(B), g$
		derivabile in $f(x_0)$:

		\begin{equation}
			(g \circ f)' (x_0) = g' (f(x_0)) \cdot f' (x_0)
		\end{equation}
	\item L'inversa della derivata è la complementare della derivata:

		\begin{equation}
			\left( f^{-1} \right)' (y_0) = \frac{1}{f'(x_0)} 
		\end{equation}
\end{itemize}

\begin{defn}[Estremanti massimo, minimi e relativi]
	Sia $\acr, f: A \to \R$ derivabile, sia $x_0 \in \dot{A}$ ( che vuol dire 
	che esiste un intorno di $x_0$ tutto dentro a ad $A$).\\
	Se $x_0$ è un punto \bt{estremante relativo} (min o max realtivo) allora 
	$f'(x_0) = 0$
\end{defn}

\begin{dimo}
	Prendiamo come esempio $x_0$ max relativo, allora sarà vero:

	\begin{equation}
		\exists \rho > 0: f(x)- f(x_0) \leq 0\ \forall x \in ] 
			x_0 - \rho, x_0 + \rho [
	\end{equation}
	Sapendo che $x_0$ è tutto interno, supponiamo che sia tutto incluso
	in $A$:

	\begin{equation*}
		\frac{f(x)-f(x_0)}{x - x_0} \leq 0\ \forall x \in 
		] x_0, x_0 + h [
	\end{equation*}

	\begin{equation*}
		\frac{f(x)-f(x_0)}{x - x_0} \geq 0\ \forall x \in 
		] x_0 - h, x_0 [
	\end{equation*}
	Dalla derivabilità della funzione possiamo capire che:

	\begin{equation}
		\begin{cases}
			f'(x_0)= \lim_{x \to x_0^+} \frac{f(x) - f(x_0)}{x - x_0}
				\leq 0\\
			f'(x_0)= \lim_{x \to x_0^-} \frac{f(x) - f(x_0)}{x - x_0}
				\geq 0\\
		\end{cases}
	\end{equation}
	Quindi $f'(x_0)$ deve essere uguale a $0$   
\end{dimo}



\begin{center}
\begin{TAB}(r,0.5cm,1cm)[5pt]{|c|c|}{|c|c|c|c|c|c|c|c|c|c|c|c|c|c|c|c|c|c|c|c|c|}
	Funzione $f(x)$ & Derivata della funzione $f'(x_0)$\\
	$ k $ & 0\\
	$ |x| $ &  $sgn x$\\
	$ x^\alpha $ & $\alpha x^{\alpha -1}$\\
	$ x $ & $1$\\
	$ \log_a x $ & $\frac{1}{x}\log_a e$\\
	$ \log x $ & $\frac{1}{x}$\\
	$ \log_a |x| $ & $ \frac{1}{|x|} $\\
	$ \log |x| $ & $ \frac{1}{x}  $\\
	$ \log | f(x) | $ & $ \frac{f'(x_0)}{f(x)}  $\\
	$ a^x  $ & $ a^x \log a$\\
	$ e^x  $ & $ e^x $ \\
	$ \sinh x $ & $ \cosh x $\\
	$ \cosh	x $ & $ \sinh x $\\
	$ \sin x $ & $ \cos x $\\
	$ \cos x $ & $ - \sin x $\\
	$ \arcsin x $ & $ \frac{1}{\sqrt[]{1-x^2}}$\\
	$ \arccos x $ & $ - \frac{1}{\sqrt[]{1-x^2}}  $\\
	$ \tan x $ & $ 1 + \tan^2 x $\\
	$ \arctan x  $ & $ \frac{1}{1+x^2}  $\\
	$ \sqrt[]{x} $ & $ \frac{1}{2 \cdot \sqrt[]{x}}  $\\
\end{TAB}
	
\end{center}



\end{document}

