\documentclass[12pt, a4paper]{article}
\usepackage[utf8]{inputenc}

\usepackage{graphicx, float, amssymb, wrapfig, amsthm, 
	    enumitem, amsmath, mathtools, ulem}

\graphicspath{ {pics/} }


% break style 
\newtheoremstyle{break}
{\topsep}
{\topsep}%
{\itshape}
{}%
{\bfseries}
{:}%
{\newline}
{}%

\newtheoremstyle{lemma}% style name
{2ex}% above space
{2ex}% below space
{\upshape}% body font
{}% indent amount
{\scshape}% head font
{.}% post head punctuation
{\newline}% post head punctuation
{}% head spec

%theorem styles

\theoremstyle{break}
\newtheorem{defn}{Definizione}
\AfterEndEnvironment{definizione}{\noindent\ignorespaces}


\theoremstyle{lemma}
\newtheorem{eser}{Esercizio}


\theoremstyle{lemma}
\newtheorem{dimo}{Dimostrazione}

\theoremstyle{lemma}
\newtheorem{esem}{Esempio}

% custom fraction
\newcommand*{\bfrac}[2]{\genfrac{\lbrace}{\rbrace}{0pt}{}{#1}{#2}}
\title{Ripasso}
\author{Corso Informatica UNIPD A.A 2021/2022}
\date{\today}

\begin{document}
\maketitle

\section{PRINCIPIO D'INDUZIONE E BINOMIO DI NEWTON}
\begin{dimo}[IPOTESI: $P(n)$ è vera]
	
	\begin{equation}
	P(n)\ =\quad (a+b)^n\ =\ \sum_{k=0}^{n}\ (_{n}^{k})\ a^k\ b^{n-k}
	\end{equation}
Per il principio di induzione dovremmo essere capaci di dimostrare anche: 
$P(n+1)$ come vera.

	\begin{equation}
	P(n+1)\ =\quad (a+b)^{n+1}\ =\ 
	\sum_{k=0}^{n+1}\ (_{n+1}^{k})\ a^k\ b^{n+1-k} 
	\end{equation}

	\begin{equation}
	(a+b)^{n+1}\ =\ (a+b)(a+b)^n\ \Rightarrow\ 
	(a+b)\sum_{k=0}^{n}\ (_{n}^{k})\ a^k\ b^{n-k}
	\end{equation}
Ricordiamo la proprietà distributiva: $(a+b)c\ =\ ac + bc$

	\begin{equation}
	\sum_{k=0}^{n}\ (_{n}^{k})\ a^{k+1}\ b^{n-k}\ +\ 
	\sum_{k=0}^{n}\ (_{n}^{k})\ a^k\ b^{n+1-k}
	\end{equation}

	\begin{equation} 
	\big\Downarrow
	\end{equation}

	\begin{equation}
	\sum_{k=1}^{n+1}\ (_{n}^{k-1})\ a^{k}\ b^{n+1-k}\ + \ 
	\sum_{k=0}^{n}\ (_{n}^{k})\ a^k\ b^{n+1-k}\ + (_{n}^{0})\ a^0\ b^{n+1-0}
	\end{equation}

	\begin{equation} 
	\big\Downarrow
	\end{equation}
	
	\begin{equation}
	\sum_{k=0}^{n+1}\ [(_{n}^{k-1}) + (_{n}^{k})]\ a^{k}\ b^{n+1-k}\
	\end{equation}
	
	\begin{equation}
	\big\Downarrow
	\end{equation}
	
	\begin{equation}
	\frac{n!}{(k-1)!(n-k+1)!}\ +\ \frac{n!}{k!(n-k)!}\ =\
	\frac{\underline{n!}}{(\underline{k-1)!}(n-k+1)(\underline{n-k)!}}\ +\ 
	\frac{\underline{n!}}{k\underline{(k-1)!}\ \underline{(n-k)!}}
	\end{equation}
			
	\begin{equation}
	\big\Downarrow
	\end{equation}
	
	\begin{equation} 
	\frac{n!}{(k-1)!(n-k)!}\ \left( \frac{1}{(n-k+1)}\ + \frac{1}{k} \right)\
	\end{equation}

	\begin{equation}
	\big\Downarrow
	\end{equation}
	

	\begin{equation}
	\frac{n!}{(k-1)!(n-k)!}\ \cdot\ 
	\frac{\xout{k} + n - \xout{k} + 1}{k (n-k+1)}\
	\end{equation}

	\begin{equation}
	\big\Downarrow
	\end{equation}
	
	
	\begin{equation}
	\overbrace{
	\frac{n!}{(k-1)!(n-k)!}\ \cdot\
	\frac{n+1}{k(n-k+1)}\ } \quad =
	\frac{(n+1!)}{k!((n+1)-k)!}
	\end{equation}
L'ultima semplificazione è conseguenza di: $(n+1)!\ =\ (n+1) \cdot n!$
e di $(n-1)!\ \cdot \ n\ =\ n!$

In conclusione abbiamo che:
	\begin{equation}
	\sum_{k=0}^{n+1}\ [(_{n}^{k-1}) + (_{n}^{k})]\ a^{k}\ b^{n+1-k}\ =
	\sum_{k=0}^{n+1}\ (_{n+1}^{k})\ a^{k}\ b^{n+1-k}\ =
	(a + b)^{n+1}
	\end{equation}

\end{dimo}
\end{document}
