\input{preamble}

\begin{document}
\maketitle

\begin{defn}[Teorema dei 2 carabinieri]

Se $ \underbrace{ \{ a_n \}\ , \{ b_n \} }_{\mbox{convergono a $\ l$}} , \{ c_n \}$

\begin{equation}
	\mbox{è ovvio che:}\ a_n \leq c_n \leq b_n \ \Longrightarrow\ 
	c_n \mbox{converge a}\ l
\end{equation}
\end{defn}

\begin{dimo}
\begin{equation}
	\forall \epsilon > 0, \exists \overline{n_1}, \overline{n_2} \in \N :
\end{equation}

\begin{equation*}
	\big\Downarrow
\end{equation*}

\begin{equation}
	l - \epsilon < a_n < l + \epsilon \qquad \& \qquad
	l - \epsilon < b_n < l + \epsilon
\end{equation}
se $ n > \max \{ \overline{n_1}, \overline{n_2} \} $

\begin{equation*}
	\big\Downarrow
\end{equation*}

\begin{equation}
	l - \epsilon < a_n \leq c_n \leq b_n  < l + \epsilon \qquad
	\forall n > \overline{n}
\end{equation}

\begin{equation}
	\underbrace{ l - \epsilon < c_n < l + \epsilon }_{
	| c_n - l | < \epsilon } \Longrightarrow 
	\lim \limits_{n \to +\infty} c_n = l
\end{equation}

\end{dimo}
 
\end{document}
