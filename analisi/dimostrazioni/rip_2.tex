\documentclass[12pt, a4paper]{article}
\usepackage[utf8]{inputenc}

\usepackage{graphicx, float, amssymb, wrapfig, amsthm, 
	    enumitem, amsmath, mathtools, ulem}

\graphicspath{ {pics/} }


% break style 
\newtheoremstyle{break}
{\topsep}
{\topsep}%
{\itshape}
{}%
{\bfseries}
{:}%
{\newline}
{}%

\newtheoremstyle{lemma}% style name
{2ex}% above space
{2ex}% below space
{\upshape}% body font
{}% indent amount
{\scshape}% head font
{.}% post head punctuation
{\newline}% post head punctuation
{}% head spec

%theorem styles

\theoremstyle{break}
\newtheorem{defn}{Definizione}
\AfterEndEnvironment{definizione}{\noindent\ignorespaces}


\theoremstyle{lemma}
\newtheorem{eser}{Esercizio}


\theoremstyle{lemma}
\newtheorem{dimo}{Dimostrazione}

\theoremstyle{lemma}
\newtheorem{esem}{Esempio}

% custom fraction
\newcommand*{\bfrac}[2]{\genfrac{\lbrace}{\rbrace}{0pt}{}{#1}{#2}}

% custom commands
\newcommand{\R}{\mathbb{R}}
\newcommand{\N}{\mathbb{N}}

\title{Ripasso}
\author{Corso Informatica UNIPD A.A 2021/2022}
\date{\today}

\begin{document}
\maketitle

\begin{dimo}[$x^n < y$]

	\begin{equation}
	x^n < y \ \iff\ \exists\epsilon \in \R, 
	\epsilon > 0 \ :\ (x + \epsilon)^n < y
	\end{equation}
Sia $\epsilon \in ]0,1[$

	\begin{equation}
	(x + \epsilon)^n\ =\ ((x + \epsilon)^n - x^n) + x^n\ =\
	((\xout{x} + \epsilon)- \xout{x})((x+ \epsilon)^{n-1}\ +\	
	\dots\ +\ x^{n-1}) + x^n\
	\end{equation}
	
	\begin{equation}
	\big\Downarrow
	\end{equation}

	\begin{equation}
	\epsilon ( (x + \epsilon)^{n-1} + \dots + x^{n-1}) + x^n\ \leq\
	\epsilon \cdot n \cdot (x + 1)^{n-1} + x^n
	\end{equation}

	\begin{equation}
	\big\Downarrow
	\end{equation}
	
	\begin{equation}
	\epsilon n (x+1)^{n-1} + x^n < y \ \iff\ 
	\epsilon < \frac{y - x^n}{n(x+1)^{n-1}} : \overset{def.}{=} \epsilon > 0
	\end{equation}

	\begin{equation}
	0\ <\ \epsilon\ <\ \frac{y - x^n}{n(x+1)^{n-1}} 	
	\end{equation}
Questo è valido per: $ \epsilon \in \R $ 

		
\end{dimo}

\end{document}
