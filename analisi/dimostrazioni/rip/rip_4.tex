\documentclass[../rip.tex]{subfiles}

\begin{document}

\begin{dimo}[Il limite se esiste è unico]
\begin{equation}
	\lim \limits_{x \to \infty} a_n = l \quad \land \quad 
	\lim \limits_{x \to \infty} a_n = m \quad
	\iff \quad l\ =\ m
\end{equation}
\end{dimo}

\begin{esem}
Poniamo per assurdo che $l \neq m$
Fissiamo $\epsilon > 0$
\begin{equation}
	\underbrace{
	\underbrace{| a_n - l | < \frac{\epsilon}{2} }_{ n>\overline{n_1} }
	\quad \& \quad 
	\underbrace{| a_n - m | < \frac{\epsilon}{2} }_{ n>\overline{n_2}} }_{
	n\ >\ \max \{ \overline{n_1}, \overline{n_2} \} }
\end{equation}

\begin{equation*}
	\big\Downarrow
\end{equation*}
Ricordiamo che $ | a_n - m | = | m - a_n |$

\begin{equation}
	| \cancel{-a_n} - l - \cancel{-a_n} + m |
	| a_n - l |\ +\ |m - a_n | < 
	\frac{\epsilon}{2} + \frac{\epsilon}{2} = \epsilon
\end{equation}


\begin{equation*}
	\big\Downarrow
\end{equation*}

\begin{equation}
	| m - l | < \epsilon \quad \Longrightarrow \quad | m - l | = 0
\end{equation}
Ma questo è assurdo perchè: $\epsilon > 0, \forall \epsilon \in \R$

\begin{equation}
	m = l 
\end{equation}
\end{esem}


\end{document}

