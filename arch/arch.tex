\documentclass[12pt, a4paper]{article}
\usepackage[utf8]{inputenc}

	\usepackage{graphicx, float, amssymb, wrapfig, amsthm, enumitem, amsmath, mathtools}

\graphicspath{ {pics/} }


% break style 
\newtheoremstyle{break}
{\topsep}
{\topsep}%
{\itshape}
{}%
{\bfseries}
{:}%
{\newline}
{}%

\newtheoremstyle{lemma}% style name
{2ex}% above space
{2ex}% below space
{\upshape}% body font
{}% indent amount
{\scshape}% head font
{.}% post head punctuation
{\newline}% post head punctuation
{}% head spec

%theorem styles

\theoremstyle{break}
\newtheorem{defn}{Definizione}
\AfterEndEnvironment{definizione}{\noindent\ignorespaces}


\theoremstyle{lemma}
\newtheorem{eser}{Esercizio}


\theoremstyle{lemma}
\newtheorem{dimo}{Dimostrazione}

\theoremstyle{lemma}
\newtheorem{esem}{Esempio}

% custom fraction
\newcommand*{\bfrac}[2]{\genfrac{\lbrace}{\rbrace}{0pt}{}{#1}{#2}}
\title{Appunti Architettura}
\author{Andreas Araya Osorio}
\date{\today}

\begin{document}
\maketitle


\section{ TODO }
\subsection{Interruzioni}
Il meccansimo tramite il quale dei moduli possono interrompere la normale di sequenza di esecuzione. \newline

\begin{itemize}
	\item Program
	\item Timer
	\item I/O
	\item Guasto Hardware
	\end{itemize}
Si interrompe per
\begin{itemize}
	\item efficienza elaborazione
\end{itemize}

Ciclo interruzione:
\begin{itemize}
	\item viene aggiunto al ciclo di esecuzione
	\item la cpu controlla (fetch) le interruzioni pendenti
	\item se non ce ne sono, prende la prossima istruzione
	\item se ce ne sono: 
	\begin{itemize}
		\item sospende esecuzione
		\item salva contesto
		\item imposta il pc all'indirizzo di inizio del programma di gestione
		\item esegue il programma di gestione dell'hardware
		\item rimette il contesto al suo posto e continua il programma interrotto
	\end{itemize}
\end{itemize}
In caso di interruzioni multiple: esistono vari livelli di interruzione. Le int. di basso livello hanno minore priorità rispetto a quelle di alto livello. 
Il sistema operativo blocca quelle di basso livello per risolvere quelle di alto livello e così via
\subsection{Connessioni}
Tutti i componenti \textbf{devono} essere connessi \newline
Esistono vari tipi di connessioni per vari tipi di componenti
\begin{itemize}
	\item CPU
	\item Memoria 
	\item I/O
\end{itemize}

\subsection{Bus}
Tutti i dispostivi sono collegati dal bus di sistema \newline
Il bus:
\begin{enumerate}
	\item collega \textbf{2 o più} dispositivi
	\item mezzo trasmissione condiviso
	\item un segnale trasmesso ad un bus è disponibile a tutti i dispositivi
	\item arbitro bus: solo un dispostivo alla volta può trasmettere
	\item varie linee di comunicazione ( trasmettono uno 0 o un 1)
	\item varie linee trasmettono in parallelo numeri binari. Un bus da 8 bit trasmette un dato di 8 bit
\end{enumerate}

Bus di sistema:
\begin{itemize}
	\item connette cpu, i/o, M
	\item da 50 a qualche centinaio di linee
	\item 3 gruppi di linee
	\begin{enumerate}
		\item bus dati
		\item indirizzi
		\item controllo
	\end{enumerate}
\end{itemize}

Bus dati:
\begin{itemize}
	\item trasporta dati o istruzioni
	\item ampiezza --$ > $ efficienza  del sistema
	\begin{itemize}
		\item con poche linee --$>$ accessi in memoria
	\end{itemize}
\end{itemize}

Bus indirizzi
\begin{itemize}
	\item indica sorgente o destinazione dati
	\item l'ampiezza determina la massima quantità di M indirizzabile
\end{itemize}

Bus controllo
\begin{itemize}
	\item per controllare accesso, uso linee dati e indirizzi
	\begin{enumerate}
		\item M write
		\item M read
		\item richiesta bus
		\item bus grant
		\item interrupt request
		\item clock
	\end{enumerate}
\end{itemize}

Bus usage:
se un modulo vuole inviare dati ad un altro:
\begin{itemize}
	\item bus grant
	\item data transfer
\end{itemize}

se un module vuole ricevere dati da un altro:
\begin{itemize}
	\item bus grant
	\item trasferire una richiesta all'altro modulo sulle linee di controllo 
	\item attendere invio dati
\end{itemize}

Bus singoli e multipli
\begin{itemize}
	\item singolo bus = ritardo e congestione
	\item vari bus = risoluzione problema
\end{itemize}

\subsection{Temporizzazione}
\begin{itemize}
	\item Coordinazione degli eventi su un bus
	\item Sincrona
	\begin{itemize}
		\item clock determined events
		\item single clock line
		\item single sequence is a clock cicle
		\item every device connected to the bus can read the clock line
		\item every event starts at the beginning of a clock cycle
	\end{itemize}
\end{itemize}



\end{document}
