\documentclass[12pt, a4paper]{article}
\usepackage{appunti}

\title{Appunti di Architettura degli Elaboratori}
\author{Andreas Araya Osorio}
\date{\today}

\begin{document}
\maketitle



\section{Rappresentazione dei numeri}

I numeri possono essere rappresentati in qualsiasi base, noi utilizziamo la base
10 come conseguenza del numero delle nostre dita.\\
I \bt{calcolatori} utilizzano la base 2, ovvero il binario, che può facilmente
essere ricondotta alla condizione di uno stato elettrico
o \bt{positivo (1)} o \bt{negativo (0)}\\
Per la rappresentazione di un numero in qualsiasi base:
\begin{defn}
La sequenza di $k$ cifre:

\begin{equation}
	d_k \cdot d_{k-1} \cdot \dots \cdot d_1 \cdot d_0
\end{equation}
E questa sequenza la moltiplichiamo per la base scelta:

\begin{equation}
	d_k \times b^k \cdot d_{k-1} \times b^{k-1} \cdot 
	\dots \cdot d_1 \times b^1 \cdot d_0 \times b^0
\end{equation}
dove \bt{b} è la base da noi scelta.
\end{defn}

\begin{defn}[Numero di cifre in una base N]
In una qualsiasi base $N$ il numero di cifre equivale a:


\begin{equation}
	\mbox{cifre} = N - 1, N - 2, \dots, 1, 0.
\end{equation}
\end{defn}
\subsection{Binario}
Con la base 2 abbiamo le cifre dallo 0 allo 1.\\
Per la definizione precedente la rappresentazione sarà:


\begin{equation}
	d_k \times 2^k \cdot d_{k-1} \times 2^{k-1} \cdot 
	\dots \cdot d_1 \times 2^1 \cdot d_0 \times 2^0
\end{equation}

\begin{defn}[Valore minimo e massimo]
Il valore minmo e massimo di un numero di $n$ cifre è:	
	\begin{itemize}
		\item \bt{Valore minimo} = $ 000000\dots00(n\ \mbox{times}) = 0$
		\item \bt{Valore massimo} =$ 111111\dots11(n\ \mbox{times}) = 2^n-1$
		
		\begin{equation}
			2^{n-1} + 2^{n-2} + \dots + 2^2 + 2^1 + 2^0 = 2^n -1
		\end{equation}
	\end{itemize}
\end{defn}

\begin{esem}


\begin{equation} 
	n = 3 \implies 111 = 2^2 + 2^1 + 2^0 = 7 = 2^3 - 1 = 8 - 1
\end{equation}
\end{esem}

\subsection{Notazione}
\begin{defn}[BIT]
\bt{Bit} = binary digit, uno dei due simboli (0, 1) del sistema numerico binario.
Esso è l'unità elementare dell'informazione trattata da un elaboratore.\\
Numeri di 8, 16, 32 bit equivale in base 10 
a parlare di numeri a 3, 4, 5, ... cifre \\
\end{defn}

\begin{defn}[BYTE]
\bt{Byte} = 8 bit, è una sequenza di bit, convenzionalmente l'unità di misura
delle capacità di una memoria.\\
Può assumere $2^8 = 256 (0\ -\ 255)$ possibili valori
\end{defn}

\begin{defn}[Parola]
\bt{Word/Parola} = corrisponde a 16, 32 o 64 bit in base al tipo di IS
(Instruction Set), essa è l'unità più piccola di informazione su cui un elaboratore
può intervenire.
\end{defn}

\input{sub/byte_tab}
Convenzionalmente si utlizzano come unità di misura: 
\begin{itemize}
	\item il B(yte) per la capacità di una memoria
	\item il b(it)/s per la velocità di trasmissione di dati.
\end{itemize}
\subsection{Decimale}
Nella rappresentazione decimale abbiamo le cifre dallo 0 al 9.\\
Un qualsiasi numero in base decimale si può rappresentare come:


\begin{equation}
	d_k \times 10^k \cdot d_{k-1} \times 10^{k-1} \cdot 
	\dots \cdot d_1 \times 10^1 \cdot d_0 \times 10^0
\end{equation}


	\input{sub/dec_tab}

\subsection{Ottale}
Nella rappresentazione ottale si hanno le cifre dallo 0 al 7.\\
Questa base viene utilizzata perchè essendo un multiplo di 2 si possono facilmente
convertire numeri ottali in binario:\\
Con 3 cifre si possono rappresentare 8 bit.
In questo modo si possono avere numeri più facilmente maneggiabili da umani,
rispetto a lunge stringhe di 0 o 1.

\input{sub/oct_tab}
\subsection{Esadecimale}

Nella rappresentazione esadecimale si hanno 15 cifre: dallo 0, ..., 9, A, ..., F.\\
Un simbolo (cifra) in questa base rappresenta 4 cifre binarie (4 bit).\\
Con 4 cifre esadecimali si possono rappresentare 16 bit.

\input{sub/hex_tab}
\subsection{Algebra di Boole}
Essa viene utillizzata per la specifica di funzioni logiche.\\
Una qualsiasi variabile può assumere 2 valori:
\bt{vero} o \bt{falso}

\begin{defn}[Operazoini logiche di base]
	\begin{equation}
	\begin{aligned}
		A\ AND\ B\ &=\ A \cdot B\\
		A\ OR\ B\ &=\ A + B\\
		NOT\ A\ &=\ \ol{A}
	\end{aligned}
	\end{equation}
	\input{sub/log_tab}
\end{defn}
Si indica con: $\oplus = xor$, è l'or esclusivo, esso è vero solo quando una delle due variabili
è vera, non quando lo sono entrambe.\\
$\ol{A \cdot B} = nand$, l'opposto dell'AND classico\\
Grazie a XOR e NAND si possono rappresentare tutte le altre funzioni logiche
attraverso delle combinazioni di questi due.

\input{sub/bool_tab}

%TODO Boolean postulates and identitities.


\section{APPROACH}
There's a difference between \\
\bt{Computer architecture} and 
\bt{Computer organization}.

\begin{itemize}
	\item C.Architecture = attributes of a system visible to a programmer. \\
		\bt{ISA} = Instruction set architecture, is a synonym of C.A.
	\item C.Organization = operational units and their interconnections 
		that realize the architectural specifications. Examples:
		\begin{itemize}
		\item instruction set
		\item number of bits used to represent various data types
		\item I/O mechanisms and techniques for adressing memory
		\end{itemize}
\end{itemize} 

\subsection{Structure and function}
A modern computer is a hierarchical system, and is a set of interrelated subssytems.
These have, in turn, have subsystems of their own until we reach the lowest level
of subsystems.
There's a huge difference between:

\begin{itemize}
	\item \bt{structure} the way components are interrelated
	\item \bt{function} the operation of each individual 
\end{itemize}
The way that is considered most efficient and the clearest approach
for describing pc components is:\bt{TOP-DOWN}

\section{ Componenti principali}
Le componenti principali di un elaboratore sono:

\begin{itemize}
	\item CPU
	\item Memoria
	\item Sistemi di I/O
	\item Interconnessioni
\end{itemize}

\begin{defn}[Architettura di Von Neumann]
Secondo questo tipo di architettura, un elaboratore è composto di questi
principali componenti:
\begin{itemize}
	\item dati e istruzioni in memoria
	\item memoria accessibile per indirizzo
	\item esecuzione sequenziale delle istruzioni
\end{itemize}
\end{defn}

\begin{defn}[Programma cablato]
	Conaiste nel costruire i componenti logici in modo tale che il risultato 
	sia quello voluto e non può essere modificato in seguito.\\
	Vuol dire "programmare" a livello hardware, ovvero con le 
	componenti fisiche.\\
	Non è un sistema flessibile, esegue solo operazioni predeterminate.
\end{defn}

\begin{defn}[Programma]
	Un programma è una sequenza di passi\\
	Ogni passo corrsiponde ad un'operazionie logica.\\
	Ogni operazione determina un diverso insieme di segnali di controllo.\\
\end{defn}

\begin{defn}[Programmazione software]
	Nasce con Von Neumann, si parte da un hardware generico, 
	si ha una parte che preleva il codice di una istruzione, è generale:
	L'hardware di cui parliamo si dice \bt{general purpose}, utile a vari scopi.
	Si hanno poi dei segnali di controllo corrispondenti.\\
	Questo sistema è molto più flessibile di quello "cablato".\\
	La CPU assume delle funzioni diverse ovvero:
	
	\begin{itemize}
	\item interprete delle istruzioni
	\item generico modulo per operazioni aritmetico logiche = ALU
	\end{itemize}
	I segnali di controllo sono necessari per far eseguire 
	al giusto modulo la giusta operazione: ALU\\
	ALU prende segnali di controllo ed esegue le istruzioni codificate\\
	In questo sistema si ha la codifica 
	delle istruzioni e la decodifica delle istruzioni.

\end{defn}



\begin{defn}[Memoria principale nell'architettura di Von Neumann]
Si ha la possibilità di salti oltre che all'esecuzione sequenziale(in serie)\\
Per esempio con le operazioni che richiedono accesso a più dati in memoria
nello stesso momento.\\
Inoltre essa ha il compito di immaggazzinare dati e istruzioni
\end{defn}

\begin{esem}
Somma con 2 numeri in locazione di memoria diverse
\end{esem}

\subsection{CPU}
Essa non deve solo eseguire istruzioni ma anche gestire dei segnali di controllo e 
getstire delle risorse.\\
Composta da Vari componenti principali:

\begin{itemize}
\item bt{EU} = execution unit = alu

\item bt{IR} = instruction register, registro che 
	contiene l'istruziuone da eseguire successivamente a quella nel PC.
\item \bt{PC} = program counter, puntatore all'istruzione 
	indirizzo dell'istruzione da eseguire presente nella memoria.
\item \bt{MAR} = memory address register, registro di interfaccia con 
	il bus di sistema, contiene solo registri
\item \bt{MBR} = memory buffer register, contiene solo dati.\\
Il MAR e il MBR mantengono le infromazioni fino a che non è 
disponibile il bus di sistema per essere impiegato.

	\begin{itemize}
		\item in caso di lettura raccolgono il dato dal bus.
		\item in caso di scrittura contengono il dato.
	\end{itemize}
\item \bt{I/O AR} indirizzo periferica con cui scambiare dati, 
	specificare periferica 
\item \bt{I/O BR} raccolta dati 
\end{itemize}
Quando si ha un salto nell'esecuzione delle 
istruzioni incrementa l'indirizzo del PC\\
Inoltre è presente un buffer nel modulo I/O: è una memoria interna al sistema di 
input output, esso serve perchè la CPU invia dati troppo velocemente rispetto 
ed esso non può riceverli alla stessa velocità, per questo 
il buffer dell'I/O, mantiene in memoria i dati inviati dalla più veloce CPU 
\subsection{Ciclo CPU esecuzione}
\begin{itemize}
	\item \bt{Fetch}: reperimento, prelievo dell'istruzione dalla memoria
	\item \bt{Execute}: esecuzione dell'istruzione prelevata dalla memoria
\end{itemize}
Il registro \bt{PC} contiene l'indirizzo di memoria della
cella di Memoria contenente l'istruzione da eseguire.
Quando si ha un prelievo di istruzioni dalla memoria, si ha un incremento del PC.\\
L'istruzione prelevata viene messa in IR poi viene eseguita.


\subsection{Tipi di Operazioni}

\begin{enumerate}
	\item Processore-memoria: trasferimento dati dalla CPU alla Memoria R/W
	\item Processore-I/O: trasferimento dati da CPU a I/O               R/W
	\item Elaborazione dati: operazioni logiche e aritmetiche sui dati
		operazioni della ALU
	\item Controllo: può alterare la sequenza delle istruzioni, per esempio
		il salto
\end{enumerate}

\begin{esem}	
Parola = 16bit\\
Istruzione = 16bit\\
Codici operativi = 4 bit a sinistra = $2^4$ combinazioni = 16\\

\begin{itemize}
	\item 0001 carica in AC (accumulatore) una cella di M
	\item 0010 scrive in M il contenuto di AC
	\item 0101 somma una cella di M ad AC
\end{itemize}
\end{esem}

\subsection{Ciclo di esecuzione}

\begin{enumerate}
	\item Instruction Adress Calculation
	\item Instruction Fetch
	\item Instruction operation decoding
	\item Operand address calculation
	\item Operand Fetch
	\item Data operation
	\item Operand address calculation 
	\item Operand Store
\end{enumerate}
Per l'esecuzione di una singola operazione, bisogna prima eseguire una serie
di altri sottocompiti, che devono essere ben eseguiti. \\
Per questo è essenziale \bt{l'unità di controllo}, per accertarsi che
ogni operazione venga eseguita correttamente e nel giusto ordine.\\
La CPU può eseguire \bt{più operazioni} momentaneamente, mantenendo ogni sua parte attiva
, breve introduzione al concetto di pipeline


\subsection{Interruzioni}
Il meccansimo tramite il quale dei moduli possono interrompere la normale di sequenza di esecuzione. \newline
Il ciclo di esecuzione con interruzioni è differente: al \bt{termine}
di un ciclo si ha il controllo delle interruzioni,

\begin{itemize}
	\item se ce ne sono, esse vengono risolte
	\item se non ce ne sono, il ciclo ricomincia
\end{itemize}

Tipi di interruzioni:
\begin{itemize}
	\item Program, overflow, divisione per zero
	\item Timer, da un timer inerno alla CPU
	\item I/O, termine di un'operazione di I/O
	\item Guasto Hardware
	\end{itemize}
Si interrompe per
\begin{itemize}
	\item efficienza elaborazione
\end{itemize}

Ciclo interruzione:
\begin{itemize}
	\item viene aggiunto al ciclo di esecuzione
	\item la cpu controlla (fetch) le interruzioni pendenti
	\item se non ce ne sono, prende la prossima istruzione
	\item se ce ne sono: 
	\begin{itemize}
		\item sospende esecuzione
		\item salva contesto
		\item imposta il pc all'indirizzo di inizio del 
			programma di gestione
		\item esegue il programma di gestione dell'hardware
		\item rimette il contesto al suo posto e continua 
			il programma interrotto
	\end{itemize}
\end{itemize}
Lunghezza di attesa:

\begin{itemize}
	\item \bt{breve attesa}, tempo di operazione di I/O minore del
		tempo tra due istruzione WRITE
	\item \bt{lunga attesa}, 
\end{itemize}

\subsubsection{Interruzioni multiple}
In caso di \bt{interruzioni multiple}: esistono vari livelli di interruzione e 
differenti tipi di politica di gestione.\\
Esistono interruzioni di alto livello e basso livello, a seconda dell'importanza
che hanno per il funzionamento del sistema.\\
\bt{Politiche di interruzione}:

\begin{itemize}
	\item Disabilitare le interruzioni:
		\begin{itemize}
			\item La CPU \bt{ignora} le altre interruzioni e
				gestisce la prima e "maschera" le altre
			\item Le interruzioni rimangono \bt{pendenti}
			\item vengono gestite nell'\bt{ordine} in cui arrivano
		\end{itemize}
	\item Definire le priorità
		\begin{itemize}
			\item Interruzioni di bassa priorità vengono
				interrotte quando si presentano quelle di 
				alta priorità
			\item Quando è stata gestita la priorità di 
				alto livello viene maneggiata
				quella di basso livello
			\item Si hanno delle interruzioni \bt{annidate}
		\end{itemize}
\end{itemize}

\subsection{Connessioni}
Tutti i componenti \textbf{devono} essere connessi \newline
Esistono vari tipi di connessioni per vari tipi di componenti, il bus collega:
\begin{itemize}
	\item CPU
	\item Memoria 
	\item I/O
\end{itemize}
Composto da 50 a qualche centinaio di linee
\subsection{Bus}
Tutti i dispostivi sono collegati dal bus di sistema \newline
Il bus:
\begin{enumerate}
	\item collega \textbf{2 o più} dispositivi
	\item mezzo trasmissione condiviso
	\item un segnale trasmesso ad un bus è disponibile a tutti i dispositivi
	\item arbitro bus: solo un dispostivo alla volta può trasmettere
	\item varie linee di comunicazione ( trasmettono uno 0 o un 1)
	\item varie linee trasmettono in parallelo numeri binari. 
		Un bus da 8 bit trasmette un dato di 8 bit
\end{enumerate}
Solitamente l'ampiezza del bus corrisponde ad un multiplo della parola.

\subsubsection{Bus di sistema:}
\begin{itemize}
	\item connette cpu, i/o, M
	\item da 50 a qualche centinaio di linee
	\item 3 gruppi di linee
	\begin{enumerate}
		\item bus dati
		\item indirizzi
		\item controllo
	\end{enumerate}
\end{itemize}

\subsubsection{Bus dati:}
\begin{itemize}
	\item trasporta dati o istruzioni
	\item ampiezza --$ > $ efficienza  del sistema
	\begin{itemize}
		\item con poche linee --$>$ maggiori accessi in memoria
	\end{itemize}
\end{itemize}

\subsubsection{Bus indirizzi}
\begin{itemize}
	\item indica sorgente o destinazione dati
	\item l'ampiezza determina la massima quantità di M indirizzabile
\end{itemize}
Deve essere di almeno 1 parola.

\begin{esem}
Architettura a 64 bit = $ 2^{64} - 1 $ indirizzi\\
Il bus deve essere di almeno 64 bit.
\end{esem}


\subsubsection{Bus controllo}
Per controllare accesso e uso di:

\begin{itemize}
	\item \bt{linee dati}
	\item \bt{indirizzi}
	\begin{enumerate}
		\item M write
		\item M read
		\item richiesta bus
		\item bus grant
		\item interrupt request
		\item clock
	\end{enumerate}
\end{itemize}

\subsubsection{Uso del bus}
Con modulo intendiamo una generica componente del computer.\\
se un modulo vuole inviare dati ad un altro:
\begin{itemize}
	\item bus grant
	\item data transfer
\end{itemize}
se un module vuole ricevere dati da un altro:
\begin{itemize}
	\item bus grant
	\item trasferire una richiesta all'altro modulo sulle linee di controllo 
	\item attendere invio dati
\end{itemize}


\subsubsection{Bus singoli e multipli}
\begin{itemize}
	\item singolo bus = ritardo e congestione
	\item vari bus = risoluzione problema
\end{itemize}
Esempio di Bus multiplo:

\begin{itemize}
	\item \bt{connessione punto a punto} fra CPU e cache, il che rende
	il trasferimento dei dati molto più veloce ed efficiente.\\
	\item \bt{Bus di espansione},	si interfaccia con i dispositivi I/O
	\item \bt{Bus ad alta velocità}, si interfaccia con i dispositivi I/O
		e fornisce una trasmissione di dati più rapida 
	\item \bt{Bus di sistema}, fra cache e Memoria principale.
\end{itemize}



\subsection{Temporizzazione}
Coordinazione degli eventi su un bus
\begin{itemize}
	\item Asincrona
		\begin{itemize}
		\item più complessa da implementare
		\end{itemize}
	\item Sincrona
	\begin{itemize}
		\item clock determined events
		\item single clock line, with an alternate sequence of 0 and 1,
			with equal length
		\item single 1-0 sequence is a clock cicle
		\item every device connected to the bus can read the clock line
		\item every event starts at the beginning of a clock cycle
	\end{itemize}
\end{itemize}

\subsubsection{QPI}
\bt{Interconnessione Punto a Punto}, con l'aumentare della velocità dei 
processori è sempre più frequenta una distribuzione di dati densa e veloce.\\
Un bus condiviso però rallenta la comunicazione del processore (bottleneck)
, per questo si adottano delle connessioni punto a punto.\\

\begin{itemize}
	\item Connessioni dirette multiple:\\
		più componenti del sistema godono di connessioni dirette a 
		coppie con altri componenti.
	\item Architettura di protocollo a strati
	\item Trassferimento Dati a pacchetto:\\
		i dati non sono inviati come flussi di dati non elaborato
		ma come una sequenza di pacchetti, essi includono intestazioni 
		di controllo e codici di correzione dell'errore.
\end{itemize}
Livelli di QPI:

\begin{itemize}
	\item \bt{fisico}\\
		la parte hardware, l'unità di trasferimento è di 20 bit (Phit)
		, tutto è sincronizzato da un clock.
	\item \bt{Link}\\
		trasmissione affidabile del controllo del flusso,
		l'unità di trasferimento è di 80 bit (Flit).\\
		Protocollo con pacchetti da 72 (dati) + 8 (codice
		di correzione errore) bit.
		\begin{itemize}
			\item Controllo del flusso:\\
				il mittente non può inviare più dati di quelli
				che il destinatario può ricevere.
			\item Controllo dell'erroreç
				8 bit per rilevare errori di trasmissione 
				sugli altri 72 bit.\\
				In caso di errore il mittente deve re-inviare
				il pacchetto errato.
		\end{itemize}
	\item \bt{Routing}\\
		Struttura per dirigere i pacchetti attraverso essa. Determina il
		percorso che un pacchetto deve seguire.\\
		Supportato da tabelle di instradamento:\\
		\begin{itemize}
			\item definite da software di basso livello che
				contiene delle istruzioni
			\item descrizione percorso che un pacchetto può seguire
			\item utile in sistemi di maggiori dimensione
		\end{itemize}
	\item \bt{Protocollo}\\
		Insieme di regole ad alto livello per lo scambio di pacchetti.
		Un pacchetto è un numero intero di Flits. Il contenuto di 
		un pacchetto è flessibile, per gestire esigenze diverse.\\
		Supporta il protocollo di coerenza della cache, per garantire 
		coerenza fra i contenuti della cache dei core e la memoria 
		principale
\end{itemize}


\subsection{Memoria}
Caratteristiche principali della memoria:

\begin{itemize}
	\item Locazione: processore (cache), interna 
		(principale RAM), esterna (secondaria)
	\item Capacità: dimensione parola (dipende dall
		'architettura), numero di parole
	\item Unità di trasferimento: parola, fra cache e RAM in blocco ( 
		insieme contiguo di parole)
	\item Metodo di accesso:
		\begin{itemize}
			\item sequenziale: accedere prima a tutte le informazioni
				che vengono prima: lento. Utilizzato nei nastri
				magnetici, backup sistemi.
			\item diretto, organizzata in gruppi,
				accesso sequenziale ai gruppi. accesso diretto 
				ad un insieme di informazioni. HDD
			\item casuale, accesso diretto a quella locazione
				di memoria, non importa dove si trova la 
				locazione di memoria dell'informazione. RAM
			\item associativo, informazione non indviduata da 
				indirizzo, ma da una parte dell'informazione.
				CACHE
		\end{itemize}
	\item Prestazioni: tempo di accesso, ciclo e velocità di trasferimento
	\item Modello fisico:
		\begin{itemize}
			\item semiconduttore (ROM)
			\item magnetico, campi magnetici (HDD)
			\item ottico, laser ottico (CD)
			\item magnetico-ottico
		\end{itemize}
	\item Caratteristiche fisiche:
		\begin{itemize}
			\item volatile (cache, quando termina il flusso di 
				corrente elettrica i dati vengono cancellati)
			\item non volatile (hdd, i dati vegnono memorizzati
				permanentemente quando si spegne il computer)
			\item riscrivibile (RAM, HDD, cache)
			\item non riscrivibile (ROM, read only memory).
		\end{itemize}
	\item Organizzazione, se suddivisa in un singolo chip oppure vari,
		più o meno moduli.
\end{itemize}
Generalmente all'aumentare del costo della memoria aumenta la sua
velocità ma diminuisce la sua ampiezza.\\
In ordine decrescente per velocità, costo e crescente per ampiezza:

\begin{itemize}
	\item Registro (1 parola) 16 - 64 bit
	\item Cache (1 indirizzo)
	\item SRAM
	\item DRAM (8gb - 64gb)
	\item SSD (1TB - 8TB)
	\item HDD (16TB-24TB)
	\item CD - DVD-ROM
	\item Nastro (vari PB)
\end{itemize}
Tutte le locazioni di memoria sono suddivise in blocchi. \newline
La memoria è suddivisia in 2 tipi differenti:
\begin{itemize}
	\item Cache la più veloce e suddivisa in diversi livelli
		\begin{itemize}
			\item L1 cache
			\item L2 cache
			\item L3 cache
		\end{itemize}
	\item Ram più lenta della cache ma più capiente
\end{itemize}

La memoria Ram è composta da:
\begin{enumerate}
	\item indirizzo di memoria
	\item blocco di memoria 
\end{enumerate}
Il numero di parole in un blocco è una potenza di 2. \newline
Una parola è composta da 4 byte, possiamo identificare i primi 14 bit come "indirizo" del bit, mentre i restanti 2 come identificativi del bit.

\subsubsection{Gerarchia di memoria}
Un blocco di memoria richiesto dalla CPU può essere presente \textbf{hit} o non presente \textbf{miss} in memoria. (generalmente è presente). \newline
$T_a$: Tempo medio di accesso ad un dato in memoria cache 
\begin{equation} T_a\ =\ T_h \times P_h + T_m(1-P_h) \end{equation}
$T_h$: tempo di accesso ad un dato presnte in cache
$T_m$: tempo medio di accesso ad un dato \textbf{non} in cache (dimensione blocco)
$P_h$: probabilità di hit \newline
\textit{Tecnica generale} \newline
\begin{enumerate}
	\item Suddivisione della memoria centrale in blocchi logici
	\item dimensionamento della cache in multiplo di blocchi
	\item ogni indirizzo emesso dalla cpu
		\begin{itemize}
			\item hit $\iff$ il dato viene fornito immediatamente alla cpu
			\item miss
				\begin{enumerate}
					\item la cache richiede il dato al livello inferiore 
					\item viene posto in cache 
					\item viene fornito alla cpu 
				\end{enumerate}
		\end{itemize}
\end{enumerate}

\begin{defn}[\textbf{associazione diretta / direct mapping}] 
Ogni blocco del livello ineriore può essere allocato solo in una specifica posizione  \textbf{linea/slot} del livello superiore
\begin{enumerate}
	\item \textbf{ILS} = indirizzo di livello superiore
	\item \textbf{ILI} = indirizzo di livello inferiore 
	\item $ILS = ILI\ mod\ N$
\end{enumerate}	

\begin{enumerate}
	\item vantaggi
		\begin{itemize}
			\item semplicità traduzione indirizzo ILI a ILS
			\item determinazione velocità hit o miss
		\end{itemize}
	\item svantaggi
		\begin{itemize}
			\item necessità di contraddistinguere blocco in ILS
			\item swap frequenti per accesso a dati di blocchi adiacenti		    \end{itemize}
\end{enumerate}
\end{defn}

\begin{defn}[\textbf{associazione completa / fully associative}] 
Ogni blocco del livello inferiore può essere posto in qualunque posizione del livello superiore. \newline
Ad una cache di N blocchi viene associata una tabella di N posizioni contenenti il numero di blocco effettivo (tag)
\begin{itemize}
	\item vantaggi: massima efficienza di allocazione
	\item molto tempo per la corrispondenza ILS-ILI e della verifica hit/miss
\end{itemize}
\end{defn}

\begin{defn}[\textbf{associazinone a N-gruppi / N-way set associative}] 
Ogni blocco di un certo insieme di blocchi del livello inferiore può essere allocato liberamente in uno specifico gruppo di blocchi del livello superiore

\begin{esem}
Per una cache di 32 linee con un $N$ equivalente a 2, ogni gruppo avrà 16 linee.
\end{esem}
Questo tipo di associazione è una via di mezzo fra gli altri due tipi. 
La cache composta da $R$ gruppi di $N$ posizioni di blocco, si affiancano $R$ tabelle di $N$ elementi contenenti i tag.
Ha una buona efficienza di allocazione, nonstante abbia una certa complessità
\end{defn}

\begin{defn}[Politiche di rimpiazzo dei blocchi]
Quando si ha un miss, come si decide quale blocco della cache dobbiamo rimpiazzare?
Nell'associazione diretta non ci si pone questo problema, perchè ogni linea della cache corrisponde un blocco della memoria centrale.

\begin{enumerate}
	\item \textit{casuale}, viene occupato lo spazio omogeneamente, facile implementazione
	\item \textit{First-In-First-Out(FIFO)}, il blocco rimasto più a lungo in cache, complicata implementazione
	\item \textit{Least Frequently Used(LFU)}, il blocco con meno accessi, complicata implementazione hardware
	\item \textit{Least Recently Used(LRU)}, il blocco con l'accesso più distante, per preservare quelli accessi più recentemente, implementazione difficile.
\end{enumerate}
A minor quantità di cache si hanno migliori prestazioni con il rimpiazzo LRU. Ad aumentare il livello di cache è sempre meno significativo il miglioramento offerto da queste tecnologie.
\end{defn}

La scrittura dati determina incoerenza tra il blocco in cache e quello nei livelli inferiori
\begin{defn}[write through]
\begin{enumerate}
	\item scrittura contemporanea in cache e livello inferiore
	\item aumento traffico per frequenti scritture nel medesimo blocco, dati coerenti fra blocchi
	\item si ricorre a buffere asincroni verso la memoria
\end{enumerate}
\end{defn}

\begin{defn}[write back]
\begin{enumerate}
	\item scrittura in memoria inferiore differita al rimpiazzo del blocco di cache corrisp.
	\item occore ricordare operazioni di scrittura nel blocco
	\item ottimizazione del traffico tra livelli
	\item periodi di incoerenza
\end{enumerate}
\end{defn}
Occore ricordare che tra memoria centrale (RAM) e cache si passano \textbf{BLOCCHI} e non \textbf{PAROLE}.



\begin{esem}[scenario problematico]
\begin{itemize}
	\item più dispositivi connessi allo stesso bus con cache locale
	\item memoria centrale condivisa
\end{itemize}
Nessun tipo di "write" (through, back) può assicurare coerenza.
\end{esem}
Possibili soluzioni



\begin{itemize}
	\item \textbf{monitoraggio del bus con write through}, controllori intercettano modifiche locazioni condivise
	\item \textbf{trasparenza hardware}, hardware aggiuntivo: modifica a RAM = modifica a cache
	\item \textbf{memoria non cacheable}, solo una porzione è condivisa e non cacheable
\end{itemize}
Si può estendere il discorso fatto a livelli più alti prendendo in considerazione come memorie la ram e la memoria di tipo swap, basata sui dischi di archiviazione. 
\end{document}



































