\documentclass[arch.tex]{subfiles} 

\begin{document}

\chapter{Seconda Parte}
\subsection{Input/Output}%
\label{sub:input_output}
Un computer è collegato a innumerevoli apparecchi i quali

\begin{itemize}
	\item gestiscono quantità di dati differenti
	\item velocità diverse
	\item formati diverse
\end{itemize}
È certo che siano più lenti della CPU e della RAM, per ottenere il massimo delle
prestazioni della propria CPU è necessario che le periferiche esterne siano 
controllate da dei moduli comuni: \bt{moduli di input output}.\\

Vari tipi di dispositivi esterni:
\begin{itemize}
	\item Comprensibili dall'uomo\\
		Monitor, stampanti, ...
	\item Comprensibili dalla macchina
	\item Comunicazione\\
		Modem, router
\end{itemize}

I componenti di un dipositivo esterno:

\begin{itemize}
	\item Logica di controllo\\
		Comunica con il modulo di I/O
	\item Buffer di memoria\\
		immaggazzina memoria, esempio (logistica container)
	\item Transducer\\
		trasforma dati fisici (analogici) in bits o viceversa
\end{itemize}

Componenti di un modulo I/O:

\begin{itemize}
	\item 					
\end{itemize}

\subsection{Tecniche di gestione input/output}%
\label{sub:tecniche_di_gestione_input_output}

\subsubsection{I/O da programma}
Attesa attiva della CPU e si occupa del trasferimento dei dati 
I/O memory mapped o separato

\subsubsection{I/O guidato da interrupt}
Non c'è l'attesa della CPU, essa viene interrota quando il dispositivo è pronto

\subsubsection{Direct Memory Access}
Hardware aggiuntivo

Modulo DMA:

\begin{itemize}
	\item Data Lines
		\begin{itemize}
			\item Data Count
			\item Data Register
			\item Address Register
		\end{itemize}
	\item Address Lines\\
		Address Register
	\item Control Logic
\end{itemize}

Operazioni DMA:

\begin{itemize}
	\item CPU comunica al controllore DMA:

		\begin{itemize}
			\item lettura scrittura
			\item 
		\end{itemize}
	\item 
	\item 
\end{itemize}

Il trasferimento dei dati DMA occupa il bus e lo sottrae alla CPU, si possono
utilizzare due metodi differenti per accedere al canale:

\begin{itemize}
	\item Una parola alla volta, cycle stealing, sottrae di tanto in tanto
		per un solo ciclo
	\item Per blocchi, burst mode, è il metodo più efficace in quanto 
		l'occupazione del bus è un'operazione molto onerosa.
\end{itemize}
a

\section{Rappresentazione binari numeri reali}

\section{Linguaggio Macchina}%
\label{sub:linguaggio_macchina}

Il linguaggio macchina è costituito da:
\begin{itemize}
	\item Insieme istruzioni eseguibili dalla CPU
	\item Tipologia dei dati manipolabili
	\item Tipi di operandi
\end{itemize}
Nessun linguaggio macchina è superiore ad un altro, ed esso dipende dall'ambito
applicativo e dalle funzioni che si devono far svolgere ad esso.\\
Gli elementi dell'istruzione della macchina:

\begin{itemize}
	\item Codice operativo, specifica l'operazione da eseguire
	\item Riferimento all'operando sorgente, specifica l'operando che 
		rappresenta l'input dell'operazione
	\item Riferimento all'operando risultato, dove va messo
		il risultato ottenuto
	\item Riferimento all'istruzione successiva
\end{itemize}

Gli operandi si possono trovare:

\begin{itemize}
	\item Memoria centrale, o ( virtuale ) 
	\item Registri della CPU, ognuno ha il proprio numero identificativo
	\item Dato immediato con l'istruzione
	\item Dipsositivi di I/O 
\end{itemize}

L'istruzione è una sequenza di bit divisa in campi, viene usata una 
rappresentazione simbolica delle configurazioni di bit. Pure gli operandi
hanno la propria rappresentazione simbolica

Il formato di un'istruzione di 16 bit è:

\begin{itemize}
	\item 4 bits, Codice operativo ( opcode ) 
	\item 6 bits, Indirizzo operando, ( operand reference )
	\item 6 bits, Indirizzo operando, ( operand reference ) 
\end{itemize}

I vari tipi di istruzioni sono di:

\begin{itemize}
	\item elaborazione dati, istruzioni aritmetico logiche, nei registri della
		CPU
	\item Immagazinamento dei dati in Memoria o recupero dati dalla Memoria
	\item Trasferimento dati ( I/O ) 
	\item Controllo del flusso del programma		
\end{itemize}
Gli indirizzi necessari per un'istruzione possono essere:

\begin{itemize}
	\item un indirizzo per ogni operando ( 1 o 2 ) 
	\item uno per il risultato
	\item indirizzo dell'istruzione successiva
\end{itemize}
Quindi possono essere al massimo 4, ( cosa molto rara e dispendiosa), ma in genere
sono 1, 2 o 3 per gli operandi/risultati\\
Quando si riferisce ad un solo registro vuol dire che il secondo è implicito ed è 
memorizzato in un registro, per esempio nell'accumulatore.\\
Quando non si riferisce a nesssun indirizzo vuol dire che tutti gli indirizzi sono
impliciti e si sta utilizzando una pila in cui si accumulano i vari indirizzi
utilizzati.\\

\begin{itemize}
	\item Se si utilizzano \bt{meno indirizzi} di conseguenza si eseguiranno
		delle istruzioni più elementari e corte, se si hanno 
		più istruzioni per uni stesso programma porterà ad un tempo 
		di esecuzione più lungo. L'archiettura RISC utilizza questa 
		filosofia, significa infatti \bt{Reduced Instruction Set Computer}
		e si basa sul velocizzare le istruzioni più frequenti.
	\item se si utilizzano \bt{più indirizzi} le istruzioni diventeranno 
		più complesse quindi impiegherà meno tempo per eseguire istruzioni
		più complesse ma userà più potenza per istruzioni semplici.\\
		L'architettura RISC si basa su questo principio 
		\bt{Complex Instruction Set Computer}
\end{itemize}

Cosa comporta progettare un insieme di istruzioni:

\begin{itemize}
	\item Repertorio:\\
		quante e quali operazioni
	\item Tipo di dato:\\
		su quali dati
	\item Formato:\\
		lunghezza, numero indirizzi, dimensione dei campi
	\item Registri:\\
		numero di registri della CPU indirizzabili dalle istruzioni
	\item Indirizzamento:\\
		modo di specificare gli indirizzi degli operandi
\end{itemize}

\subsubsection{Tipi degli operandi}

\begin{itemize}
	\item Indirizzi, rappresentati come interi senza segno\\
	\item Numeri\\
		limite al modulo\\
		limite alla precisione\\
	\item Caratteri ( stringhe ) 
	\item Dati logici, variabili booleane per il controllo del flusso 
		dell'esecuzione.
\end{itemize}

I numeri vengono rappresentati:

\begin{itemize}
	\item Interi ( con la virgola fissa ) 
	\item Virgola Mobile ( Floating point, IEEE754 ) 
	\item Decimali impaccati, nelle operazioni di I/O,
		non efficienti in quanto per rappresentare una cifra decimale
		si utilizzano 4 bit, che vuol dire solo 10 delle 16 configurazioni
		disponibili vengono utilizzate, si utilizza per evitare la 
		conversione.\\
		es: 246 viene rappresentato come:

		\begin{equation}
			246 = \underset{200}{0010}\ \underset{ + 40	}{0100}\
			\underset{ + 6}{0110}
		\end{equation}
\end{itemize}

I caratteri vengono rappresentati in ASCII ( American Standard Code for
Infromation Exchange ) da 7 bit, quindi si hanno in totale 128 configurazioni
disponibili.\\
Si ha di solito 1 bit per i caratteri di controllo.\\
C'è inoltre una versione estesa da 8 bit in cui si possono rappresentare 256 
configurazioni.\\
Dati Logici:

\begin{itemize}
	\item n bit, invece che un singolo dato
	\item manipolare bit separatamente
\end{itemize}

Tipi di dati Intel x86

\begin{itemize}
	\item 8 (byte), 16 (word), 32 (doppia parola), 64 (quadword), 128 (double quadword) bits
	\item L'indirizzamento è per unità di 8 bit (1 byte)
	\item Una double word da 32 bit inizia da un indirizzo divisibile per 4
	\item Non si ha la necessità di allineare gli indirizzi per le strutture dati in memoria
	\item Bisogna allineare i dati per i trasferimenti dati nel bus
\end{itemize}

Ci sono vari tipi di operazioni:

\begin{itemize}
	\item Trasferimento Dati,\\
		comporta specificare: la sorgente (dove si trova il dato), la destinazione (dove andrà messo il d.),
		la lunghezza del dato da trasferire
	\item Aritmetiche,\\
		somma, sottrazione, moltiplicazione, divisione.\\
		I numeri interi hanno sempre il segno, viene utilizzata anche per numeri con la virgola mobile e inoltre
		possono esserci le operazioni di:\\
		incremento (+1), decremento(-1), negazione (inversione del segno, trovare il numero opposto), calcolo valore 
		assoluto
	\item Logiche,\\
		sono operazioni dirette sugli specifici bit, operazioni di: AND, OR, NOT, XOR, EQUAL,
		possono anche essere eseguite parallelamente su tutti i bit di un registro
	\item Conversione
	\item I/O
	\item Sistema
	\item Trasferimento del controllo,\\
		Salto condizionato (branch), per esempio BRE R1, R2, X (salta a X se R1 equivale a R2),
		si ha la necessità di saltare se si deve eseguire più volte una stessa operazione come in 
		un 'for, while' loop, il che dà spazio alla programmazione modulare.\\
		Salto incondizionato, salta alla prossima istruzione, non ha operandi in quanto non si devono verificare 
		delle condizioni.\\
		Chiamata di procedura: una proc. è un pezzo di programma a cui si può dare un nome il che permetto di eseguirlo
		indicandolo con il nome. Questo permette di risparmiare di scrivere codice e di poter affidare a qualcun'altro 
		la scrittura di questo. Dobbiamo avere due istruzioni: la chiamata e il ritorno.\\
		Al fine di memorizzare l'indirizzo di ritorno ci sono 3 luoghi di memorizzazione differenti:

		\begin{itemize}
			\item Registro, non funzionale quando sono presenti dei cicli ricorsivi (ovvero che si 'richiamano')
			\item All'inzio delle procedura chiamata, non funzionali sempre in presenza di cicli ricorsivi
			\item Cima della pila, ovvero si utilizza una porzione di M dove le scritture e le letture
				avvengono sempre in cima. In questo modo si richiamano gli indirizzi che sono stati per ultimi scritti
				nella pila e che si trovano appunto in cima, evitando il problema dei loop ricorsivi.
		\end{itemize}
	\end{itemize}

\subsection{Linguaggio Assembly}%
\label{sub:linguaggio_assembly}
Questo linguaggio è ad un livello più alto rispetto al linguaggio macchina ed è più comprensbile dall'uomo.\\
Gli indirizzi numerici (binario) vengono interpretati come indirizzi simbolici (A, B, ...,Z),\\
I codici operativi diventano simboli (SUB, ADD, BRE).\\
L'assemblatore è un programma che traduce dal linguaggio assembly al linguaggio macchina.\\

\subsubsection{Big / Little Endian}
I bit nella memoria vengono memorizzati in maniera differente in base all'architettura del calcolatore, ovvero:

\begin{itemize}
	\item Big Endian,\\
		I bit più significativi vengono memorizzati prima negli indirizzi, da sinistra a destra
	\item Little Endian,\\
		I bit meno significativi vengono memorizzati negli ultiimi indirizzi, da destra a sinistra.	
\end{itemize}


\subsection{Modi di indirizzamento}%
\label{sub:modi_di_indirizzamento}

Esistono diversi tipi per specificare l'indirizzo degli operandi:

\begin{itemize}
	\item Immediato
	\item Diretto
	\item Indiretto
	\item Registro
	\item Registro indiretto
	\item Spiazzamento
	\item Pila
\end{itemize}


\subsubsection{Immediato}
L'operando è specificato nell'istruzione stessa, (nella parte del campo 
indirizzo)\\
\bt{VANTAGGIO}: non si esegue nessun accesso in Memoria
(operazione molto onerosa, in quanto implica l'occupazione del bus)\\
\bt{SVANTAGGIO}: limitato dalla dimensione del campo indirizzo.\\
Se abbiamo per esempio 6 bit destinati al campo indirizzo, avremo  $ 2^6 $ 
valori diversi.


\subsubsection{Diretto}
Il campo indirizzo contiene l'indirizzo dell'operando in Memoria.\\
\bt{SVANTAGGIO}: un accesso in Memoria (operazione onerosa) e, spazio di indirizzamento 
limitato legato alla grandezza della memoria.


\subsubsection{Indiretto}
Il campo indirizzo contiene l'indirizzo di una cella di Memoria che contiene 
l'indirizzo dell'operando. \\
\bt{VANTAGGIO}: Con parole di lungehzza N si possono indirizzare
$ 2^n $ entità diverse, $ n $ deve essere comunque uguale o inferiore alla
grandezza del campo indirizzo.\\
\bt{SVANTAGGIO}: è che si avrà bisogno di 2 accessi alla memoria.


\subsubsection{Registro}
L'operando si trova in un registro indicato nel campo indirizzo\\
\bt{VANTAGGIO}: pochi bit per l'indirizzamento.\\
\bt{SVANTAGGIO}: limitato dal numero di registri disponibili dal 
tipo di architettura.


\subsubsection{Registro Indiretto}
Si basa sullo stesso principio dell'indirizzamento a registro, l'operando 
si trovan in una cellla di Memoria puntata dal contenuto del registro.
\bt{VANTAGGIO}: si ha solo 1 accesso in memoria a differenza dell'indirizzamento indiretto,
si ha anche un grande spazio di indirizzamento (che dipende dal numero di registri e dalla
lunghezza del campo indirizzo )


\subsubsection{Spiazzamento}
È la combinazione dell'indirizzamento diretto e a registro indiretto. Il campo indirizzo
è suddiviso in due parti:\\
il valore di base ( A )\\
il registro che contiene l'indirizzo di un valore da sommare ad A\\
Una versione di questo tipo è l'indirizzamento \bt{relativo}, 
in cui non si ha un registro casuale ma il Program Counter.\\
L'indirizzamento \bt{registro-base}, in esso A contiene lo spiazzamento, R contiene il 
puntatore all'indirizzo base.\\
L'\bt{indicizzazione} ha in 'A' l'indirizzo base e nel campo registro lo 'spiazzamento',
per indicare gli operandi da un certo punto della memoria in poi indicheremo con 'A' 
questo punto di partenza e per accedere a tutti i dati successivi basta incrementare il contenuto
del campo registro di 1.

\subsection{Stack/Pila}
La pila è una sequenza lineare di locazioni riservate della Memoria, c'è un puntatore (che si 
trova nel registro SP, stack pointer, ha come indirizzo la cima della pila). L'operando
si trova nella cima della pila, si può considerare come un'evoluzione dell'indirizzamento a
registro indiretto.

\subsection{Formato delle istruzioni}%
\label{sub:formato_delle_istruzioni}
Il formato delle istruzioni è come sono scritte le istruzioni date alla macchina, influsice
la struttura dei campi dell'istruzione, include il codice operativo, include uno o più operandi
e generalmente si ha più di un formato per linguaggio macchina.

\subsubsection{Lunghezza delle istruzioni}
È strettamente correlata al formato delle istruzioni, è influenzata e influenza:

\begin{itemize}
	\item La dimensione della Memoria
	\item L'organizzazione della Memoria
	\item Struttura del bus
	\item La complessità della CPU
	\item La velocità della CPU
\end{itemize}
Per sfruttare al meglio tutte le risorse di un calcolatore deve essere un giusto compromesso
fra un repertorio delle istruzioni potente e la necessità di risparmiare spazio ( fisicamente
nella parte hardware della macchina). \\


Esiston due tipi di formati delle istruzioni:
\begin{itemize}
	\item Lungehzza fissa ( fixed length ),\\
		esempio: PDP-8, PDP-10
	\item Formati a lunghezza variabile o ibrida ( hybrid/variable length ),\\
		esempio: PDP-11, VAX, Intel x86, questo tipo aggiunge complessità
		al realizzazione della macchina, ma può renderla più potente per certi 
		utilizzi specifici.
\end{itemize}

\subsubsection{Allocazione dei bit}
Come i bit vengono allocati dipende dai diversi tipi di indrizzamnento usati da una determinata
architettura, dal numero variabile degli operandi ( 0, 1, 2 ),
il numero dei registri, dei banchi registri ( tipi di 'buffer' di memoria ).
Dipende anche dall'intervallo degli indirizzi ( ogni quanto viene ripetuto un indirizzo ) e dalla
loro granularità ( se sono a byte o parola ), l'indrizzamento a byte è più oneroso
ma utile per la manipolazione dei caratteri.

\subsection{Approfondimento sul funzionamento della CPU}%
\label{sub:approfondimento_sul_funzionamento_della_cpu}

La CPU, essendo il componente hardware più potente della macchina è la parte più importante,
essa ha i compiti di:

\begin{itemize}
	\item Prelevare le istruzioni ( Instruction Fetch )
	\item Interpretare le istruzioni ( Instruction Decode ) 
	\item Prelevare Dati ( Operand Fetch ) 
	\item Elaborare Dati ( Execute ) 
	\item Scrivere Dati ( Write Back ) 
\end{itemize}
Le componenti principali della CPU sono:

\begin{itemize}
	\item ALU, l'unità aritmetico logica, il 'cervello' della CPU
	\item Registri, la memoria della CPU
	\item Unità di Controllo, le parti che verificano il corretto funzionamento del
		processore
\end{itemize}
La CPU è inoltre collegata al bus di sistema per poter interagire con gli altri componenti 
dell'elaboratore


\subsubsection{Registri}
I registri sono uno 'spazio' di lavoro in cui la CPU può memorizzare i dati che ha 
elaborato senza dover interagire con la Memoria Principale, in quanto questa è un'operazione
molto onerosa e che si deve cercare di limitare al più possibile.\\
I registri sono al vertice della gerarchia di memoria. Essi possono avere funzioni diverse
determinate dall'impianto progettuale della CPU.\\

I tipi di Registri sono:

\begin{itemize}
	\item Utente, vengono utilizzati dal 'programmatore' per memorizzare internamente
		i dati alla CPU e successivamente da elaborare.
	\item di Controllo e di Stato, utilizzati dall'unità di controllo per monitorare
		le operazioni svolte dalla CPU, sono anche utilizzati dai programmi del sistema
		operativo per controllare l'esecuzione dei programmi.
\end{itemize}
Con 'programmatore' ci si riferisce a:

\begin{itemize}
	\item L'umano che programma in assembly, ( che poi viene trasformato in codice macchina
		dall'assemblatore)
	\item Il compilatore che produce un codice in assembly da un programma in HLL 
		( Linguaggio ad Alto Livello ) 
\end{itemize}
I registri che sono visibili all'utente sono:

\begin{itemize}
	\item Uso generale ( general purpose ),\\
		possono essere ad uso generale o dedicati a particolari funzioni, possono
		memorizzare indirizi, dati
	\item memorizzazione dei dati
	\item memorizzazione di indirizzi
	\item memorizzazione dei codici di condizione
\end{itemize}
Inoltre la memoria principale può essere organizzata logicamente come un insieme di
'segmenti' ( spazi di indirizzamento multipli ).\\
Un segmento al suo interno contiene locazioni di memoria indirizzabili, inoltre si può indicare
all'interno della memoria fisica la 'base' del segmento ( dove comincia ) e la sua
'lunghezza' ( quanto è lungo effettivamente ) 

\paragraph{Registri ad uso generale:}
Questi registri si possono dividere in due sottocategorie:

\begin{itemize}
	\item Effettivamente ad uso generale,\\
		aumentano la flessibilità e le opzioni disponibili al 'programmatore',\\
		aumentano le dimensioni dell'istruzione e della sua complessità, in quanto necessitano
		di uno spazio separato nel formato dell istruzioni che può variare la lunghezza 
		in base al numero di registri.
	\item Specializzati,\\
		le loro istruzioni sono più piccole e veloci,\\
		a discapito di un'inferiore flessibilità
\end{itemize}
In genere il numero dei registri generali varia da 8 a 32, se ce ne fossero meno di 8 si avrebbe un maggior
numero di accessi in memoria principale ( operazione onerosa ), se fossero più di 32 non si limiterebbe
l'accesso alla memoria e aggiungerebbe molta complessità alla struttura della CPU.\\
Tuttavia l'architettura RISC arriva ad utilizzare fino a centinaia di registri.\\

\paragraph{Lunghezza dei registri:}
Generalmente un registro è lungo abbastanza da poter contenere un indirizzo della memoria principale e 
da contenere una 'full word'

\paragraph{Registri per memorizzazione di Codici di Condizione:}
Si tratta dell'insieme di bit individuali che possono essere letti implicitamente da un programma
non possono essere impostati da un programma

\paragraph{Registri di Controllo e di Stato:}
Sono registri che abbiamo già incontrato:

\begin{itemize}
	\item Progrma Counter ( PC ), il registro ceh tiene il 'conto' delle istruzioni da svolgere
	\item Instruction Register ( IR ), il registro che contiene le istruzioni da eseguire
	\item Memory address register ( MAR ), il registro degli indirizzi di memoria
	\item Memory Buffer Register ( MBR ), un registro di buffer, ovvero una memoria temporanea interna al
		processore
\end{itemize}

\paragraph{Program Status Word:}
La PSW è un insieme di bit che include codici di condizione fra cui:\\
il segno dell'ultimo risultato, zero, riporto, uguale, overflow, abilitazione/disabilitazione interrupt, supervisore

\paragraph{Modo Supervisore:}
Si tratta di una modalità che permette al Sistema Operativo di utilizzare le procedure del Kernel, che agiscono su componenti
critiche del sistema, ovvero l'esecuzione di istruzioni privilegiate. Essa è disponibile SOLAMENTE al sistema operativo e non 
all'utente o al programmatore in assembler.

\subsubsection{Ciclo di esecuzione CPU con indirettezza:}
Al ciclo di esecuzione della CPU nella fase di fetch e di execute hanno un sottociclo che è l'indirettezza.\\
Al fine di recuperare gli operandi indicati in un'istruzione può essere necessario accedere più volte in memoria
secondo la modalità di indirzzamento indiretto.

\paragraph{Flusso dei dati}%
\label{par:flusso_dei_dati}
L'instruction Fetch generalmente si suddivide in queste operazioni:

\begin{itemize}
	\item L'indirizzo dell'istruzione successiva 'x' è contenuta nel PC;
	\item L'indirizzo dell'istruzione 'x' viene spostato nel MAR;
	\item L'indirizzo 'x' viene emesso nel bus indirizzi;
	\item L'unità di controllo richiede una lettura nella memoria principale;
	\item Viene letto l'indirizzo nella memoria principale;
	\item L'indirizzo ottenuto viene inviato tramite il bus dati e viene 
		ricevuto e copiato dal MBR;
	\item L'indirizzo viene spostato nell'IR
	\item Viene incrementato il PC con l'indirizzo dell'istruzione 'x+1'
\end{itemize}


\paragraph{Data Fetch}%
\label{par:data_fetch}
Il ciclo del data fetch comprende:

\begin{itemize}
	\item Viene esaminato l'IR;
	\item Se il codice operativo ( opcode ) dell'istruzione richiede un indirizzamento indiretto
		si esegue il ciclo di indirettezza:
		\begin{itemize}
			\item N bit più a destra del MBR vengono trasferiti nel MAR;
			\item L'unità di controllo richiede la lettura dalla memoria principale;
			\item Il risultato della lettura, l'indirizzo dell'operando, viene trasferito
				nel MBR;
		\end{itemize}
\end{itemize}

\paragraph{Execute}%
\label{par:execute}
Questa parte del ciclo può variare molto in base al tipo di architettura, dipende dal tipo 
di istruzioni che bisogna eseguire e può includere le operazioni di:

\begin{itemize}
	\item Lettura/scrittura della Memoria;
	\item Input/Output;
	\item Trasferimento di dati fra registri e/o nei registri;
	\item Operazioni della ALU;
\end{itemize}

\paragraph{Interrupt}%
\label{par:interrupt}
Esso è semplice e prevedibile si svolge come segue:

\begin{itemize}
	\item Viene salvato il contenuto del PC, al fine di recuperare il ripristino dell'esecuzione dopo
		la gestione dell'interruzione;
	\item Il PC viene caricato con l'indirizzo della prima istruzione della routine di gestione 
		dell'interruzione;
	\item Il fetch dell'istruzione viene puntato al PC;
\end{itemize}

\paragraph{Prefetch}%
\label{par:prefetch}

È la fase di prelievo dell'istruzione e accede alla memoria principale, quest'istruzione viene di solito
eseguita durante la fase di esecuzione dell'istruzione corrente ( durante questa fase non si deve accedere in memoria ).\\
Questa procedura può diventare inutile se sono in esecuzione delle istruzioni di 'jump' o 'branch' che vanno
a modificare il contenuto delle istruzioni da eseguire.
\end{document}
